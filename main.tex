\input{book preambles}

\setcounter{chapter}{-1}

\input{commands}

\begin{document}

    \title{Techniques of algebraic geometry}
    
    \author{Dat Minh Ha}
    \maketitle
    
    {
      \hypersetup{} 
      \dominitoc
      \tableofcontents %sort sections alphabetically
    }

    \listoftodos

    \input{introduction}
    
    \part{Some scheme theory}
        \chapter{Schemes, algebraic spaces, and algebraic stacks}
            \begin{abstract}
                
            \end{abstract}
            
            \minitoc

            \section{Schemes}
    \subsection{The Zariski topology and affine schemes}
        \subsubsection{The prime spectrum of a commutative ring} \index{$\Spec$}
            \begin{definition}[Spectra of commutative rings]
                To any commutative ring $R$, let us \textit{contravariantly functorially} associate, first of all, a set $\Spec R$ consisting of all prime ideals of $R$. This is called the spectrum, or the prime spectrum, of $R$. In other words, $\Spec$ is a functor from $\Cring^{\op}$ to $\Sets$.
            \end{definition}
            \begin{remark}[Why is $\Spec$ a contravariant functor ?]
                One can very well construct the theory of affine schemes based on an alternative \textit{covariant} functor $"\Spec": \Cring \to \Sets$, which assigns to a commutative ring its set of prime ideals too. However, this would not have made our lives easy, as the images under ring homomorphisms of a prime ideal is not necessarily prime, whereas the preimage under ring homomorphisms of a prime ideal is always prime. In particular, this means that requiring $\Spec$ to be a contravariant functor ensures that to-be morphisms between affine schemes would always exist, given that the correpsonding morphism between commutative rings exists. In fact, this also proves that $\Spec$ is a well-defined functor, as it guarantees that each commutative ring homomorphism $f: A \to B$ is sent to a \textit{function} $\Spec f$ that maps each element $\q \in \Spec B$ to a unique element $(\Spec f)(\q)$ of $\Spec A$.
            \end{remark}
            \begin{example}[Some interesting underlying sets of ring spectra] \label{example: spectra_sets}
                \noindent
                \begin{enumerate}
                    \item \textbf{(Spectra of fields):} If $k$ is any field, then $\Spec k = \{(0)\}$. To see why this is the case, first of all, let $I$ be an ideal of $k$ that is neither $(0)$ nor $k$. We know that ideals are closed under linear combinations; in this particular instance, $I$ is closed under $k$-linear combinations. Thus, if we view $k$ as a vector space over itself, then $I$ must be a non-zero proper subspace of $k$, since $I$ is a subset of $k$ that is closed under $k$-linear combinations (one could also argue that by the first isomorphism theorem, $I$ is the kernel of some ring homomorphism whose domain is $k$, and we know that kernels are subspaces). Either way, this would mean that:
    					$$0 = \dim_k 0 < \dim_k I < \dim_k k = 1$$
    				and since dimensions of vector spaces are natural numbers, there will be no such natural number $\dim_k I$, i.e. $I$ does not exist. Note that the zero ideal $(0)$ is trivially the zero subspace $0$ of $k$.
    				\item \textbf{(Spectrum of the zero ring):} $\Spec 0 = \varnothing$, because prime ideals are defined to be proper.
    				\item \textbf{(The zero ideal):} The zero ideal is not necessarily prime; as a matter of fact, it is only so inside an integral domain. When zero-divisors are present, say in $\Z/n\Z$ with $n$ composite, the statement:
    				    $$xy \in (0)$$
				    might not imply that $x = y = 0$, but instead, that $x \in (p)$ and $y \in (q)$, with $p, q$ integers such that $pq = n$.
    				\item \textbf{(The complex affine line):} Due to the algebraic closure of $\bbC$, the points of the complex affine line $\A^1_{\bbC} := \Spec \bbC[x]$ (with $t$ some formal variable) are prime ideals of the form $(x - a)$ wherein $a \in \bbC$, along with the prime ideal $(0)$. To see in more details why this is the case, let us first recall because $\bbC$ is a separably closed field, every single-variable polynomial over $\bbC$ splits completely into linear factor. This, along with the fact that $\bbC[x]$ is a PID, tells us that ideals of $\bbC[x]$ are actually all contained in \textit{principal} ideals generated by linear polynomials; note that these principal ideals are prime, precisely because $\bbC$ is separably closed. Lastly. because $\bbC$ is algebraically closed, there is a bijective correspondence between the principal ideals generated by linear polyonomials $(x - a)$ and the individual complex numbers $a \in \bbC$. Thus, the prime ideals of $\bbC[x]$ are either of the form $(0)$ or $(x - a)$, wherein $a \in \bbC$. 
    				\\
    				Below is an illustration by Ravi Vakil of the complex affine line:
    				    \begin{figure}[H]
    				        \centering
    				        \includegraphics[width=\linewidth,height=\textheight,keepaspectratio]{Figures/complex affine line.png}
    				        \caption{The complex affine line $\A^1_{\bbC}$ (\cite{risingsea}, figure 3.1)}
    				        \label{fig: complex_affine_line}
    				    \end{figure}
    				This result generalises in an obvious manner to algebraically closed fields other than $\bbC$; so for instance, studying schemes over the field $\overline{\Q}$ of algebraic numbers might help one understand more about polynomials with rational coefficients.
    				\item \textbf{(The affine line over a separably closed but not algebraically closed field):} Let $k$ be a field that is separably closed but not algebraically closed (we can take $k = \F_p(t)^{\sep}$, for example). Then, the set of non-zero prime ideals of $\A^1_k$ need not be in bijection with $k$ itself.
    				\item \textbf{(Spectrum of the integers):} The prime ideals of $\Z$ are either generated by prime numbers themselves, or the zero ideal $(0)$. Thus, the set $\Spec \Z$ is in bijection with the \textit{union} of the set of all prime numbers and the set $\{(0)\}$. 
                \end{enumerate}
            \end{example}
        
        \subsubsection{The Zariski topology as a point-set topology} \index{Topology!Zariski}
            \begin{definition}[Zariski-closed subsets] \label{def: zariski_closed}
                Let $R$ be a commutative ring and let $\calF$ be an arbitrary subset of $R$. Then, let us declare that sets of the following form are closed in the to-be Zariski topology:
                    $$V(\calF) := \left\{\p \in \Spec R \mid \p \supset \calF \right\}$$
                When it might be possible to confuse Zariski-closed subsets of different ring spectra, we will write $V_R(\calF)$ instead of simply $V(\calF)$.
            \end{definition}
            \begin{remark}
                Of course, sets of the form $\Spec R \setminus V(\calF)$ are Zariski-open (that is, if we are assuming that the \href{https://ncatlab.org/nlab/show/excluded+middle}{\underline{the Law of Excluded Middle}} holds).
            \end{remark}
            
            \begin{proposition}[Well-definiteness of the Zariski topology] \label{prop: zariski_closed_well_definiteness}
                Let $R$ be an arbitrary commutative ring and assume the Law of Excluded Middle. Then, Zariski-closed subsets as defined in \ref{def: zariski_closed} actually define a topology on $\Spec R$, which of course, is called the Zariski topology.
            \end{proposition}
                \begin{proof}
                    For each subset $\calF$ of $R$, let us write $I(\calF)$ for the $R$-ideal generated by $\calF$. Let us now verify the axioms defining topologies on sets one-by-one.
                        \begin{enumerate}
                            \item \textbf{(The empty set and the whole set are closed):} The empty set is just $V(R)$ and that $\Spec R$ is just $V\left((0)\right)$, which are, by definition, closed in the Zariski topology. Thus, both the empty set and the whole space are Zariski-closed.
                            \item \textbf{(Finite unions of closed sets are closed):} Let $\{V(\calF_{\alpha})\}_{\alpha \in A}$ be a \textit{finite} set of Zariski-closed subsets of $\Spec R$ and consider the following chain of logical \textit{implications} (wherein $\p$ is a prime ideal of $R$, even though this fact can be inferred from the statements themselves):
                                $$
                                    \begin{aligned}
                                        & \p \in \bigcup_{\alpha \in A} V(\calF_{\alpha})
                                        \\
                                        \iff & \exists \alpha \in A: \p \in V(\calF_{\alpha})
                                        \\
                                        \iff & \exists \alpha \in A: \p \supset \calF_{\alpha}
                                        \\
                                        \iff & \bigvee_{\alpha \in A} (\p \supset \calF_{\alpha})
                                        \\
                                        \implies & \forall \left(f_{\alpha}\right)_{\alpha \in A} \in \prod_{\alpha \in A} \calF_{\alpha}: \prod_{\alpha \in A} f_{\alpha} \in \p
                                        \\
                                        \iff & \bigwedge_{\left(f_{\alpha}\right)_{\alpha \in A} \in \prod_{\alpha \in A} \calF_{\alpha}} \left(\prod_{\alpha \in A} f_{\alpha} \in \p\right)
                                        \\
                                        \iff & \p \supset \left\{\prod_{\alpha \in A} f_{\alpha} \: \bigg| \: \forall \alpha \in A: f_{\alpha} \in \calF_{\alpha} \right\}
                                        \\
                                        \iff & \p \in V\left(\left\{\prod_{\alpha \in A} f_{\alpha} \: \bigg| \: \forall \alpha \in A: f_{\alpha} \in \calF_{\alpha} \right\}\right)
                                    \end{aligned}
                                $$
                            wherein the fourth line, in particular, holds due to the fact that ideals, by definition, are closed under scalar multiplication by elements of their ambient rings. Now, to upgrade the fourth line to an equivalence, we can show that:
                                $$\bigwedge_{\left(f_{\alpha}\right)_{\alpha \in A} \in \prod_{\alpha \in A} \calF_{\alpha}} \left(\prod_{\alpha \in A} f_{\alpha} \in \p\right) \implies \bigvee_{\alpha \in A} (\p \supset \calF_{\alpha})$$
                            or, as we have assumed that the Law of Excluded Middle holds, we have the following:
                                $$
                                    \begin{aligned}
                                        & \left(\bigwedge_{\left(f_{\alpha}\right)_{\alpha \in A} \in \prod_{\alpha \in A} \calF_{\alpha}} \left(\prod_{\alpha \in A} f_{\alpha} \in \p\right) \implies \bigvee_{\alpha \in A} (\p \supset \calF_{\alpha})\right)
                                        \\
                                        \vdash & \left(\neg \bigwedge_{\left(f_{\alpha}\right)_{\alpha \in A} \in \prod_{\alpha \in A} \calF_{\alpha}} \left(\prod_{\alpha \in A} f_{\alpha} \in \p\right) \implies  \neg \bigvee_{\alpha \in A} (\p \supset \calF_{\alpha})\right)
                                    \end{aligned}
                                $$
                            meaning that we can prove the contraposition instead. To that end, consider the following:
                                $$
                                    \begin{aligned}
                                        & \neg \bigwedge_{\left(f_{\alpha}\right)_{\alpha \in A} \in \prod_{\alpha \in A} \calF_{\alpha}} \left(\p \ni \prod_{\alpha \in A} f_{\alpha}\right)
                                        \\
                                        \implies & \bigvee_{\left(f_{\alpha}\right)_{\alpha \in A} \in \prod_{\alpha \in A} \calF_{\alpha}} \neg \left(\p \ni \prod_{\alpha \in A} f_{\alpha}\right)
                                        \\
                                        \implies & \bigwedge_{\alpha \in A} \left(\bigvee_{f_{\alpha} \in \calF_{\alpha}} \neg(\p \ni f_{\alpha})\right)
                                        \\
                                        \implies & \bigwedge_{\alpha \in A} \left(\neg \bigwedge_{f_{\alpha} \in \calF_{\alpha}} (\p \ni f_{\alpha})\right)
                                        \\
                                        \implies & \bigwedge_{\alpha \in A} \neg (\p \supset \calF_{\alpha})
                                        \\
                                        \implies & \neg \bigvee_{\alpha \in A} (\p \supset \calF_{\alpha})
                                    \end{aligned}
                                $$
                            Thus, we have managed to show that:
                                $$\neg \bigwedge_{\left(f_{\alpha}\right)_{\alpha \in A} \in \prod_{\alpha \in A} \calF_{\alpha}} \left(\prod_{\alpha \in A} f_{\alpha} \in \p\right) \implies  \neg \bigvee_{\alpha \in A} (\p \supset \calF_{\alpha})$$
                            and therefore:
                                $$\p \in \bigcup_{\alpha \in A} V(\calF_{\alpha}) \iff \p \in V\left(\left\{\prod_{\alpha \in A} f_{\alpha} \: \bigg| \: \forall \alpha \in A: f_{\alpha} \in \calF_{\alpha} \right\}\right)$$
                            Because $\p$ was chosen arbitrarily, this implies that:
                                $$\bigcup_{\alpha \in A} V(\calF_{\alpha}) = V\left(\left\{\prod_{\alpha \in A} f_{\alpha} \: \bigg| \: \forall \alpha \in A: f_{\alpha} \in \calF_{\alpha} \right\}\right)$$
                            Hence, the \textit{finite} union $\bigcup_{\alpha \in A} V(\calF_{\alpha})$ is Zariski-closed by definition, and consequently, all finite unions of Zariski-closed sets are closed in the Zariski topology themselves (since $\{\calF_{\alpha}\}_{\alpha \in A}$ is an arbitrary fintie set of subsets of $R$). Note that the finiteness assumption on the index set $A$ is crucial, as without it, one would not be able to properly make sense of the product $\prod_{\alpha \in A} f_{\alpha}$.
                            \item \textbf{(Intersections of closed sets are closed):} Let $\{V(\calF_{\alpha})\}_{\alpha \in A}$ be an \textit{arbitrary} set of Zariski-closed subsets of $\Spec R$ and consider the following chain of logical equivalences (wherein $\p$ is a prime ideal of $R$, and again, this fact can be deduced from the statements themselves):
                                $$
                                    \begin{aligned}
                                        & \p \in \bigcap_{\alpha \in A} V(\calF_{\alpha})
                                        \\
                                        \iff & \forall \alpha \in A: \p \in V(\calF_{\alpha})
                                        \\
                                        \iff & \forall \alpha \in A: \p \supset \calF_{\alpha}
                                        \\
                                        \iff & \p \supset I\left(\bigcup_{\alpha \in A} \calF_{\alpha}\right)
                                        \\
                                        \iff & \p \in V\left(I\left(\bigcup_{\alpha \in A} \calF_{\alpha}\right)\right)
                                    \end{aligned}
                                $$
                            It tells us that \textit{any} prime ideal $\p$ is in an intersection of Zariski-closed subsets of $\Spec R$ defined by subsets $\calF_{\alpha}$ of $R$ if and only if it is in the Zariski-closed subset of $\Spec R$ defined by the ideal generated by the union of the sets $\calF_{\alpha}$, or in other words, that:
                                $$\bigcap_{\alpha \in A} V(\calF_{\alpha}) = V\left(I\left(\bigcup_{\alpha \in A} \calF_{\alpha}\right)\right)$$
                            In turn, this implies that the intersection of the Zariski-closed sets $V(\calF_{\alpha})$ is itself closed in the Zariski topology, and since the index set $A$ is was chosen arbitrarily, this means that arbitrary intersections of Zariski-closed sets are themselves Zariski-closed.
                        \end{enumerate}
                    Thus, with closed sets as in definition \ref{def: zariski_closed}, the Zariski topology on prime spectra of commutative rings is well-defined.
                \end{proof}
            \begin{corollary}[Quotients are closed] \label{coro: quotients_are_closed}
                Let $R$ be a commutative ring and let $I$ be an $R$-ideal. Then, $\Spec R/I$ is homeomorphic to a Zariski-closed subset of $\Spec R$. 
            \end{corollary}
                \begin{proof}
                    This comes from a straightforward application of the third isomorphism theorem for modules.
                \end{proof}
                
            \begin{definition}[A different approach: Zariski-open sets] \label{def: zariski_open}
                Let $R$ be a commutative ring. Then, let us declare that subsets of $\Spec R$ of the following form are open in the to-be Zariski topology:
                    $$D(f) := \{\p \in \Spec R \mid \p \not \ni f\}$$
                Whenever referring to more than one ring spectra, it might be beneficial to specifically write $D_R(f)$ instead of $D(f)$.
            \end{definition}
            \begin{remark} \label{remark: basic_opens_complements}
                For any commutative ring $R$, one can show through the following logical equivalences that:
                    $$D(f) = \Spec R \setminus V\left((f)\right)$$
                wherein $\p$ is an arbitrary prime ideal of $R$:
                    $$
                        \begin{aligned}
                            & \p \in D(f)
                            \\
                            \iff & \p \in \{\q \in \Spec R \mid \q \not \ni f\}
                            \\
                            \iff & \neg(\p \ni f)
                            \\
                            \iff & \neg(\p \supset (f))
                            \\
                            \iff & \p \in \Spec R \setminus \{\q \in \Spec R \mid \q \supset (f)\}
                            \\
                            \iff & \p \in \Spec R \setminus V\left((f)\right)
                        \end{aligned}
                    $$
            \end{remark}
            
            \begin{proposition}[Well-definiteness of the Zariski topology] \label{prop: zariski_open_well_definiteness}
                Let $R$ be an arbitrary commutative ring and assume the Law of Excluded Middle. Then, Zariski-open subsets as defined in \ref{def: zariski_open} actually define a topology on $\Spec R$, which of course, is called the Zariski topology.
            \end{proposition}
                \begin{proof}
                    Let us verify the axioms defining topologies on sets one-by-one.
                        \begin{enumerate}
                            \item \textbf{(The empty set and the whole set are open):} Consider the set $D(1)$, which by definition, is given by:
                                $$D(1) := \{\p \in \Spec R \mid \p \not \ni 1\}$$
                            Because prime ideals are defined to be proper, and because proper ideals are never (multiplicatively) unital, one gets that:
                                $$D(1) = \Spec R$$
                            In other words, the whole of $\Spec R$ is open by definition. Now, consider the following:
                                $$D(0) := \{\p \in \Spec R \mid \p \not \ni 0\} = \varnothing$$
                            which holds because ideals are submodules of their ambient rings, and modules over rings must contain $0$ (as an additive identity) by definition. Thus, the empty set is also Zariski-open by definition. 
                            \item \textbf{(Unions of open sets are open):} Let $\{f_{\alpha}\}_{\alpha \in A}$ be an \textit{arbitrary} set of elements of $R$ and let us apply remark \ref{remark: basic_opens_complements} to get the following chain of logical equivalences regarding the union of the sets $D(f_{\alpha})$, wherein $\p$ is an \textit{arbitrary} prime ideal of $R$:
                                $$
                                    \begin{aligned}
                                        & \p \in \bigcup_{\alpha \in A} D(f_{\alpha})
                                        \\
                                        \iff & \bigvee_{\alpha \in A} \left(\p \in D(f_{\alpha})\right)
                                        \\
                                        \iff & \bigvee_{\alpha \in A} \neg \left(\p \in V\left((f_{\alpha})\right)\right)
                                        \\
                                        \iff & \neg \bigwedge_{\alpha \in A} \left(\p \in V\left((f_{\alpha})\right)\right)
                                        \\
                                        \iff & \p \in \Spec R \setminus \bigcap_{\alpha \in A} V\left((f_{\alpha})\right)
                                    \end{aligned}
                                $$
                            This shows that:
                                $$\bigcup_{\alpha \in A} D(f_{\alpha}) = \Spec R \setminus \bigcap_{\alpha \in A} V\left((f_{\alpha})\right)$$
                            In proposition \ref{prop: zariski_closed_well_definiteness}, we have already shown using only definition \ref{def: zariski_closed} that arbitrary intersections of Zariski-closed sets are Zariski-closed themselves; in particular, this means that $\bigcap_{\alpha \in A} V\left((f_{\alpha})\right)$ is Zariski-closed. Then, by using the Law of Excluded Middle, one can see that the complement $\Spec R \setminus \bigcap_{\alpha \in A} V\left((f_{\alpha})\right)$ is necessarily Zariski-open. Thus, arbitrary unions of Zariski-open sets are Zariski-open themselves.
                            \item \textbf{(Finite intersections of open sets are open):} Let $\{f_{\alpha}\}_{\alpha \in A}$ be an \textit{finite} set of elements of $R$ and consider the following chain of logical equivalences regarding the union of the sets $D(f_{\alpha})$, wherein $\p$ is a prime ideal of $R$:
                                $$
                                    \begin{aligned}
                                        & \neg \left(\p \in \bigcap_{\alpha \in A} D(f_{\alpha})\right)
                                        \\
                                        \iff & \neg \bigwedge_{\alpha \in A} \left(\p \in D(f_{\alpha})\right)
                                        \\
                                        \iff & \bigvee_{\alpha \in A} \neg \left(\p \in D(f_{\alpha})\right)
                                        \\
                                        \iff & \bigvee_{\alpha \in A} \left(\p \in \Spec R \setminus D(f_{\alpha})\right)
                                        \\
                                        \iff & \bigvee_{\alpha \in A} \left(\p \in V\left((f_{\alpha})\right)\right)
                                        \\
                                        \iff & \p \in \bigcup_{\alpha \in A} V\left((f_{\alpha})\right)
                                    \end{aligned}
                                $$
                            (let us note that the fifth line holds thanks to remark \ref{remark: basic_opens_complements}). Now, the contraposition of the equivalence:
                                $$\neg \left(\p \in \bigcap_{\alpha \in A} D(f_{\alpha})\right) \iff \p \in \bigcup_{\alpha \in A} V\left((f_{\alpha})\right)$$
                            is:
                                $$\neg \neg \left(\p \in \bigcap_{\alpha \in A} D(f_{\alpha})\right) \iff \neg \left(\p \in \bigcup_{\alpha \in A} V\left((f_{\alpha})\right) \right)$$
                            From this, one gets the following proof:
                                $$
                                    \begin{aligned}
                                        & \neg \neg \left(\p \in \bigcap_{\alpha \in A} D(f_{\alpha})\right) \iff \neg \left(\p \in \bigcup_{\alpha \in A} V\left((f_{\alpha})\right) \right)
                                        \\
                                        \vdash & \left(\p \in \bigcap_{\alpha \in A} D(f_{\alpha})\right) \iff \left(\p \in \Spec R \setminus \bigcup_{\alpha \in A} V\left((f_{\alpha})\right) \right) 
                                        \\
                                        \vdash & \left(\bigcap_{\alpha \in A} D(f_{\alpha}) = \Spec R \setminus \bigcup_{\alpha \in A} V\left((f_{\alpha})\right)\right)
                                    \end{aligned}
                                $$
                            Lastly, let us recall that by \ref{prop: zariski_closed_well_definiteness}, the \textit{finite} union $\bigcup_{\alpha \in A} V\left((f_{\alpha})\right)$ of Zariski-closed sets is Zariski-closed itself, meaning that by the Law of Excluded Middle, the complement $\Spec R \setminus \bigcup_{\alpha \in A} V\left((f_{\alpha})\right)$ must be Zariski-open. Thus, the union $\bigcap_{\alpha \in A} D(f_{\alpha})$ is Zariski-open. Note that the finiteness assumption on the index set $A$ is crucial, as otherwise, the union $\bigcup_{\alpha \in A} V\left((f_{\alpha})\right)$ might not be Zariski-closed.
                        \end{enumerate}
                    Thus, with open sets as in definition \ref{def: zariski_open}, the Zariski topology on prime spectra of commutative rings is well-defined.
                \end{proof}
            
            \begin{proposition}[Unifying the two definitions] \label{prop: zariski_topology_equivalence}
                By asuming the Law of Excluded Middle, one gets the same topology on spectra of commutative rings via the approaches presented in definitions \ref{def: zariski_closed} and \ref{def: zariski_open}.
            \end{proposition}
                \begin{proof}
                    Let $R$ be a commutative ring. It is sufficient to show that for each element $f \in R$, the complement $\Spec R \setminus D(f)$ is closed in the sense of definition \ref{def: zariski_closed}, or equivalently, for each subset $\calF \subset R$, the complement $\Spec R \setminus V(\calF)$ is open in the sense of definition \ref{def: zariski_open}. We will be attempting the second approach. To that end, let us directly the following chain of logical equivalences:
                        $$
                            \begin{aligned}
                                & \p \in \bigcup_{\alpha \in A} D(f_{\alpha})
                                \\
                                \iff & \exists \alpha \in A: \p \in D(f_{\alpha})
                                \\
                                \iff & \exists \alpha \in A: \p \in \{\q \in \Spec R \mid \q \not \ni f_{\alpha}\}
                                \\
                                \iff & \exists \alpha \in A: \p \not \ni f_{\alpha}
                                \\
                                \iff & \bigvee_{\alpha \in A} \neg\left(\p \ni f_{\alpha}\right)
                                \\
                                \iff & \neg \bigwedge_{\alpha \in A} (\p \ni f_{\alpha}) 
                                \\
                                \iff & \neg \left(\p \supset \bigcup_{\alpha \in A} \{f_{\alpha}\}\right)
                                \\
                                \iff & \neg \left(\p \supset \{f_{\alpha}\}_{\alpha \in A}\right)
                                \\
                                \iff & \p \in \Spec R \setminus \left\{\q \in \Spec R \mid \q \supset \{f_{\alpha}\}_{\alpha \in A}\right\}
                                \\
                                \iff & \p \in \Spec R \setminus V\left(\{f_{\alpha}\}_{\alpha \in A}\right)
                            \end{aligned}
                        $$
                    Thus:
                        $$\bigcup_{\alpha \in A} D(f_{\alpha}) = \Spec R \setminus V\left(\{f_{\alpha}\}_{\alpha \in A}\right)$$
                    i.e. the complement of the Zariski-closed set $V\left(\{f_{\alpha}\}_{\alpha \in A}\right)$ inside $\Spec R$ is a union of Zariski-open sets $D(f_{\alpha})$, which we know from proposition \ref{prop: zariski_open_well_definiteness} to be Zariski-open itself. As stated, this implies that definitions \ref{def: zariski_closed} and \ref{def: zariski_open} give us the same Zariski topology on prime spectra of commutative rings.
                \end{proof}
            \begin{corollary} \label{coro: zariski_basis}
                Let $R$ be a commutative ring and let $f$ denote elements of $R$. Then, the distinguished Zariski-open sets $D(f)$ form a base of the Zariski topology on $\Spec R$.
            \end{corollary}
                \begin{proof}
                    This is a direct consequence of the fact that complements of Zariski-closed sets are unions of Zariski-open sets of the form $D(f)$.
                \end{proof}
            
            We have managed to show that on each ring spectrum, there is a canonical topology, namely the Zariski topology. A naturaly follow-up question is thus: can we upgrade $\Spec$ to a functor whose domain is $\Top$ instead of $\Sets$ ? Luckily, the answer is yes, although we will need to do some work to show that this is the case. 
            \begin{proposition}[Continuous functions between spectra] \label{prop: continuous_functions_between_spectra}
                By equipping prime spectra of commutative rings with the Zariski topology (in either the sense of definition \ref{def: zariski_closed} or \ref{def: zariski_open}), one naturally gets a functor:
                    $$\Spec: \Cring^{\op} \to \Top$$
                assigning to commutative rings their respective Zariski topological spaces.
            \end{proposition}
                \begin{proof}
                    It will suffice to show that given any ring homomorphism $f: A \to B$, the induced map $\Spec f: \Spec B \to \Spec A$ is continuous, which we can do by showing that preimages of Zariski-open subsets of $\Spec A$ under $\Spec f$ are closed in $\Spec B$; in fact, we can restrict our attention to \textit{basic} open subsets of $\Spec A$ (i.e. subsets of the form $D_A(a)$, for some $a \in A$), as they form a basis for the Zariski topology on $\Spec A$. Let $D_A(a)$ be such a basic open set. It preimage under $\Spec f$ is thus the following subset of $\Spec B$:
                        $$(\Spec f)^{-1}\left(D_A(a)\right) = \{\q \in \Spec B \mid (\Spec f)(\q) \in D_A(a)\}$$
                    Writing out the definition of $D_A(a)$ (cf. definition \ref{def: zariski_open}) then gives the following chain of logical equivalences:
                        $$
                            \begin{aligned}
                                & \q \in (\Spec f)^{-1}\left(D_A(a)\right)
                                \\
                                \iff & (\Spec f)(\q) \in D_A(a)
                                \\
                                \iff & f^{-1}(\q) \in D_A(a)
                                \\
                                \iff & \neg(f^{-1}(\q) \ni a)
                                \\
                                \iff & \neg(\q \ni f(a))
                                \\
                                \iff & \q \in D_B(f(a))
                            \end{aligned}
                        $$
                    which proves that:
                        $$(\Spec f)^{-1}\left(D_A(a)\right) = D_B(f(a))$$
                    and because $D_B(f(a))$ is open by definition, so is the preimage $(\Spec f)^{-1}(D_A(a))$. As stated at the beginning, this implies that $\Spec f$ is a continuous function, and thus there exists a functor:
                        $$\Spec: \Cring^{\op} \to \Top$$
                    assigning commutative rings and homomorphisms between them to ring spectra equipped with the Zariski topology and continuous maps in between.
                \end{proof}
                
            \begin{example}[Topologically interesting ring spectra]
                \noindent
                \begin{enumerate}
                    \item \textbf{(The complex affine plane and complex affine $n$-spaces):}
                        \begin{figure}[H]
                            \centering
                            \includegraphics[width=\linewidth,height=\textheight,keepaspectratio]{Figures/complex affine plane.png}
                            \caption{The complex affine plane $\A^2_{\bbC}$ (\cite{risingsea}, figure 3.3)}
                            \label{fig: complex_affine_plane}
                        \end{figure}
                    \item \textbf{(Revisiting $\Spec \Z$):}
                        \begin{figure}[H]
                            \centering
                            \includegraphics[width=\linewidth,height=\textheight,keepaspectratio]{Figures/Spec Z.png}
                            \caption{$\Spec \Z$ (\cite{risingsea}, figure 3.2)}
                            \label{fig: Spec_Z}
                        \end{figure}
                    \item \textbf{(A conic):}
                        \begin{figure}[H]
                            \centering
                            \includegraphics[width=\linewidth,height=\textheight,keepaspectratio]{Figures/conic.png}
                            \caption{A conic which is Zariski-closed inside $\A^3_{\bbC}$ (\cite{risingsea}, figure 3.4)}
                            \label{fig: conic}
                        \end{figure}
                \end{enumerate}
            \end{example}
            
            \begin{remark}[Comparing $\Spec$ and $\Spm$]
                Historically (and only because mathematicians were more interested in complex algebraic geometry back in the days), it was not the set of prime ideals of a commutative ring that was considered, but rather, the set of \textit{maximal} ideals. This was not out of \textit{na\"ivet\'e}, though. Maximal ideals enjoy being closed points in prime spectra of commutative rings (one can prove this by first looking at varieties $V(\m)$ associated to maximal ideals $\m$ of some commutative ring $R$, and then applying the definition of the (underlying set of) these varieties as spaces whose points are prime ideals containing $\m$, and then lastly, applying the usual definition of topological closures; as a corollary, one gets that prime ideals that are not maximal get sent by $\Spec$ to non-closed points in $\Spec R$), and so doing geometry with them is a lot more intuitive (albeit more restrictive as well) then doing so with all prime ideals. For instance, the underlying set of $\Spm \bbC[x]$ is precisely $\bbC$, whereas that of $\Spec \bbC[x]$ can be thought of as $\bbC \cup \{\infty\}$, i.e. as the Riemann sphere; in particular, the zero ideal $(0)$ corresponds to the point \say{at infinity}, which we denote by $\infty$. 
            \end{remark}
            
            \begin{example}[Non-isomorphic rings with homeomorphic spectra] \label{example: nonisomorphic_rings_with_the_same_spectra}
                The following examples are of non-isomorphic rings with homeomorphic prime spectra; through them, we are able to show that the functor $\Spec: \Cring^{\op} \to \Top$ is not an equivalence of categories (nor even a fully faithful inclusion). 
                \begin{enumerate}
                    \item \textbf{(Fields):} The prime spectra of any field is just the one-point space, but clearly, not all fields are isomorphic.
                    \item \textbf{(Discrete valuation rings):} The spectrum of any \href{https://en.wikipedia.org/wiki/Discrete_valuation_ring}{\underline{discrete valuation ring}} is homeomorphic to the \href{https://ncatlab.org/nlab/show/Sierpinski+space}{\underline{Sierpi\'nski space}} (to see why this is the case, firstly check that discrete valuation rings only have two prime ideals, one being the zero ideal and one being the unique maximal ideal, and that the latter is a closed point in the spectrum whereas the former is generic), but of course, not all discrete valuation rings are isomorphic to one another.
                \end{enumerate}
            \end{example}
        
        \subsubsection{Affine schemes}
            Next, we will be discussing the idea of so-call \textbf{structure sheaves}, but in order to make sense of these entities, we will need to know what $\C$-valued sheaves are for categories $\C$ more general than $\Sets$:
            \begin{definition}[$\C$-valued sheaves] \label{def: C_valued_sheaves}
                \noindent
                \begin{enumerate}
                    \item \textbf{($\C$-valued sheaves):} Let $(\S, J)$ be a site\footnote{... which is not necessarily small, as cases such as $\S \cong \Top$ and $\S \cong \Mfd^{\smooth}_{/\R}$ are interesting in their own rights.} and let $\C$ be a category with \textit{enough small limits} and \textit{enough filtered colimits} (the purpose of the second hypothesis is to ensure that stalks, should they exist, are well-defined); note that $\C$ need not be small. Additionally, fix an \textit{arbitrary} object $x$ of $\S$ along with a covering sieve $\calU_{/x} \in J$ thereon. Also, let $j: \S \to \Psh_{\C}(\S)$ be the Yoneda embedding. Then, a \textbf{$\C$-valued sheaf} on $(\S, J)$ is a functor $\calF: \S^{\op} \to \C$ such that $\calF(x) \cong \calF\left( \underset{u \in \calU_{/x}}{\colim} ju \right)$.
                    \item \textbf{($\C$-topoi):} $\C$-valued sheaves on a given site $(\S, J)$ form a category in the obvious manner. We shall be writing $\Sh_{\C}(\S, J)$ for this category, and such categories will be called \textbf{$\C$-topoi}, even though this is an abuse of terminology.
                \end{enumerate}
            \end{definition}
            \begin{example}[Sheaves of rings]
                The notion of sheaves of rings, which subsumes that of structure sheaves (cf. proposition \ref{prop: structure_sheaf}), follows suite from definition \ref{def: C_valued_sheaves}. Note that such constructions are well-defined, as the category of rings is both complete and cocomplete.
            \end{example}
            
            Having defined sheaves that might take values categories other than $\Sets$, let us now try to define affine schemes as locally ringed spaces whose underlying topological spaces are spectra of commutative rings, and whose structure presheaves have a certain condition imposed upon them, which happens to guarantee that:
                \begin{enumerate}
                    \item these structure presheaves are indeed sheaves (proposition \ref{prop: structure_sheaf}) with local stalks (corollary \ref{coro: structure_sheaf_properties}), and
                    \item they are unique (proposition \ref{prop: structure_sheaf_uniqueness}), which is an important feature, because ringed spaces are uniquely defined by their structure sheaves; this fact will also be used to establish the fully faithfulness of $\Spec$ as a functor from $\Cring^{\op}$ to the category $\Loc\Ringed\Spc$ of locally ringed spaces. 
                \end{enumerate}
            Our efforts will culminate in definition \ref{def: affine_schemes}.
                
            \begin{proposition}[Structure sheaves of affine schemes] \label{prop: structure_sheaf} \index{Structure sheaves}
                Let $k$ be a base commutative ring, and let $\calO_{\Spec R}$ be \textit{a} presheaf of commutative rings on $\Ouv(\Spec R)$ determined by the following rule on objects:
                    $$\calO_{\Spec R}(D_R(f)) \cong R_f$$
                for all element $f \in R$. Any presheaf on $\Ouv(\Spec R)$ that are defined this way is a Zariski sheaf (i.e. a sheaf on the site $\Ouv(\Spec R)$ of Zariski-open subsets of $\Spec R$), and is called \textit{a} \textbf{structure sheaf} on $\Spec R$.
            \end{proposition}
            \begin{corollary}[On the locality of stalks] \label{coro: structure_sheaf_properties}
                Let $R$ be a commutative ring and let $\p$ be an arbitrary prime ideal of $R$. Then one has the following characterisation of the stalk $\calO_{\Spec R, \p}$ at $\p$ of the structure sheaf $\calO_{\Spec R}$:
                    $$\calO_{\Spec R, \p} \cong R_{\p}$$
                This shows that affine schemes are, in fact, \textit{locally} ringed spaces and not just ringed spaces. 
            \end{corollary} 
                \begin{proof}
                    Recall that the stalk $\calF_x$ of a sheaf (of sets) $\calF$ on a topological space $(X, \Ouv(X))$ is given by the filtered colimit indexed by the poset of open neighbourhoods of the chosen point $x \in X$:
                        $$\calF_x \cong \underset{U \in \{V \in \Ouv(X) \mid V \ni x\}}{\colim} \calF(U)$$
                    By adapting this definition to the underlying Zariski-topological spaces of affine schemes, we get that:
                        $$\calO_{\Spec R, \p} \cong \underset{U \in \{V \in \Ouv(\Spec R) \mid V \ni \p\}}{\colim} \calO_{\Spec R}(U)$$
                    with $\Ouv(\Spec R)$ the Zariski topology defined via open sets as in definition \ref{def: zariski_open}. In corollary \ref{coro: zariski_basis}, we have already seen how the distinguished Zariski-open sets defined in definition \ref{def: zariski_open} form a basis for the Zariski topology on commutative ring spectra, and so the above filtered colimit can be rewritten as:
                        $$\calO_{\Spec R, \p} \cong \underset{D_R(f) \in \{V \in \Ouv(\Spec R) \mid V \ni \p\}}{\colim} \calO_{\Spec R}\left(D_R(f)\right)$$
                    and because $D_R(f) = \{\p \in \Spec R \mid \p \not \ni f\}$, one subsequently gets:
                        $$\calO_{\Spec R, \p} \cong \underset{f \in \{g \in R \mid g \not \in \p\}}{\colim} \calO_{\Spec R}\left(D_R(f)\}\right)$$
                    Lastly we have the following isomorphism:
                        $$\calO_{\Spec R, \p} \cong \underset{f \in \{g \in R \mid g \not \in \p\}}{\colim} \calO_{\Spec R}\left(D_R(f)\right) \cong \underset{f \in R \setminus \p}{\colim} R_f \cong R_{\p}$$
                    Thus $\calO_{\Spec R, \p} \cong R_{\p}$ as claimed.
                \end{proof}
                
            \begin{proposition}[Uniqueness of structure sheaves] \label{prop: structure_sheaf_uniqueness}
                Let $R$ be a commutative ring. Then, there is only one unique structure sheaf attached to $\Spec R$. 
            \end{proposition}
                \begin{proof}
                    Suppose to the contrary that there exist two \textit{distinct} Zariski sheaves of $R$-algebras on ${}^{R/}\Comm\Alg^{\op}$ $\calF$ and $\calG$ such that:
                        $$\forall f \in R: \calF(\Spec R_f) \cong \calG(\Spec R_f) \cong R_f$$
                    However, the localisation of any commutative at its multiplicative identity is just itself, and so:
                        $$\calF(\Spec R) \cong \calG(\Spec R) \cong R$$
                    for all commutative rings $R$. This means that the functors $\calF$ and $\calG$ are naturally isomorphic, i.e. they can not be distinct. Thus, the structure sheaf attached to a given ring spectrum is unique (up to natural isomorphisms, of course).
                \end{proof}
                
            \begin{example}[Spotting structure sheaves in the wild]
                Let $R$ be a discrete valuation ring that is a \href{https://en.wikipedia.org/wiki/Dedekind_domain}{\underline{Dedekind domain}} (so the only proper ideals of $R$ would be $(0)$ and its unique maximal ideal) with unique maximal ideal $\p$, and recall that its spectrum is (homeomorphic to) the Sierpi\'nski space (see example \ref{example: nonisomorphic_rings_with_the_same_spectra} for more details); in particular, the subset $\{(0)\}$ of $\Spec R = \{(0), \p\}$ is the only non-empty open proper subset. Now, suppose that $\calF$ is a Zariski sheaf on $\Spec R$ given by the following formula:
                    $$
                        \calF(U) \cong 
                        \begin{cases}
                            \text{$R$ if $U = \Spec R$}
                            \\
                            \text{$\Frac R$ if $U = \{(0)\}$}
                        \end{cases}
                    $$
                (note that discrete valuation rings are integral domains, so it makes sense to consider their fields of fractions). The point that is to be made here is that $\calF$ qualifies as a structure sheaf on $\Spec R$. To see why this is the case, note that because $R$ has only two prime ideals, namely $(0)$ and $\p$, 
            \end{example}
            
            \begin{example}[The complex affine line]
                Recall that in example \ref{example: spectra_sets}, we have seen how a point of the complex affine line $\A^1_{\bbC}$ is either the zero ideal, or of the form $(t - a)$ for any complex number $a$; one should keep the following picture in mind:
                    \begin{figure}[H]
				        \centering
				        \includegraphics[width=\linewidth,height=\textheight,keepaspectratio]{Figures/complex affine line.png}
				        \caption{The complex affine line $\A^1_{\bbC}$ (\cite{risingsea}, figure 3.1)}
				        \label{fig: complex_affine_line_stalks}
				    \end{figure}
			    \noindent
			    Now, as an affine scheme, $\A^1_{\bbC}$ comes equipped with a structure sheaf $\calO_{\A^1_{\bbC}}$, whose stalks, as shown in corollary \ref{coro: structure_sheaf_properties}, are precisely the localisations of $\bbC[t]$ at its prime ideals. There are thus two cases:
			        \begin{enumerate}
			            \item The stalk at $(0)$ is given by:
			                $$\calO_{\A^1_{\bbC}, (0)} \cong \bbC[t]_{(0)} \cong \bbC(t)$$
		                and thus the residue field is trivially $\bbC(t)$.
			            \item At non-zero primes, the stalks of the structure sheaf $\calO_{\A^1_{\bbC}}$ are given by the following localisations:
			                $$\calO_{\A^1_{\bbC}, (t - a)} \cong \bbC[t]_{(t - a)}$$
		                whose elements we note to be fractions of the form $\frac{f(t)}{g(t)}$ whose denominators do not vanish at $t = a$. Now, recall that the localisation of any commutative ring at a prime ideal is a local ring, and that inside \textit{any} local commutative ring, elements in the complement of the unique maximal ideal are units; in particular, these facts imply that the elements of the complement $\bbC[t]_{(t - a)} \setminus (t - a)\bbC[t]_{(t - a)}$ are all invertible. Consequently, these elements must be fractions $\frac{f(t)}{g(t)}$ whose numerators and denominators both do not vanish at $t = a$. Thus, a reasonable description of the canonical quotient map is the evaluation map:
		                    $$\frac{f(t)}{g(t)} \mapsto \frac{f(a)}{g(a)}$$
	                    whose image is precisely $\bbC$. Therefore, the stalks of $\calO_{\A^1_{\bbC}}$ are all isomorphic to $\bbC$.
			        \end{enumerate}
            \end{example}
        
            \begin{definition}[Affine schemes] \label{def: affine_schemes}
                An \textbf{affine scheme} is a locally ringed space that is isomorphic to one of the form $(|\Spec R|, \calO_{\Spec R})$ for some commutative ring $R$. Morphisms of affine schemes are morphisms of locally ringed spaces, and as such, one has a full subcategory $\Sch^{\aff}$ of affine schemes within the category of locally ringed spaces. 
            \end{definition}
            
            \begin{theorem}[Isbell Duality for locally ringed spaces] \label{theorem: isbell_duality_for_locally_ringed_spaces}
                There is an adjunction as follows:
                    $$
                        \begin{tikzcd}
                        	{\Cring^{\op}} & \Loc\Ringed\Spc
                        	\arrow[""{name=0, anchor=center, inner sep=0}, "\Spec"', bend right, from=1-1, to=1-2]
                        	\arrow[""{name=1, anchor=center, inner sep=0}, "\Gamma"', bend right, from=1-2, to=1-1]
                        	\arrow["\dashv"{anchor=center, rotate=-90}, draw=none, from=1, to=0]
                        \end{tikzcd}
                    $$
            \end{theorem}
            \begin{corollary}
                The adjunction from theorem \ref{theorem: isbell_duality_for_locally_ringed_spaces} restricts down to an adjoint equivalence $\Sch^{\aff} \cong \Cring^{\op}$.
            \end{corollary}

    \subsection{The category of schemes}
        \subsubsection{Schemes}
            \begin{definition}[Schemes] \label{def: schemes}
                A \textbf{scheme} is a locally ringed space $(|X|, \calO_X)$ such that every point $x \in |X|$ as a Zariski-open neighbourhood $U_x \ni x$ that is isomorphic to an affine scheme. Morphisms of schemes are nothing but morphisms of locally ringed spaces, meaning that schemes form a category $\Sch$ which embeds fully faithfully into the category of locally ringed spaces.
            \end{definition}
            
            \begin{proposition}[Open subschemes are open locally ringed subspaces] \label{prop: open_subschemes_are_open_locally_ringed_subspaces}
                Let $X$ be a scheme and let $U \subseteq X$ be an open locally ringed subspace. Then, $U$ will be an open subscheme of $X$. 
            \end{proposition}
                \begin{proof}
                    
                \end{proof}
            \begin{corollary}[Zariski-bases of schemes] \label{coro: zariski_bases_of_schemes}
                
            \end{corollary}
    
        \subsubsection{Properties of schemes and their morphisms}
        
        \subsubsection{Topologies on schemes; descent-theoretic results}
    
    \subsection{Varieties}

    \subsection{Cohomology of schemes and derived categories of coherent sheaves}

            \input{Algebraic geometry/Schemes, algebraic spaces, and algebraic stacks/algebraic_spaces}

            \input{Algebraic geometry/Schemes, algebraic spaces, and algebraic stacks/algebraic_stacks}
        
        \chapter{Quasi-coherent modules}
            \begin{abstract}
                
            \end{abstract}
            
            \minitoc

            \input{Algebraic geometry/Quasi-coherent modules/6_functors_for_QCoh}

            \section{Some applications}
    \subsection{Coherent sheaves and perfect complexes; cohomology of projective spaces}
        \begin{definition}[Coherent modules] \label{def: coherent_modules}
            Let $X$ be a scheme with a fixed Zariski covering:
                $$\{ f_i: \Spec A_i \to X \}_{i \in I}$$
            The category $\Coh(X)$ of \textbf{coherent $\scrO_X$-modules} will then be the full subcategory of $\QCoh(X)$ consisting of finite-type objects, which is to say that $\scrM \in \Ob( \QCoh(X) )$ is coherent if and only if $\Gamma(\Spec A_i, f_i^*\scrM)$ is a finite $A_i$-module for every $i \in I$.
        \end{definition}
        Even though we just gave a definition of coherent sheaves on a general scheme, we will usually be considering coherent $\scrO_X$-modules over \textit{locally Noetherian} schemes $X$, for the same reason that finitely generated modules are usually considered over Noetherian rings. 
        \begin{proposition}[Coherent modules over (locally) Noetherian schemes] \label{prop: coherent_modules_over_noetherian_schemes}
            Let $X$ be a locally Noetherian scheme. Then, an $\scrO_X$-module $\scrM$ will be coherent if and only if it is finitely presented.
        \end{proposition}
        \begin{example}
            Let $X$ be a locally Noetherian scheme. Then the structure sheaf $\scrO_X$ itself is coherent. 
        \end{example}
        \begin{example}[Incoherent rings]
            If $X$ is a scheme such that $\scrO_X$ is not coherent as a module over itself, then naturally, we shall refer to $\scrO_X$ as an \textbf{incoherent sheaf of rings}. There is the following example of such a ring, due to Brian Conrad: \href{http://math.stanford.edu/~vakil/216blog/incoherent.pdf}{http://math.stanford.edu/~vakil/216blog/incoherent.pdf}.
        \end{example}

    \subsection{Zariski's Main Theorem}
        \begin{theorem}[The Theorem of Formal Functions] \label{theorem: formal_function_theorem}

        \end{theorem}
            \begin{proof}
                
            \end{proof}

        \begin{theorem}[Zariski's Main Theorem] \label{theorem: zariski_main_theorem}
            
        \end{theorem}
            \begin{proof}
                
            \end{proof}

        \begin{theorem}[Stein's Factorisation Theorem] \label{theorem: stein_factorisation}
            
        \end{theorem}

    \subsection{The archimedean GAGA principle of Serre}
        \begin{convention}
            In this subsection, we work exclusively over $\bbC$. 
        \end{convention}

        \begin{lemma}[Complex-analytic completions of finite-type $\bbC$-algebras] \label{lemma: complex_analytic_completions_of_finite_type_C_algebras}
            There is an \textbf{analytification functor}:
                $$(-)^{\an}: \bbC\-\Comm\Alg^{\ft} \to \bbC\-\Ban\Comm\Alg^{\locconvex}$$
            from the category of finite-type $\bbC$-algebras to that of locally convex Banach $\bbC$-algebras, sending objects $A$ of the former to objects $A^{\an}$ of the former, given as the complex-analytic completion of $A$. Furthermore, this functor is compatible with base-change in the sense that, if:
                $$\phi: A \to B$$
            is a homomorphism of finite-type $\bbC$-algebras then:
                $$B^{\an} \cong (A \tensor_{A, \phi} B)^{\an} \cong A^{\an} \hattensor_{A, \phi} B$$
        \end{lemma}
            \begin{proof}
                
            \end{proof}
        \begin{remark}[Associated complex-analytic topological spaces] \label{remark: associated_complex_analytic_topological_spaces}
            Let $X$ be a finite-type $\bbC$-scheme. Since $\bbC$ is algebraically closed, closed points of $X$ are in bijection with $X(\bbC)$ by Hilbert's \textit{Nullstellensatz}. At the same time, $X(\bbC)$ - by definition - consists of all solutions to the system of polynomials in $\bbC^n$ or $\P^{n, \an}_{\bbC}$ cut out by $X$ (say, $\dim X \leq n$), so $X(\bbC)$ naturally inherits the subspace topology from the complex-analytic space $\bbC^n$ or $\P^{n, \an}_{\bbC}$. In this sense, we have that:
                $$X^{\an} := X(\bbC)$$
            is the natural complex-analytic topological space associated to the underlying topological space of $X$.

            The identification of closed points of $X$ gives rise to a natural continuous embedding:
                $$i_X: X^{\an} \to X$$
        \end{remark}
        In order to obtain complex-analytic locally ringed spaces associated to locally finite-type $\bbC$-schemes, we need also to somehow complex-analytically complete the structure sheaves of these schemes. 
        \begin{proposition}[Complex-analytic completions of locally finite-type $\bbC$-schemes] \label{prop: complex_analytic_completions_of_locally_finite_type_C_schemes}
            There is an \textbf{analytification functor}:
                $$(-)^{\an}: \Sch_{/\Spec \bbC}^{\lft} \to \An\Spc$$
                $$(X, \scrO_X) \mapsto (X^{\an}, \scrO_{X^{\an}})$$
            from the category of locally finite-type $\bbC$-schemes to that of complex-analytic spaces, determined by:
                $$\scrO_X^{\an} := (i_X^*\scrO_X)^{\an}$$
            with notations as in remark \ref{remark: associated_complex_analytic_topological_spaces}. Furthermore, if:
                $$f: X \to Y$$
            is a morphism of locally finite-type $\bbC$-schemes then we will obtain a pullback square:
                $$
                    \begin{tikzcd}
                    {X^{\an}} & {Y^{\an}} \\
                    X & Y
                    \arrow["{f^{\an}}", from=1-1, to=1-2]
                    \arrow["f", from=2-1, to=2-2]
                    \arrow["{i_X}"', from=1-1, to=2-1]
                    \arrow["{i_Y}", from=1-2, to=2-2]
                    \arrow["\lrcorner"{anchor=center, pos=0.125}, draw=none, from=1-1, to=2-2]
                    \end{tikzcd}
                $$
        \end{proposition}
            \begin{proof}
                
            \end{proof}
        \begin{lemma}[(Faithful) flatness of complex-analytifications] \label{lemma: flatness_of_complex_analytifications}
            Let $A$ be an arbitrary finite-type $\bbC$-algebra. Then $A^{\an}$ will be flat over $A$. 
        \end{lemma}
            \begin{proof}
                
            \end{proof}
        \begin{corollary} \label{coro: flatness_of_complex_analytifications}
            For any locally finite-type $\bbC$-scheme $X$, the pullback functor:
                $$i_X^*: \scrO_X\mod \to \scrO_{X^{\an}}\mod$$
            is exact. 
        \end{corollary}
        
        \begin{remark}
            Consider a morphism:
                $$f: X \to Y$$
            between schemes locally of finite type over $\Spec \bbC$. This gives rise to a pullback square of locally ringed spaces:
                $$
                    \begin{tikzcd}
                    {X^{\an}} & {Y^{\an}} \\
                    X & Y
                    \arrow["{f^{\an}}", from=1-1, to=1-2]
                    \arrow["f", from=2-1, to=2-2]
                    \arrow["{i_X}"', from=1-1, to=2-1]
                    \arrow["{i_Y}", from=1-2, to=2-2]
                    \arrow["\lrcorner"{anchor=center, pos=0.125}, draw=none, from=1-1, to=2-2]
                    \end{tikzcd}
                $$
            as stated in proposition \ref{prop: complex_analytic_completions_of_locally_finite_type_C_schemes}, which in turn gives rise to the following cohomological base-change map/spectral sequence:
                $$Li_Y^* \circ R f_* \Rightarrow R f^{\an}_* \circ Li_X^*$$
            but since the functors $i_X^*, i_Y^*$ are exact \textit{a priori} (cf. corollary \ref{coro: flatness_of_complex_analytifications}), this reduces down to a cohomological comparison map as follows:
                $$i_Y^* \circ R f_* \Rightarrow R f^{\an}_* \circ i_X^*$$
            The existence of such a natural transformation allows us to formulate a version of proper base-change in this context, which yields us cohomological comparison isomorphisms that interpolate between the Zariski and complex-analytic topologies on locally finite-type $\bbC$-schemes and their associated analytic spaces respectively (see corollary \ref{coro: GAGA_cohomological_comparison}). 
        \end{remark}
        \begin{theorem}[Relative analytification of sheaves of modules] \label{theorem: relative_analytification_of_sheaves_of_modules}
            Suppose that:
                $$f: X \to Y$$
            is a morphism between schemes locally of finite type over $\Spec \bbC$. If $f$ is proper, then the canonical natural transformation:
                $$i_Y^* \circ R f_* \Rightarrow R f^{\an}_* \circ i_X^*$$
            will be a natural isomorphism of t-exact functors $D^+(\scrO_X\mod) \to D^+(\scrO_{Y^{\an}}\mod)$.
        \end{theorem}
            \begin{proof}
                This is a result of proper base-change for general ringed spaces (see \cite[\href{https://stacks.math.columbia.edu/tag/09V4}{Tag 09V4}]{stacks} for now). 
            \end{proof}
        \begin{corollary}[GAGA cohomological comparison] \label{coro: GAGA_cohomological_comparison}
            When $Y \cong \Spec \bbC$ (and hence $i_Y^*$ is just the identity functor), there are isomorphisms of $\bbC$-vector spaces:
                $$H^{\bullet}(X, \scrM) \cong H^{\bullet}(X^{\an}, i_X^*\scrM)$$
            In turn, this implies that the pullback functor:
                $$i_X^*: \scrO_X\mod \to \scrO_{X^{\an}}\mod$$
            is fully faithful on top of being exact, and since both of its domain and codomain are abelian categories, this implies in particular that short exact sequences are reflected. 
        \end{corollary}

        We shall now see that when we restrict our attention to coherent modules only, the module analytification functor $i_X^*$ (for any locally finite-type $\bbC$-scheme $X$) will furthermore be essentially surjective, thus giving rise to an adjoint equivalence:
            $$i_X^*: \Coh(X) \leftrightarrows \Coh(X^{\an}): i_{X *}$$
        between the categories of coherent modules on $X$ and on $X^{\an}$, with the latter being given as the category of coherent modules on the ringed space $(X^{\an}, \scrO_{X^{\an}})$ (cf. \cite[\href{https://stacks.math.columbia.edu/tag/01BU}{Tag 01BU}]{stacks}). The fact that:
            $$i_X^*: \scrO_X\mod \to \scrO_{X^{\an}}\mod$$
        is fully faithful means that, should its restriction down to coherent modules:
           $$i_X^*: \Coh(X) \to \Coh(X^{\an})$$
       be well-defined (cf. lemma \ref{lemma: absolute_analytification_of_coherent_modules}) then we will be able to exploit the compact generation of $\Coh(X)$ to see that the set of (compact) generators via finite colimits of $\Coh(X^{\an})$ contains that of $\Coh(X)$ as a subset. The proof of essential surjectivity then reduces down to a proof of essential surjectivity on these generators. 
        \begin{remark}[A few reminders on coherent modules on ringed spaces]
            Again, we refer the reader to \cite[\href{https://stacks.math.columbia.edu/tag/01BU}{Tag 01BU}]{stacks} for a more detailed discussion, but the following list of properties is important enough for our purposes to warrant at least a mention. Suppose for a moment that $(X, \scrO_X)$ is a general ringed space and write $\Coh(X)$ to denote the category of coherent $\scrO_X$-modules.
            \begin{itemize}
                \item $\Coh(X)$ is an abelian subcategory of $\scrO_X\mod$, and this is somewhat interesting, seeing how $\QCoh(X)$ is not generally even abelian for ringed spaces, unlike for schemes where quasi-coherent modules are extremely well-behaved (cf. theorem \ref{theorem: qcoh_homological_properties})
                \item In fact, $\Coh(X)$ is closed under all extensions/short exacct sequences, and thus is a Serre subcategory of $\scrO_X\mod$ by definition.
                \item $\Coh(X)$ has enough injectives, and said injective objects are flasque. 
                \item An $\scrO_X$-module is coherent if and only if it is finitely presented.
                \item If:
                    $$f: X \to Y$$
                is a morphism between general ringed spaces and if $\scrN$ is a coherent $\scrO_Y$-module, then we will not usually be guaranteed that the pullback $f^*\scrN$ is coherent over $\scrO_X$. When $f$ is proper, though, and if $\scrM$ is some coherent $\scrO_X$-module, then the pushforward $f_*\scrM$ will in fact be coherent (cf. lemma \ref{lemma: pushforwards_of_analytic_coherent_modules}), and this is one of the reasons why having the cohomological base-change formula as in theorem \ref{theorem: relative_analytification_of_sheaves_of_modules} is important for our purposes!

                That said, there is a well-defined pullback functor:
                    $$f^*: \scrO_Y\mod^{\ft} \to \scrO_X\mod^{\ft}$$
                between the categories of finitely generated/finite-type $\scrO_Y$- and $\scrO_X$-modules. The issue mentioned above stems from the fact that the pullback of a finitely presented module is only finitely generated in general.
            \end{itemize}
            
            These properties will from now on be used without explicit mention.
        \end{remark}
        \begin{lemma}[Pushforwards of analytic coherent modules] \label{lemma: pushforwards_of_analytic_coherent_modules}
            Suppose that:
                $$f: \calX \to \calY$$
            is a morphism of complex-analytic spaces. If $f$ is proper then there will be a well-defined t-exact functor:
                $$Rf_*: D^+(\Coh(\calX)) \to D^+(\Coh(\calY))$$
        \end{lemma}
            \begin{proof}
                
            \end{proof}
        \begin{lemma}[Absolute analytification of coherent modules] \label{lemma: absolute_analytification_of_coherent_modules}
            Let $X$ be a locally finite-type $\bbC$-scheme. Then, there is a well-defined functor:
                $$i_X^*: \Coh(X) \to \Coh(X^{\an})$$
            (i.e. the pullback functor $i_X^*: \scrO_X\mod \to \scrO_{X^{\an}}\mod$ in particular does send coherent $\scrO_X$-modules to coherent $\scrO_{X^{\an}}$-modules).
        \end{lemma}
            \begin{proof}[Sketch]
                $i_X^*$ is exact (cf. corollary \ref{coro: GAGA_cohomological_comparison}), so it preserves compactness of objects. 
            \end{proof}
        \begin{theorem}[Relative analytification of coherent modules] \label{theorem: relative_analytification_of_coherent_modules}
            Suppose that:
                $$f: X \to Y$$
            is a morphism between schemes locally of finite type over $\Spec \bbC$. If $f$ is proper, then the canonical natural transformation:
                $$i_Y^* \circ R f_* \Rightarrow R f^{\an}_* \circ i_X^*$$
            will be a natural isomorphism of t-exact functors $D^+(\Coh(X)) \to D^+(\Coh(Y^{\an}))$.

            When $Y \cong \Spec \bbC$, the above implies that:
                $$H^{\bullet}(X, \scrM) \cong H^{\bullet}(X^{\an}, i_X^*\scrM)$$
            are finite-dimensional $\bbC$-vector spaces for any coherent $\scrO_X$-module $\scrM$.
        \end{theorem}
        \begin{remark}[\textit{D\'evissage}]
            Let us recall Chow's Lemma, which says that should $S$ be a Noetherian base scheme and $\pi: X \to S$ be a proper $S$-scheme, then there will exist a projective $S$-scheme $\pi': X' \to S$ along with a morphism of $S$-schemes:
                $$f: X' \to X$$
            for which there is a \textit{dense} open subscheme $U \subset X$ such that:
                $$X' \x_{f, X} U \cong U$$
            If, in addition, both $X$ and $X'$ are irreducible then $f$ will be birational. Furthermore, if $X$ is reduced, irreducible, or integral, then the same can be assumed for $X'$; in particular, this means that if $X$ is a variety (i.e. when all those adjectives are satisfied and $S$ is the spectrum of a field) then we can assume without loss of generality that $X'$ too is a variety over the same field.

            Using Chow's Lemma, and letting $S := \Spec \bbC$, we see that any proper algebraic $\bbC$-variety $Y$ is birationally equivalent to a projective $\bbC$-variety $X$, for which there is some closed immersion:
                $$j_X: X \hookrightarrow \P^n_{\bbC}$$
            We know that the abelian category:
                $$\Coh(\P^n_{\bbC})$$
            is generated via finite colimits by finitely many (compact) objects (namely Serre's twisting line bundles)\footnote{This statement remains true when we replace $\bbC$ with an arbitrary commutative ring.}, 
        \end{remark}

        \begin{theorem}[Compact generation of coherent modules over analytifications] \label{theorem: compact_generation_of_coherent_modules_over_analytifications}
            For any locally finite-type proper $\bbC$-scheme $X$, any coherent $\scrO_{X^{\an}}$-module $\scrM$ admits a 
        \end{theorem}
            \begin{proof}
                
            \end{proof}
        The following is a corollary to a combination of corollary \ref{coro: GAGA_cohomological_comparison} and theorem \ref{theorem: compact_generation_of_coherent_modules_over_analytifications}.
        \begin{corollary}[Serre's GAGA]
            For any locally finite-type proper $\bbC$-scheme $X$, there is an adjoint equivalence of categories:
                $$i_X^*: \Coh(X) \leftrightarrows \Coh(X^{\an}): i_{X *}$$
        \end{corollary}
            
        \begin{appendices}
            \chapter{Fibred categories, descent, and stacks}
                \begin{abstract}
            
                \end{abstract}
                
                \minitoc
                
                \section{Fibred categories}
    \subsection{Generalities on \texorpdfstring{$2$}{}-categories}
        \begin{definition}[$2$-categories] \label{def: 2_categories}
            
        \end{definition}
        
        \begin{definition}[$2$-commutative diagrams] \label{def: 2_commutative_diagrams}
            Let $\calK$ be a $2$-category. A $2$-commutative diagram therein is thus a square as follows, wherein there exists a $2$-isomorphism $\eta: p'f \Rightarrow pg$:
                $$
                    \begin{tikzcd}
                    	{y'} & y \\
                    	{x'} & x
                    	\arrow["{p'}"', from=1-1, to=2-1]
                    	\arrow["f", from=2-1, to=2-2]
                    	\arrow["g", from=1-1, to=1-2]
                    	\arrow["p", from=1-2, to=2-2]
                    	\arrow["\exists \eta", shorten <=8pt, shorten >=8pt, Rightarrow, from=2-1, to=1-2]
                    \end{tikzcd}
                $$
        \end{definition}
    
    \subsection{(Co)fibred categories}
        \subsubsection{Slice \texorpdfstring{$2$}{}-categories}
            \begin{definition}[Slice $2$-categories] \label{def: slice_2_categories}
                Let $\calK$ be a $2$-category and let $x \in \Ob(\calK)$ be an object therein. Then, we define the \textbf{slice} $2$-category $\calK_{/x}$ to be the $2$-category wherein:
                    \begin{itemize}
                        \item the objects are $1$-morphisms $(f: y \to x) \in 1\-\Mor(\calK)$,
                        \item the $1$-morphisms $\phi: (y', f') \to (y, f)$ are $1$-commutative triangles of $1$-morphisms $(f': y' \to x), (f: y \to x) \in 1\-\Mor(\calK)$ in $\calK$ of the following form:
                            $$
                                \begin{tikzcd}
                                	{y'} && y \\
                                	& x
                                	\arrow["\phi", from=1-1, to=1-3]
                                	\arrow["{f'}"', from=1-1, to=2-2]
                                	\arrow["f", from=1-3, to=2-2]
                                \end{tikzcd}
                            $$
                        \item and the $2$-morphisms between $1$-morphisms $\phi, \psi: (y', f') \to (y, f)$ are $2$-morphisms $(\eta: \psi \Rightarrow \phi) \in 2\-\Mor(\calK)$ such that the following diagram is $2$-commutative:
                            $$
                                \begin{tikzcd}
                                	{y'} && y \\
                                	& x
                                	\arrow[""{name=0, anchor=center, inner sep=0}, "\phi"', bend right, from=1-1, to=1-3]
                                	\arrow["{f'}"', from=1-1, to=2-2]
                                	\arrow["f", from=1-3, to=2-2]
                                	\arrow[""{name=1, anchor=center, inner sep=0}, "\psi", bend left, from=1-1, to=1-3]
                                	\arrow[shorten <=2pt, shorten >=2pt, Rightarrow, from=1, to=0, "\eta"']
                                \end{tikzcd}
                            $$
                    \end{itemize}
            \end{definition}
            \begin{example}[Over-categories] \label{example: over_categories}
                Let $\C$ be a category and let $1\-\Cat_2$ be the $2$-category with $1$-categories, functors, and natural transformations as objects, $1$-morphisms, and $2$-morphisms respectively. Then, there is a natural slice $2$-category $(1\-\Cat_2)_{/\C}$ wherein the objects are functors $p: \S \to \C$, $1$-morphisms are the evident $1$-commutative triangles of functors, and $2$-morphisms between $1$-morphisms $F, G: (\S', p') \to (\S, p)$ are natural transformations $\eta \in \Nat(F, G)$ such that $p(\eta_y) = \id_{p'(y)}$ for all $y \in \Ob(\S')$, i.e. such that the following square is strictly $2$-commutative:
                    $$
                        \begin{tikzcd}
                        	{p'(y)} & {p(F(y))} \\
                        	{p'(y)} & {p(G(y))}
                        	\arrow["{p(\eta_y)}", from=1-2, to=2-2]
                        	\arrow[from=1-1, to=1-2]
                        	\arrow["{\id_{p'(y)}}"', from=1-1, to=2-1]
                        	\arrow[from=2-1, to=2-2]
                        	\arrow[shorten <=8pt, shorten >=8pt, Rightarrow, from=2-1, to=1-2]
                        \end{tikzcd}
                    $$
                In this sense, $(1\-\Cat_2)_{/\C}$ is not just a $2$-category, but a strict one.
            \end{example}
            \begin{proposition}[$2$-pullbacks in slice $2$-categories] \label{prop: 2_pullbacks_in_slice_2_categories}
                
            \end{proposition}
                \begin{proof}
                            
                \end{proof}
            
            \begin{definition}[Prefibrations] \label{def: prefibrations}
                Consider a $1$-functor $p: \S \to \C$ between $1$-categories $\S$ and $\C$ is a \textbf{pre-fibration} if and only if it is surjective on the level of both objects and ($1$-)morphisms. 
            \end{definition}
            \begin{remark}[Fibres of prefibrations] \label{remark: fibres_of_prefibrations}
                Consider a prefibration $p: \S \to \C$, viewed as a $1$-morphism of $1\-\Cat_2$, and recall that $1\-\Cat_2$ admits $1$-terminal objects, namely categories that are equivalent to the singleton category $\pt$, as well as $1$-pullbacks. By viewing objects $U \in \Ob(\C)$ as functors $U: \pt \to \C$, one can define \textbf{fibres} of the prefibration $p: \S \to \C$ as $1$-pullbacks of the following kind:
                    $$
                        \begin{tikzcd}
                        	{\S_U} & \S \\
                        	\pt & \C
                        	\arrow["p", from=1-2, to=2-2]
                        	\arrow["U", from=2-1, to=2-2]
                        	\arrow["{p_U}"', from=1-1, to=2-1]
                        	\arrow[from=1-1, to=1-2]
                        	\arrow["\lrcorner"{anchor=center, pos=0.125}, draw=none, from=1-1, to=2-2]
                        \end{tikzcd}
                    $$
                It is then easy to see that $\S_U$ is the category wherein the objects are objects $x \in \S$ such that $p(x) = U$ and the morphisms are morphisms $(\phi: y \to x) \in \Mor(\S)$ such that $p(\phi) = \id_U$, and as such one ought to view $p_U: \S_U \to \pt$ as the unique functor from $\S_U$ to the singleton category whose only object is $U$ and whose only morphism is $\id_U$. Therefore, one can instead define prefibrations as functors $p: \S \to \C$ with non-empty fibres in the sense above and identify them by the families of $1$-pullbacks $\{\S \x_{p, \C, U} \pt\}_{U \in \Ob(\C)}$.
            \end{remark}
            \begin{remark}[The strict $2$-category of pre-fibrations]
                By definition, prefibrations $p: \S \to \C$ are objects of the strict $2$-category $(1\-\Cat_2)_{/\C}$, and so we might be tempted to declare that there is a $2$-full subcategory $\Pre\Fib(\C) \overset{2}{\subset} (1\-\Cat_2)_{/\C}$ spanned by prefibrations over $\C$, and indeed there is. To check that this is the case, one can verify that given any $1$-morphism $(F: (\S', p') \to (\S, p)) \in 1\-\Mor((1\-\Cat_2)_{/\C})$ and any object $U \in \C$, there exists a functor $F_U: \S'_U \to \S_U$ making the following diagram $1$-commutative in $1\-\Cat_2$:
                    $$
                        \begin{tikzcd}
                        	{\S'_U} &&&& {\S_U} \\
                        	{\S'} &&&& \S \\
                        	&& \pt \\
                        	&& \C
                        	\arrow["{p'}"', from=2-1, to=4-3]
                        	\arrow["U", from=3-3, to=4-3]
                        	\arrow["{p'_U}"', from=1-1, to=3-3]
                        	\arrow["\lrcorner"{anchor=center, pos=0.125}, draw=none, from=1-1, to=4-3]
                        	\arrow[from=1-1, to=2-1]
                        	\arrow["{p_U}", from=1-5, to=3-3]
                        	\arrow[from=1-5, to=2-5]
                        	\arrow["p", from=2-5, to=4-3]
                        	\arrow["F", from=2-1, to=2-5]
                        	\arrow["{F_U}", from=1-1, to=1-5, dashed]
                        	\arrow["\lrcorner"{anchor=center, pos=0.125, rotate=-90}, draw=none, from=1-5, to=4-3]
                        \end{tikzcd}
                    $$
                and similarly for $2$-morphisms in $\Pre\Fib(\C)$. We leave this as an exercise for the reader.
            \end{remark}
        
        \subsubsection{Fibred categories}
                
                \section{Descent theory}
    \subsection{Sieves and sites}
    
    \subsection{Descent data, sheaves, and stacks}

            \chapter{Spectral sequences}
                \begin{abstract}
                    I can't just handwave whenever someone asks me what a \say{spectral sequence} is anymore ... I've been avoiding writing these notes for long enough. We will be discussing spectral sequences here mostly from a higher-categorical POV.
                \end{abstract}
                
                \minitoc

                \section{Some generalities on stable \texorpdfstring{$\infty$}{}-categories}
    \subsection{Triangles and stability}
        \begin{definition}[Stable $\infty$-categories] \label{def: stable_infinity_categories} \index{$\infty$-categories! stable}
            
            \begin{enumerate}
                \item \textbf{(Triangles):} Let $\C$ be an $\infty$-category with zero objects $0$ (i.e. let $\C$ be a so-called \textbf{pointed category}). A \textbf{triangle} in $\C$ is just a commutative square of the form:
                    $$
                        \begin{tikzcd}
                            x & y \\
                            0 & z
                            \arrow[from=1-1, to=2-1]
                            \arrow[from=1-1, to=1-2]
                            \arrow[from=1-2, to=2-2]
                            \arrow[from=2-1, to=2-2]
                        \end{tikzcd}
                    $$
                If it is in addition a pullback square (i.e. the limit of a diagram $\simp[1] \x \simp[1] \to \C$), then it will be commonly referred to as a \textbf{fibre sequence}; the dual notion (i.e. a colimit of a diagram $\simp[1] \x \simp[1] \to \C$) is that of \textbf{cofibre sequences}; one speaks also of \textbf{(co)fibres} of morphisms, which are nothing more than pullbacks/pushouts along the canonical morphism from/to the zero object $0$ to the domain of said morphisms (e.g. in the situation above, should the square be a pullback square then it will be a fibre of $y \to z$). Thanks to the universal property of zero objects, one can image triangles in $\C$ as diagrams of shape $\simp[1] \x \simp[1]$.
                \item \textbf{(Stable $\infty$-categories):} An $\infty$-category $\C$ is \textbf{stable} if and only if:
                    \begin{enumerate}
                        \item it is pointed,
                        \item all morphisms in $\C$ admit fibres and cofibres, and
                        \item a triangle in $\C$ is a fibre sequence if and only if it is also a cofibre sequence.
                    \end{enumerate}
            \end{enumerate}
        \end{definition}
        \begin{example}
            
            \begin{itemize}
                \item The $\infty$-category of spectra (sequences of topological spaces indexed by $\N$ along with loopings and suspensions) is stable.
                \item As we shall eventually see, the derived category of an abelian category is stable as an $\infty$-category.
            \end{itemize}
        \end{example}
        
        \begin{definition}[Triangulated $\infty$-categories] \label{def: triangulated_infinity_categories} \index{$\infty$-categories! triangulated}
            
            \begin{enumerate}
                \item \textbf{(Distinguished triangles):} Within a pointed $\infty$-category, we define \textbf{distinguished triangles} to be a cofibre sequence (cf. definition \ref{def: stable_infinity_categories}):
                    $$
                        \begin{tikzcd}
                            x & y \\
                            0 & z
                            \arrow[from=1-1, to=2-1]
                            \arrow[from=1-1, to=1-2]
                            \arrow[from=1-2, to=2-2]
                            \arrow[from=2-1, to=2-2]
                            \arrow["\lrcorner"{anchor=center, pos=0.125, rotate=180}, draw=none, from=2-2, to=1-1]
                        \end{tikzcd}
                    $$
                which admits an extension by another cofibre sequence $y \to z \to \suspension x$ into:
                    $$
                        \begin{tikzcd}
                            x & y & 0 \\
                            0 & z & {\suspension x}
                            \arrow[from=1-1, to=2-1]
                            \arrow[from=1-1, to=1-2]
                            \arrow[from=1-2, to=2-2]
                            \arrow[from=2-1, to=2-2]
                            \arrow["\lrcorner"{anchor=center, pos=0.125, rotate=180}, draw=none, from=2-2, to=1-1]
                            \arrow[from=2-2, to=2-3]
                            \arrow[from=1-2, to=1-3]
                            \arrow[from=1-3, to=2-3]
                            \arrow["\lrcorner"{anchor=center, pos=0.125, rotate=180}, draw=none, from=2-3, to=1-2]
                        \end{tikzcd}
                    $$
                Note that by the universal property of zero objects, giving an extension in the above fashion the same as giving a right-exact functor (called a \textbf{suspension functor}):
                    $$\suspension: \C^{\simp[1] \x \simp[1]} \to \C^{\simp[2] \x \simp[1]}$$
                that pastes onto a triangle:
                    $$
                        \begin{tikzcd}
                            x & y \\
                            0 & z
                            \arrow[from=1-1, to=2-1]
                            \arrow[from=1-1, to=1-2]
                            \arrow[from=1-2, to=2-2]
                            \arrow[from=2-1, to=2-2]
                        \end{tikzcd}
                    $$
                a so-called right-extension via pushing out along the canonical map $y \to 0$:
                    $$
                        \begin{tikzcd}
                            x & y & 0 \\
                            0 & z & w
                            \arrow[from=1-1, to=2-1]
                            \arrow[from=1-1, to=1-2]
                            \arrow[from=1-2, to=2-2]
                            \arrow[from=2-1, to=2-2]
                            \arrow[from=1-2, to=1-3]
                            \arrow[from=1-3, to=2-3]
                            \arrow[from=2-2, to=2-3]
                            \arrow["\lrcorner"{anchor=center, pos=0.125, rotate=180}, draw=none, from=2-3, to=1-2]
                        \end{tikzcd}
                    $$
                \item \textbf{(Triangulated categories):} A pointed $\infty$-category admitting all cofibre sequences will naturally have all distinguish triangles, and thus shall be called \textbf{triangulated}.
            \end{enumerate}
        \end{definition}
        \begin{remark}[Elementary properties of triangulated $\infty$-categories] \label{remark: elementary_properties_of_triangulated_categories} \index{$\infty$-categories! triangulated! properties} \index{$\infty$-categories! stable! properties}
            
            \begin{enumerate}
                \item Clearly, stable $\infty$-categories are triangulated.
                \item It is not hard to see how the opposite of a stable $\infty$-category is necessarily stable too. The opposite of a triangulated $\infty$-category is not necessary triangulated; this happens if and only if the $\infty$-category is stable.
                \item 
                    \begin{enumerate}
                        \item Full subcategories of triangulated $\infty$-categories that are stable under taking cofibre sequences are triangulated themselves; these subcategories are known as \textbf{stable $\infty$-subcategories}. So-called \textbf{Serre $\infty$-subcategories} - stable subcategories of abelian $\infty$-categories (which are \textit{a priori} stable) - are special cases of these stable $\infty$-categories. These Serre $\infty$-subcategories are themselves special cases of (left-)exact reflective localisations of stable $\infty$-categories, which are stable $\infty$-categories whose associated fully faithful embeddings into their ambient stbale $\infty$-categories admit (left-)exact left-adjoints.
                        \item Let $\C$ be a stable $\infty$-category and let $\C_0$ be a triangulated full $\infty$-subcategory. $\C_0$ is thus also stable.
                    \end{enumerate}
                \item If $K$ is a simplicial set and $\C$ is any stable $\infty$-category, then the functor category $\C^K$ is also stable.
            \end{enumerate}
        \end{remark}
        
        \begin{proposition}[Stability criteria for pointed $\infty$-categories] \label{prop: stability_criteria_for_pointed_infinity_categories} \index{$\infty$-categories! stable! criteria}
            A pointed $\infty$-category $\C$ is stable if and only if the following equivalent criteria are satisfied:
                \begin{enumerate}
                    \item $\C$ is finitely complete and finitely cocomplete.
                    \item Finite pushouts and pullbakcs in $\C$ coincide (i.e. $\C$ admits all finite fibred biproducts).
                \end{enumerate}
        \end{proposition}
            \begin{proof}
                
                \begin{enumerate}
                    \item \textbf{(Proof of equivalence):} 
                        \begin{enumerate}
                            \item Assume that $\C$ is finitely complete and finitely cocomplete. This implies that pushouts and pullbacks exist in $\C$, and hence $\C$ admits all cofibre and fibre sequences. It thus remains to show that cofibres and fibres in $\C$ are the same; one can then make use of the universal property of zero objects and base change to show the existence of finite fibred biproducts. However, note that again thanks to the universal property of zero objects, the following triangle is simultaneously a fibre sequence:
                                $$
                                    \begin{tikzcd}
                                        0 & x \\
                                        0 & x
                                        \arrow[from=1-1, to=2-1]
                                        \arrow[from=1-2, to=2-2]
                                        \arrow[from=2-1, to=2-2]
                                        \arrow[from=1-1, to=1-2]
                                        \arrow["\lrcorner"{anchor=center, pos=0.125, rotate=180}, draw=none, from=2-2, to=1-1]
                                    \end{tikzcd}
                                $$
                            and the following a cofibre sequence:
                                $$
                                    \begin{tikzcd}
                                        x & 0 \\
                                        x & 0
                                        \arrow[from=1-2, to=2-2]
                                        \arrow[from=2-1, to=2-2]
                                        \arrow[from=1-1, to=2-1]
                                        \arrow[from=1-1, to=1-2]
                                        \arrow["\lrcorner"{anchor=center, pos=0.125}, draw=none, from=1-1, to=2-2]
                                    \end{tikzcd}
                                $$
                            for all objects $x$ of $\C$; one can them simply base change to see how cofibre and fibre sequences must coincide.
                            \item Conversely, assume that all finite pushouts and pullbacks in $\C$ coincide. This in particular tells us that cofibre sequences and fibre sequences are the same, and also, that one can build monos and epis using the zero object $0$ using biproducts of the following form:
                                $$
                                    \begin{tikzcd}
                                        x & 0 \\
                                        y & z
                                        \arrow[from=1-2, to=2-2]
                                        \arrow[two heads, from=2-1, to=2-2]
                                        \arrow[tail, from=1-1, to=2-1]
                                        \arrow[from=1-1, to=1-2]
                                        \arrow["\lrcorner"{anchor=center, pos=0.125}, draw=none, from=1-1, to=2-2]
                                        \arrow["\lrcorner"{anchor=center, pos=0.125, rotate=180}, draw=none, from=2-2, to=1-1]
                                    \end{tikzcd}
                                $$
                        \end{enumerate}
                    \item \textbf{(Proof of stability):} By definition \ref{def: stable_infinity_categories}, pointed $\infty$-categories that satisfy the second criterion are stable. Conversely, suppose that our pointed $\infty$-category $\C$ is stable. Then, one can simply make use of the universal property of zero objects and base change (co)fibre sequences to show that all finite pullbacks and all finite pushouts must exist in $\C$. 
                \end{enumerate}
            \end{proof}
        \begin{convention}
            Often, finite biproducts in stable $\infty$-categories shall be denoted by $\oplus$, especially when the category is abelian.
        \end{convention}  
        
        \begin{remark}[Regular cardinals]
            From now on we will be using the notion of regular cardinals often. For details on the notion, see definition \ref{def: limit_cardinal}.
        \end{remark}
        
        \begin{proposition}[Accessible stable $\infty$-category] \label{prop: accessible_stable_infinity_categories} \index{$\infty$-categories! stable! accessible} \index{$\infty$-categories! stable! ind-completions}
            Let $\kappa$ be a regular cardinal and let $\C$ be a $\kappa$-small stable $\infty$-category. Then, its $\kappa$-ind-completion is stable as well.
        \end{proposition}
            \begin{proof}
                This is an easy consequence of the fact that filtered colimits preserve finite coproducts, and that $\C$ embeds fully faithfully into $\Ind_{\kappa}(\C)$ as a subcategory closed under finite limits and colimits. 
            \end{proof}
        \begin{remark} \index{$\infty$-categories! stable! accessible} \index{$\infty$-categories! stable! pro-completions}
            If we were to replace $\Ind_{\kappa}(\C)$ in proposition \ref{prop: accessible_stable_infinity_categories} by $\Pro_{\kappa}(\C)$, we would also get a stable $\infty$-category through an application of the following equivalence of categories and remark \ref{remark: elementary_properties_of_triangulated_categories}:
                $$\Pro_{\kappa}(\C) \cong \Ind_{\kappa}(\C^{\op})^{\op}$$
        \end{remark}
        
        \begin{proposition}[Homotopy category of triangulated categories] \label{prop: homotopy_category_of_triangulated_categories} \index{$\infty$-categories! stable! homotopy categories}
            Let $\C$ be a triangulated $\infty$-category (or better, a stable $\infty$-category) and suppose that the suspension functor thereon:
                $$\suspension: \C^{\simp[1] \x \simp[1]} \to \C^{\simp[2] \x \simp[1]}$$
            is fully faithful. Then, the homotopy category $h\C$ can also be endowed with the structure of a triangulated category (however this time with all trivial higher morphisms).
        \end{proposition}
            \begin{proof}
                This is straightforward from the universal property of homotopy categories. 
            \end{proof}
                
    \subsection{Exact functors}
        \begin{remark}[Functors between stable $\infty$-categories] \label{remark: functors_between_stable_infinity_categories} \index{$\infty$-categories! stable! limits} \index{$\infty$-categories! stable! exact functors}
            
            \begin{enumerate}
                \item It is a straight-forward consequence of proposition \ref{prop: stability_criteria_for_pointed_infinity_categories} that a functor (i.e. one that preserves finite limits and finite colimits):
                    $$F: \C \to \D$$
                between two stable $\infty$-categories is exact if and only if it preserves distinguished triangles.
                \item Exact functors between them from a stable $\infty$-category $\C$ to another $\D$ span a full $\infty$-subcategory of the functor category $\infty\-\Cat(\C, \D)$.
                \item Stable $\infty$-categories and exact functors between them form a \textit{locally non-full} $(\infty, 2)$-subcategory of $\infty\-\Cat$, which shall be denoted by $\infty\-\Stab\Cat$ or $\Stab\Cat$. 
            \end{enumerate}
        \end{remark}
        
        \begin{proposition}[(Co)limits of stable $\infty$-categories] \label{prop: (co)limits_of_stable_infinity_categories} \index{$\infty$-categories! stable! limits} \index{$\infty$-categories! stable! colimits}
            Let $\kappa$ be a regular cardinal. Then, the $(\infty, 1)$-category $\Stab\Cat^{< \kappa}$ of $\kappa$-small stable $\infty$-categories is complete and closed under $\kappa$-small limits. Additionally, it admits all $\kappa$-small filtered and is closed under these colimits. 
        \end{proposition}
            \begin{proof}
                
            \end{proof}

                \section{Homological algebra in stable \texorpdfstring{$\infty$}{}-categories}
    \subsection{t-structures and their sweet little hearts}
        \begin{remark}[Suspensions and loops] \label{remark: suspensions_and_loops}
            Let $\C$ be a triangulated $\infty$-category and let:
                $$\suspension: \C^{\simp[1] \x \simp[1]} \to \C^{\simp[2] \x \simp[1]}$$
            be the suspension functor on $\C$. By definition \ref{def: triangulated_infinity_categories}, it is the functor which extends a triangle:
                $$
                    \begin{tikzcd}
                        x & y \\
                        0 & z
                        \arrow[from=1-1, to=2-1]
                        \arrow[from=1-1, to=1-2]
                        \arrow[from=1-2, to=2-2]
                        \arrow[from=2-1, to=2-2]
                    \end{tikzcd}
                $$
            via the construction of the pushout of $y \to z$ along $y \to 0$:
                $$
                    \begin{tikzcd}
                        x & y & 0 \\
                        0 & z & w
                        \arrow[from=1-1, to=2-1]
                        \arrow[from=1-1, to=1-2]
                        \arrow[from=1-2, to=2-2]
                        \arrow[from=2-1, to=2-2]
                        \arrow[from=1-2, to=1-3]
                        \arrow[from=1-3, to=2-3]
                        \arrow[from=2-2, to=2-3]
                        \arrow["\lrcorner"{anchor=center, pos=0.125, rotate=180}, draw=none, from=2-3, to=1-2]
                    \end{tikzcd}
                $$
            Now, thanks to the universal property of pushouts, we can instead view the suspension functor as the functor:
                $$\suspension: \C^{\simp[1] \x \simp[1]} \to \C$$
            that sends triangles $x \to y \to z$ to the suspension $\suspension x$ of its first vertex $x$ (note that this version is still right-exact, and this is crucial fact). Then, by some abstract nonsense (see \cite[Section 3]{nlab:infinity-1-limit} for instance), $\suspension$ ought to fit into the following adjoint triple:
                $$
                    \begin{tikzcd}
                        {\C^{\simp[1] \x \simp[1]}} && {\C^{\simp[1] \x \simp[1]}}
                        \arrow[""{name=0, anchor=center, inner sep=0}, "\loopspace"', shift right=5, from=1-1, to=1-3]
                        \arrow[""{name=1, anchor=center, inner sep=0}, "\suspension", shift left=5, from=1-1, to=1-3]
                        \arrow[""{name=2, anchor=center, inner sep=0}, "\const"{description}, hook', from=1-3, to=1-1]
                        \arrow["\dashv"{anchor=center, rotate=-90}, draw=none, from=1, to=2]
                        \arrow["\dashv"{anchor=center, rotate=-90}, draw=none, from=2, to=0]
                    \end{tikzcd}
                $$
            Here, we take $\const: \C \to \C^{\simp[1] \x \simp[1]}$ to be the functor given by:
                $$
                    x \mapsto 
                    \begin{tikzcd}
                        x & x \\
                        0 & x
                        \arrow[from=1-1, to=2-1]
                        \arrow[from=2-1, to=2-2]
                        \arrow["\cong", from=1-1, to=1-2]
                        \arrow["\cong", from=1-2, to=2-2]
                    \end{tikzcd}
                $$
            and $\loopspace: \C^{\simp[1] \x \simp[1]} \to \C$ (called the \textbf{loop space functor}) to be the one determined by:
                $$
                    \begin{tikzcd}
                        {\loopspace z} & x & 0 \\
                        0 & y & z
                        \arrow[from=1-2, to=1-3]
                        \arrow[from=1-2, to=2-2]
                        \arrow[from=1-3, to=2-3]
                        \arrow[from=2-2, to=2-3]
                        \arrow[from=1-1, to=2-1]
                        \arrow[from=2-1, to=2-2]
                        \arrow[from=1-1, to=1-2]
                        \arrow["\lrcorner"{anchor=center, pos=0.125}, draw=none, from=1-1, to=2-2]
                    \end{tikzcd}
                $$
            One thing to note is that $\loopspace$ actually only exists if $\C$ is stable, not just when it is merely triangulated, as triangulated $\infty$-categories are not assumed to have any sort of limits aside from initial objects. Another is that by the composability of adjoint pairs, we have the following induced adjunction:
                $$
                    \begin{tikzcd}
                        {\C^{\simp[1] \x \simp[1]}} && {\C^{\simp[1] \x \simp[1]}}
                        \arrow[""{name=0, anchor=center, inner sep=0}, "{\loopspace \circ \const}"', shift right=2, from=1-1, to=1-3]
                        \arrow[""{name=1, anchor=center, inner sep=0}, "{\const \circ \suspension}", shift left=2, from=1-1, to=1-3]
                        \arrow["\dashv"{anchor=center, rotate=-90}, draw=none, from=1, to=0]
                    \end{tikzcd}
                $$
        \end{remark}
    
        \begin{definition}[t-structures] \label{def: t_structures} \index{$\infty$-categories! stable! t-structures} \index{$\infty$-categories! stable! t-structures! hearts} \index{Short exact sequences}
            \begin{enumerate}
                \item \textbf{(t-structures):} The \textbf{t-structure} on a \textit{stable} $\infty$-category $\C$ is a pair of \textit{isomorphism-stable} full subcategories $\C^{\leq 0}$ and $\C^{\geq 0}$ containing the zero object $0$ such that:
                    \begin{enumerate}
                        \item \textbf{(Orthogonality):} For all $x \in \C^{\geq 0}$ and all $y \in \C^{\leq 0}$, the space $\C(x, \loopspace y)$ is contractible (i.e. homotopic to $0$). 
                        
                        Objects $y \in \C^{\leq 0}$ such that $\C(x, \loopspace y)$ is contractible for all $x \in \C_{> 0}$ span a full $\infty$-subcategory of $\C^{\leq 0}$, which we shall denote by $\C_{< 0}$. By applying the adjoint triple $(\suspension \ladjoint \const \ladjoint \loopspace)$ (cf. remark \ref{remark: suspensions_and_loops}), we can see that objects $x \in \C^{\geq 0}$ such that the space $\C(\suspension x, y)$ is contractible for all $y \in \C_{< 0}$ similarly span a full subcategory of $\C^{\geq 0}$, which is written $\C_{> 0}$. 
                        
                        By the definition of stable $\infty$-subcategories (cf. remark \ref{remark: elementary_properties_of_triangulated_categories}), $\C^{\leq 0}$ and $\C_{< 0}$ are stable $\infty$-subcategories of $\C$, but $\C^{\geq 0}$ and $\C_{> 0}$ are not. 
                        \item \textbf{(Translational retro-invariance):} If a right-extension:
                            $$
                                \begin{tikzcd}
                                    x & y & 0 \\
                                    0 & z & w
                                    \arrow[from=2-1, to=2-2]
                                    \arrow[from=1-2, to=2-2]
                                    \arrow[from=1-1, to=1-2]
                                    \arrow[from=1-1, to=2-1]
                                    \arrow["\lrcorner"{anchor=center, pos=0.125, rotate=180}, draw=none, from=2-2, to=1-1]
                                    \arrow[from=2-2, to=2-3]
                                    \arrow[from=1-2, to=1-3]
                                    \arrow[from=1-3, to=2-3]
                                    \arrow["\lrcorner"{anchor=center, pos=0.125, rotate=180}, draw=none, from=2-3, to=1-2]
                                \end{tikzcd}
                            $$
                        of a cofibre sequence is an object of $(\C^{\geq 0})^{\simp[2] \x \simp[1]}$, then the cofibre sequence:
                            $$
                                \begin{tikzcd}
                                    x & y \\
                                    0 & z
                                    \arrow[from=2-1, to=2-2]
                                    \arrow[from=1-2, to=2-2]
                                    \arrow[from=1-1, to=1-2]
                                    \arrow[from=1-1, to=2-1]
                                    \arrow["\lrcorner"{anchor=center, pos=0.125, rotate=180}, draw=none, from=2-2, to=1-1]
                                \end{tikzcd}
                            $$
                        itself is an object of $(\C^{\geq 0})^{\simp[1] \x \simp[1]}$ (actually, this extends to all right-extensions of triangles, because they all factor through cofibre sequences thanks to the universal property of colimits). 
                        
                        Dually, if the left-extension of a fibre sequence:
                            $$
                                \begin{tikzcd}
                                    {\loopspace z} & x & 0 \\
                                    0 & y & z
                                    \arrow[from=1-3, to=2-3]
                                    \arrow[from=2-2, to=2-3]
                                    \arrow[from=1-2, to=2-2]
                                    \arrow[from=1-2, to=1-3]
                                    \arrow["\lrcorner"{anchor=center, pos=0.125}, draw=none, from=1-2, to=2-3]
                                    \arrow[from=1-1, to=2-1]
                                    \arrow[from=2-1, to=2-2]
                                    \arrow[from=1-1, to=1-2]
                                    \arrow["\lrcorner"{anchor=center, pos=0.125}, draw=none, from=1-1, to=2-2]
                                \end{tikzcd}
                            $$
                        is an object of $(\C^{\leq 0})^{\simp[2] \x \simp[1]}$, then that fibre sequence itself:
                            $$
                                \begin{tikzcd}
                                    x & 0 \\
                                    y & z
                                    \arrow[from=1-2, to=2-2]
                                    \arrow[from=2-1, to=2-2]
                                    \arrow[from=1-1, to=2-1]
                                    \arrow[from=1-1, to=1-2]
                                    \arrow["\lrcorner"{anchor=center, pos=0.125}, draw=none, from=1-1, to=2-2]
                                \end{tikzcd}
                            $$
                        is an object of $(\C^{\leq 0})^{\simp[1] \x \simp[1]}$. This also applies to triangles which may not be fibre sequences. 
                        \item \textbf{(Torsion):} For all objects $x \in \C$, there exists a (co)fibre sequence in $\C^{\simp[1] \x \simp[1]}$:
                            $$
                                \begin{tikzcd}
                                    {x'} & 0 \\
                                    x & {x''}
                                    \arrow[from=1-2, to=2-2]
                                    \arrow[from=2-1, to=2-2]
                                    \arrow[from=1-1, to=2-1]
                                    \arrow[from=1-1, to=1-2]
                                    \arrow["\lrcorner"{anchor=center, pos=0.125}, draw=none, from=1-1, to=2-2]
                                    \arrow["\lrcorner"{anchor=center, pos=0.125, rotate=180}, draw=none, from=2-2, to=1-1]
                                \end{tikzcd}
                            $$
                        wherein $x' \in \C^{\geq 0}$, whose left and right-extensions are both zero:
                            $$
                                \begin{tikzcd}
                                    0 & {x'} & 0 \\
                                    0 & x & {x''} \\
                                    & 0 & 0
                                    \arrow[from=1-3, to=2-3]
                                    \arrow[from=2-2, to=2-3]
                                    \arrow[from=1-2, to=2-2]
                                    \arrow[from=1-2, to=1-3]
                                    \arrow["\lrcorner"{anchor=center, pos=0.125}, draw=none, from=1-2, to=2-3]
                                    \arrow["\lrcorner"{anchor=center, pos=0.125, rotate=180}, draw=none, from=2-3, to=1-2]
                                    \arrow[from=2-2, to=3-2]
                                    \arrow[from=3-2, to=3-3]
                                    \arrow[from=2-3, to=3-3]
                                    \arrow["\lrcorner"{anchor=center, pos=0.125, rotate=180}, draw=none, from=3-3, to=2-2]
                                    \arrow[from=1-1, to=2-1]
                                    \arrow[from=2-1, to=2-2]
                                    \arrow[from=1-1, to=1-2]
                                    \arrow["\lrcorner"{anchor=center, pos=0.125}, draw=none, from=1-1, to=2-2]
                                    \arrow["\lrcorner"{anchor=center, pos=0.125}, draw=none, from=2-2, to=3-3]
                                    \arrow["\lrcorner"{anchor=center, pos=0.125, rotate=180}, draw=none, from=2-2, to=1-1]
                                \end{tikzcd}
                            $$
                        (note how this implies that the mapping spaces $\C(x, \loopspace x'')$ and $\C(\suspension x', x)$ are contractible, and hence $x' \in \C^{\geq 0}$ and $x'' \in \C_{< 0}$); in other words, $(\C^{\geq 0}, \C^{\leq 0})$ is a $t$-structure if $(\C_{> 0}, \C_{< 0})$ is a (homotopical) \href{https://ncatlab.org/joyalscatlab/published/Factorisation+systems}{\underline{factorisation system}}, and in fact, an epi-mono factorisation system thanks to the universal property of zero objects. Such a sequence is known as a \textbf{short exact sequence}. 
                        
                        Also, note that in the situation above, we have the following right and left-extensions:
                            $$
                                \begin{tikzcd}
                                    0 & {x'} & 0 \\
                                    0 & x & {x''}
                                    \arrow[from=1-3, to=2-3]
                                    \arrow[from=2-2, to=2-3]
                                    \arrow[from=1-2, to=2-2]
                                    \arrow[from=1-2, to=1-3]
                                    \arrow["\lrcorner"{anchor=center, pos=0.125, rotate=180}, draw=none, from=2-3, to=1-2]
                                    \arrow[from=1-1, to=2-1]
                                    \arrow[from=2-1, to=2-2]
                                    \arrow[from=1-1, to=1-2]
                                    \arrow["\lrcorner"{anchor=center, pos=0.125, rotate=180}, draw=none, from=2-2, to=1-1]
                                \end{tikzcd}
                            $$
                            $$
                                \begin{tikzcd}
                                    {x'} & 0 \\
                                    x & {x''} \\
                                    0 & 0
                                    \arrow[from=1-2, to=2-2]
                                    \arrow[from=2-1, to=2-2]
                                    \arrow[from=1-1, to=2-1]
                                    \arrow[from=1-1, to=1-2]
                                    \arrow["\lrcorner"{anchor=center, pos=0.125}, draw=none, from=1-1, to=2-2]
                                    \arrow[from=2-2, to=3-2]
                                    \arrow[from=2-1, to=3-1]
                                    \arrow[from=3-1, to=3-2]
                                    \arrow["\lrcorner"{anchor=center, pos=0.125}, draw=none, from=2-1, to=3-2]
                                \end{tikzcd}
                            $$
                    \end{enumerate}
                \item \textbf{(Hearts of t-structures):} It is not hard to see that within a stable $\infty$-category $\C$ equipped with some choice of t-structure $(\C^{\geq 0}, \C^{\leq 0})$, short exact sequences would form an \textit{isomorphism-stable} full $\infty$-subcategory of $\C^{\simp[1] \x \simp[1]}$, which we shall call the \textbf{heart} of its t-structure. 
            \end{enumerate}
            Note how we are indexing \textit{homologically}.
        \end{definition}
        \begin{remark}
            Let $\C$ be a stable $\infty$-category. Then, its $t$-structure can be thought of as consisting of two subcategories spanned, respectively, by \say{complexes} with vanishing positive cohomlogies (e.g. projective resolutions) and those with vanishing negative cohomologies (e.g. injective resolutions). The heart of that very $t$-structure, therefore, shall be spanned by \say{complexes} concentrated in (co)homological degree $0$, i.e. \say{exact sequences}. 
        \end{remark}
        
        \begin{proposition}[t-structures and localisations] \label{prop: t_structures_and_localisations}
            Let $\C$ be a stable $\infty$-category and let $(\C^{\geq 0}, \C^{\leq 0})$ be a t-structure thereon with corresponding canonical fully faithful exact embeddings $\iota^{\geq 0}$ and $\iota^{\leq 0}$. Then, there exists the following composable adjunctions exhibiting $\C^{\leq 0}$ and $\C^{\geq 0}$ as a reflective and a coreflective $\infty$-subcategory of $\C$ respectively:
                $$
                    \begin{tikzcd}
                        {\C^{\leq 0}} & \C & {\C^{\geq 0}}
                        \arrow[""{name=0, anchor=center, inner sep=0}, "{\iota^{\leq 0}}"', shift right=2, hook, from=1-1, to=1-2]
                        \arrow[""{name=1, anchor=center, inner sep=0}, "{\tau^{\leq 0}}"', shift right=2, from=1-2, to=1-1]
                        \arrow[""{name=2, anchor=center, inner sep=0}, "{\tau^{\geq 0}}"', shift right=2, from=1-2, to=1-3]
                        \arrow[""{name=3, anchor=center, inner sep=0}, "{\iota^{\geq 0}}"', shift right=2, hook', from=1-3, to=1-2]
                        \arrow["\dashv"{anchor=center, rotate=-90}, draw=none, from=1, to=0]
                        \arrow["\dashv"{anchor=center, rotate=-90}, draw=none, from=3, to=2]
                    \end{tikzcd}
                $$
            We call the functors $\tau^{\geq 0}$ and $\tau^{\leq 0}$ the \textbf{$\geq 0$-truncation} and the \textbf{$\leq 0$-truncation} respectively.
        \end{proposition}
            \begin{proof}
                
            \end{proof}
        \begin{corollary}[Cooking up short exact sequences using truncations] \label{coro: short_exact_sequences_and_truncations}
            Suppose that $x$ is an object of a stable $\infty$-category $\C$, and let $(\C^{\geq 0}, \C^{\leq 0})$ be a t-structure thereon. Then, the heart of this t-structure can be given by:
                $$
                    \begin{aligned}
                        \C^{\heart} & \cong \iota^{\leq 0} \circ \tau^{\geq 0} \C^{\leq 0}
                        \\
                        & \cong \iota^{\geq 0} \circ \tau^{\leq 0} \C^{\geq 0} 
                        \\
                        & \cong \tau^{\geq 0} \circ \iota^{\geq 0} \circ \tau^{\leq 0} \circ \iota^{\leq 0} \C 
                        \\
                        & \cong \tau^{\leq 0} \circ \iota^{\leq 0} \circ \tau^{\geq 0} \circ \iota^{\geq 0} \C
                    \end{aligned}
                $$
            In other words, the composite adjunction $(\tau^{\leq 0} \circ \iota^{\geq 0} \ladjoint \tau^{\geq 0} \circ \iota^{\leq 0})$ is an adjoint equivalence over $\C^{\heart}$. 
        \end{corollary}
            \begin{proof}
                
            \end{proof}
        
        \begin{theorem}[Hearts are abelian] \label{theorem: hearts_are_abelian} \index{$\infty$-categories! stable! t-structures! hearts}
            The heart of the t-structure of a stable $\infty$-category is an abelian $\infty$-category.
        \end{theorem}
            \begin{proof}
                
            \end{proof}
            
    \subsection{Spectral sequences}

    \subsection{Derived categories}
            
        \end{appendices}

    \part{Formal geometry}
        \chapter{Deformation theory}
            \begin{abstract}
                
            \end{abstract}
            
            \minitoc

            \input{Formal geometry/Commutative deformation theory/formal_deformation_problems}

            \input{Formal geometry/Commutative deformation theory/cotangent_complexes}
            
        \chapter{Algebraisation}
            \begin{abstract}
                
            \end{abstract}
            
            \minitoc

            \input{Formal geometry/Algebraisation and formal geometry/formal_algebraic_spaces}

            \input{Formal geometry/Algebraisation and formal geometry/formal_algebraic_stacks}

            \section{Algebraic de Rham cohomology}
    \subsection{de Rham spaces}
        \begin{convention}
            Given any commutative ring $R$, write $R_{\red} := R/\Nil(R)$ for the \say{underlying} reduced ring (the idea here is that $\Nil(R/\Nil(R)) = 0$).
        \end{convention}
    
        \begin{definition}[Zariski-infinitesimally close points] \label{def: zariski_infinitesimally_close_points}
            Let $X$ be a scheme and let $S$ be a test scheme. Two $S$-points $x, y \in X(S) := \Mor_{\Sch}(S, X)$ are said to be \textbf{Zariski-infinitesimally close} to one another if and only if their images under the canonical map $X(S) \to X(S_{\red})$ coincide.
        \end{definition}
        Here, $S_{\red}$ is the underlying reduced subscheme of $S$. As a locally ringed space, it is given by:
            $$S_{\red} := (|S|, \scrO_{S^{\red}})$$
        with the structure sheaf given by:
            $$\scrO_{S^{\red}}(U) := \scrO_S(U)^{\red}$$
        and in particular, when $S := \Spec R$, we have $S_{\red} \cong \Spec R_{\red}$. This construction stipulates that there is a canonical closed embedding $S_{\red} \subset S$ (cut out by the ideal sheaf $\Nil(\scrO_S) \subset \scrO_S$) and as such, the condition that two $S$-points $x, y: S \toto X$ are Zariski-infinitesimally close to one another is that the following two compositions are equal to one another:
            $$
                \begin{tikzcd}
                {S_{\red}} & S & X
                \arrow[from=1-1, to=1-2]
                \arrow["y"', shift right, from=1-2, to=1-3]
                \arrow["x", shift left, from=1-2, to=1-3]
                \end{tikzcd}
            $$

        For what follows, it is useful to recall the universal property of formal completions. 
        \begin{lemma}[Universal property of formal completion] \label{lemma: universal_property_of_formal_completion}
            
        \end{lemma}
            \begin{proof}
                Exercise.
            \end{proof}

        \begin{proposition}[Infinitesimality and the diagonal] \label{prop: infinitesimality_and_the_diagonal}
            Let $X$ be a separated Noetherian scheme and let $S$ be a test scheme. Two $S$-points $x, y \in X(S)$ are Zariski-infinitesimally close to one another if and only if there is a factorisation in the category of formal schemes as follows:
                $$
                    \begin{tikzcd}
                    S & {(X, \Delta_X)^{\wedge}} \\
                    & {X^2}
                    \arrow["(x \x y)^{\wedge}", dashed, from=1-1, to=1-2]
                    \arrow["{x \x y}"', from=1-1, to=2-2]
                    \arrow["{\Delta_X^{\wedge}}", from=1-2, to=2-2]
                    \end{tikzcd}
                $$
            wherein $(X^2, \Delta_X)^{\wedge}$ is the formal completion\footnote{This is why we need $X$ to be separated and Noetherian. Otherwise, the formal completion will not be a formal scheme in the sense of \cite{hartshorne}.} of $X^2$ along the closed subscheme $\im \Delta_X$, and $\Delta_X^{\wedge}$ is the canonical immersion. 
        \end{proposition}
            \begin{proof}
                By Zariski descent, it suffices to check that the result holds for $S \in \Ob(\Sch^{\aff})$, say $S := \Spec R$. Moreover, we can assume without any loss of generality that $X$ is affine, say $X := \Spec A$; affine schemes are separated \textit{a priori}, so we need only assume that $A$ is Noetherian. With these reduction steps in place, the problem now becomes to prove that given any two ring maps $f, g: A \to R$, we have that:
                    $$\pi \circ f = \pi \circ g$$
                with $\pi: R \to R_{\red}$ being the canonical quotient map, if and only if there is a factorisation in $\Cring$ as follows:
                    \begin{equation} \label{diagram: formal_neighbourhood_of_diagonal_algebraic}
                        \begin{tikzcd}
                        R & {(A^{\tensor 2}, I)^{\wedge}} \\
                        & {A^{\tensor 2}}
                        \arrow["{(f \tensor g)^{\wedge}}"', dashed, from=1-2, to=1-1]
                        \arrow["{f \tensor g}", from=2-2, to=1-1]
                        \arrow["{\mu^{\wedge}}"', from=2-2, to=1-2]
                        \end{tikzcd}
                    \end{equation}
                wherein:
                    $$I := \ker(\mu: A^{\tensor 2} \to A)$$
                is the kernel of the multiplication map on $A$, and:
                    $$\mu^{\wedge}: A^{\tensor 2} \to (A^{\tensor 2}, I)^{\wedge}$$
                is the composition $A^{\tensor 2} \xrightarrow[]{\mu} A \to (A^{\tensor 2}, I)^{\wedge}$, wherein the second arrow is given by $\projlim_{n \geq 1} (A^{\tensor 2} \to A^{\tensor 2}/I^n)$. Also, to be clear, by $f \tensor g: A^{\tensor 2} \to R$, we mean the map given by:
                    $$(f \tensor g)(a \tensor b) := f(a) g(b)$$
                for all $a, b \in A$. As a preliminary observation, note that because $I$ is generated by the subset $\{a \tensor 1 - 1 \tensor a\}_{a \in A}$, we have:
                    \begin{equation} \label{equation: difference_of_points_along_diagonal}
                        \begin{aligned}
                            (\pi \circ (f \tensor g))(a \tensor 1 - 1 \tensor a) & = \pi(f(a) g(1) - f(1) g(a))
                            \\
                            & = \pi(f(a) - g(a))
                            \\
                            & = (\pi \circ (f - g))(a)
                        \end{aligned}
                    \end{equation}
                for all $a \in A$. From this, we see that:
                    $$\pi \circ f = \pi \circ g \iff (\pi \circ (f \tensor g))(I) = 0$$

                Firstly, suppose that the factorisation \eqref{diagram: formal_neighbourhood_of_diagonal_algebraic} exists, and then simply consider:
                    $$(\pi \circ (f \tensor g))(I) = ( (f \tensor g)^{\wedge} \circ \mu^{\wedge} )(I) = (f \tensor g)^{\wedge}(0) = 0$$
                wherein the last equality is because $I := \ker \mu$, which implies that $\mu^{\wedge}(I) = 0$. Equation \eqref{equation: difference_of_points_along_diagonal} then tells us that $\pi \circ f = \pi \circ g$, as needed.

                Conversely, suppose that $x, y \in X(S)$ are Zariski-infinitesimally close. By definition, this means that:
                    $$\pi \circ (f - g) = 0$$
                and through equation \eqref{equation: difference_of_points_along_diagonal}, we see therefore that:
                    $$(\pi \circ (f \tensor g))(I) = 0$$
                This tells us that $(f \tensor g)(I) \subseteq \Nil(R)$. 
            \end{proof}

        It can be shown that Zariski-infinitesimality between pairs of points $x, y: S \toto X$ is an equivalence relation on the set $X(S)$. The classifying space of this equivalence relation is known as the \say{de Rham space} of $X$, and in light of the Zariski-infinitesimality criterion of proposition \ref{prop: infinitesimality_and_the_diagonal}, we see that $X_{\dR}$ as in the following definition is equivalently given by the following colimit of presheaves on $(\Sch^{\aff})$:
            $$X_{\dR} := \coeq( (X, \Delta_X)^{\wedge} \toto X )$$
        wherein the arrows are composition of the immersion $\Delta_X^{\wedge}: (X, \Delta_X)^{\wedge} \to X$ with the two canonical  projections $X^2 \to X$, respectively.
        \begin{definition}[de Rham spaces] \label{def: de_rham_spaces}
            The \textbf{de Rham space} of a scheme $X$ is the presheaf $X_{\dR}: (\Sch^{\aff})^{\op} \to \Sets$ given by:
                $$X_{\dR}(S) := X(S_{\red})$$
            for all test schemes $S$.
        \end{definition}

        When $X$ is formally smooth, the canonical map $X(S) \to X_{\dR}(S)$ is surjective for all test schemes $S$, by the definition of formal smoothness. Thus, when $X$ is formally smooth, we can view $X_{\dR}$ as a quotient of $X$ by the equivalence relation identifying Zariski-infinitesimally close pairs of points of $X$. The geometry of $X_{\dR}$ in this case can therefore be understood by means of geometry of groupoids internal to $\Sch$, i.e. equivariant geometry.

    \subsection{Algebraic de Rham cohomology as quasi-coherent cohomology}
        The reason we care about $X_{\dR}$ is that when $X$ is over characteristic $0$, its big Zariski site is the same as the crystalline/infinitesimal site of $X$, and therefore its quasi-coherent cohomology coincides with the de Rham cohomology of $X$. Then, under certain conditions (e.g. $X$ smooth and proper over a field of characteristic $0$), the $5$-functor pull-push for quasi-coherent modules will automatically induce such a formalism for algebraic de Rham cohomology, thereby conceptually simplifying the latter. Applying cohomological duality then yields a pull-push formalism for D-modules on $X$.

            \input{Formal geometry/D-modules/D_modules}

    \part{Moduli problems and representability}
        \begin{convention}
            Fix a base scheme $S$ and consider a Grothendieck topology $\tau$ on $\Sch_{/S}$, and for technical reasons, we require that $\tau$ is \textit{subcanonical}\footnote{The reader can assume that it is either the fpqc topology or finer.}. For brevity, let us denote the resulting \textit{small} site by $S_{\tau}$.
    
            If $\calS$ is an algebraic space over $S$ then by $\Sch_{/\calS}$, we shall mean the full subcategory of $\Sh( S_{\fppf} )$ spanned by morphisms $X \to \calS$ from scheme $X$. Should $\Sch_{/S}$ be endowed with any subcanonical Grothendieck topology $\tau$, then that topology will also induce another on $\Sch_{/\calS}$. The resulting small site will be denoted by $\calS_{\tau}$.

            Finally, write $\Sh(S_{\tau})$ for the sheaf topos on $S_{\tau}$ (with $S$ being either a scheme or algebraic space), and $\Stk(S_{\tau}) := \Grpd(\Sh(S_{\tau}))$ for the $(2, 1)$-category of $\Grpd$-valued stacks on $S_{\tau}$. 
        \end{convention}

        \chapter{\texorpdfstring{$\Quot, \Hilb$, and Grasmannians}{}}
            \begin{abstract}
                
            \end{abstract}
            
            \minitoc

            \section{\texorpdfstring{$\Quot$ and $\Hilb$}{}}
    \subsection{Sheaf hom between quasi-coherent modules}
        \begin{convention}
            Let $S$ be a scheme and $f: X \to B$ be a morphism of algebraic spaces over $S$. If $\scrM$ is a quasi-coherent $\scrO_X$-module, then we will be writing:
                $$\scrM_T := \pr_1^*\scrM$$
            where $\pr_1: X \x_{f, B} T \to X$ is the canonical projection onto the first factor, for any $T \in \Ob( \Sch_{/B} )$.
        \end{convention}

        The object of interest in this subsection is the following. Let $\scrF, \scrG$ be quasi-coherent and then consider the presheaf $\Maps_{\scrO_X}(\scrF, \scrG)$ on $\Sch_{/B}$ whose functor of points is given by:
            $$\Maps_{\Sh(X_{\tau})}(\scrF, \scrG)(T) := \Hom_{\scrO_{X_T}}(\scrF_T, \scrG_T)$$
        for all $T \in \Ob(\Sch_{/B})$. We write $\Maps$ to avoid confusion with the $\scrO_X$-module of $\scrO_X$-linear homomorphisms $\scrF \to \scrG$, which shall be denoted by $\Hom_{\scrO_X}(\scrF, \scrG)$ as usual; in other words, we are regarding $\scrF, \scrG$ as objects of $\Sh(X_{\tau})$, rather than those of $\QCoh(X)$. It is easy to see that $\Maps_{\Sh(X_{\tau})}(\scrF, \scrG)$ satisfies $\tau$-descent whenever $\tau$ is the fpqc topology or coarser.

        \begin{question}
            Does there exist a subcanonical topology $\tau$ finer than the fpqc topology, such that $\Maps_{\Sh(X_{\tau})}(\scrF, \scrG)$ is still a sheaf ?
        \end{question}

        A good first step in checking whether or not a given sheaf $F: B_{\tau}^{\op} \to \Sets$ is representable by a scheme or algebraic space is to check if said sheaf preserves \textit{cofiltered} limits, i.e. if given a cofiltered limit:
            $$\projlim_{i \in \calI} T_i \in B_{\tau}$$
        there will be a bijection:
            $$\indlim_{i \in \calI} F(T_i) \xrightarrow[]{\cong} F( \projlim_{i \in \calI} T_i )$$
        Without any loss of generality, one can also assume that each scheme $T_i$ is affine.
        \begin{lemma}
            Let $\scrF$ be of finite presentation and $f: X \to B$ be qcqs. $\Maps_{\Sh(X_{\tau})}(\scrF, \scrG)$ will then preserve cofiltered limits.
        \end{lemma}
            \begin{proof}
                
            \end{proof}

    \subsection{\texorpdfstring{$\Isom$}{}}

    \subsection{The scheme \texorpdfstring{$\Quot$}{}}

    \subsection{The scheme \texorpdfstring{$\Hilb$}{}}

            \section{Mapping spaces}
    \subsection{Mapping spaces between proper algebraic spaces}
        A natural question in algebraic geometry is as follows:
        \begin{question}
            Given two schemes/algebraic spaces/algebraic stacks $X, Y$ over $S$, when is the mapping space $\Maps(X, Y)$ representable by a scheme/algebraic space/algebraic stacks ? Recall that the functor of points of $\Maps(X, Y)$ is given by:
                $$\Maps(X, Y)(T) := \Maps(X \x_S T, Y \x_S T)$$
            for any $S$-scheme $T$.
        \end{question}

        Many attempts have been made towards this question, usually with various practical assumptions imposed upon $X$ and $Y$. The proof techniques can vary in complexity, depending on the assumptions on $X$ and $Y$, but in most cases, the general strategy is to use Artin's Criteria for Representability, which is a deformation-theoretic method. One salient advantage of using this method is that the tangent space at a given fibre of a point $f \in \Maps(X_y, y) := \Maps(X, Y)(y)$ - where $y \in |Y|$ is some point and $X_y := X \x_Y y$ - is relatively easy to describe: it is nothing but:
            $$T_{\Maps(X_y, y), f} \cong H^0(X_{\Zar}, \Hom_{\scrO_{X_y}}( f^*\Omega^1_{X_y/y}, \scrO_{X_y} ))$$
        
        Grothendieck, for instance, was interested in maps between projective schemes, and made use of his Existence Theorem that was developed during his work on the Hilbert and Quot schemes in order to prove representability of mapping spaces. An easy but nevertheless important notion in this proof is that of the graph of a morphism in an arbitrary category. It is a natural generalisation of the definition of functions as certain ordered pairs as in material set theory. 
        \begin{definition}[Graph of morphisms] \label{def: graphs_of_morphisms}
            (Cf. \cite[\href{https://stacks.math.columbia.edu/tag/024T}{Tag 024T}]{stacks}) Let $\C$ be a category and $f: X \to Y$ be a morphism therein. Its \textbf{graph}, commonly denoted by $\Gamma(f)$ is then obtained via:
                $$
                    \begin{tikzcd}
                    {\Gamma(f)} & Y \\
                    {X \x Y} & {Y \x Y}
                    \arrow[from=1-1, to=1-2]
                    \arrow[from=1-1, to=2-1]
                    \arrow["\lrcorner"{anchor=center, pos=0.125}, draw=none, from=1-1, to=2-2]
                    \arrow["{\Delta_Y}", from=1-2, to=2-2]
                    \arrow["{f \x \id_Y}", from=2-1, to=2-2]
                    \end{tikzcd}
                $$
            \textit{should the products and pullback exist in the first place}; here, $\Delta_Y: Y \to Y \x Y$ is the diagonal morphism.
        \end{definition}
        \begin{remark}
            When $\C := \Sch_{/S}$, then we see that if $Y$ is separated over $S$, i.e. $\Delta_{Y/S}: Y \to Y \x_S Y$ is a closed immersion by definition, then because closed immersions are preserved by pullbacks, the canonical map:
                $$\Gamma(f) \to X \x_S Y$$
            coming from a morphism of $S$-schemes $f: X \to Y$ will also be a closed immersion. In other words, if $Y$ is separated, then the graph of $f$ will be a closed subscheme of the \say{plane} whose two \say{axes} are $X$ and $Y$.
        \end{remark}
        Recall firstly that the Hilbert moduli problem is given in the following manner. Suppose for a moment that $X, B$ are algebraic spaces over a fixed base scheme $S$ and that $f: X \to B$ is a morphism of finite presentation over $S$. Then, the \textbf{Hilbert moduli problem} will be the presheaf on $B_{\fpqc}$ given by:
            $$
                \Hilb_{X/S}(T) :=
                \left\{
                    \begin{array}{cc}
                         & \text{closed immersions $Z \hookrightarrow X \x_{B} T$ such that}
                         \\
                         & \text{the canonical composition $Z \to X \x_{B} T \to T$ is}
                         \\
                         & \text{of finite presentation, flat, and proper}
                    \end{array}
                \right\}
            $$
        for all $T \in \Ob(B_{\fpqc})$.
        \begin{theorem}[Grothendieck's theorem on mapping space between projective schemes]
            Let $S$ be a base scheme, let $X, Y$ be projective $S$-scheme, and suppose that $X$ is flat over $S$. Then, $\Maps_{\Sch_{/S}}(X, Y)$ will be a quasi-projective $S$-scheme.
        \end{theorem}
            \begin{proof}[Proof sketch]
                If we take for granted the representability of Hilbert schemes (\textit{the proof of which requires deformation theory}), then the strategy to prove that $\Maps_{\Sch_{/S}}(X, Y)$ is representable by an $S$-scheme (namely, a quasi-projective one) hinges on proving that there is a monomorphism of sheaves on $S_{\fpqc}$:
                    $$\Gamma: \Maps_{\Sch_{/S}}(X, Y) \hookrightarrow \Hilb_{X \x_S Y/S}$$
                that is representable by open immersions of schemes.

                For any $T \in \Ob(S_{\fpqc})$, we claim that the corresponding component:
                    $$\Gamma_T: \Maps_{\Sch_{/S}}(X, Y)(T) \hookrightarrow \Hilb_{X \x_S Y/S}(T)$$
                of the natural transformation $\Gamma$ mentioned above maps morphisms of $T$-schemes $f \in \Maps_{\Sch_{/T}}(X \x_S T, Y \x_S T) =: \Maps_{\Sch_{/S}}(X, Y)(T)$ to their graphs $\Gamma_T(f)$ (in the sense of definition \ref{def: graphs_of_morphisms}); see \cite[\href{https://stacks.math.columbia.edu/tag/0D1A}{Tag 0D1A}]{stacks}. To see that this is well-defined, simply note that because $Y$ is projective and hence separated, of finite presentation, flat, and proper by definition, $\Gamma_T(f)$ will always be a closed subscheme of $(X \x_S T) \x_T (Y \x_S T) \cong (X \x_S Y) \x_S T$ that is of finite presentation, flat, and proper over $T$, i.e. $\Gamma_T(f)$ is indeed nothing but an element of $\Hilb_{X \x_S Y/S}(T)$.

                Now, to see that $\Gamma$ is representable by open immersions, we will need to show that given any morphism of schemes $t: T \to \Hilb_{X \x_S Y/S}$, the canonical projection:
                    $$\Maps_{\Sch_{/S}}(X, Y) \x_{ \Gamma, \Hilb_{X \x_S Y/S}, t } T \to T$$
                is an open immersion of schemes. To this end, choose some:
                    $$(i: Z \hookrightarrow (X \x_S Y) \x_S T) \in \Hilb_{X \x_S Y/S}(T)$$
                We now know that the closed immersion $i: Z \hookrightarrow (X \x_S Y) \x_S T$ is the graph of a morphism of $T$-schemes $f: X \x_S T \to Y \x_S T$, i.e. of an element $f \in \Maps_{\Sch_{/S}}(X, Y)(T)$, if and only if the canonical projection:
                    $$\pr_{X \x_S T}: Z \to X \x_S T$$
                is an isomorphism of $T$-schemes. Then, using the Factorisation Criterion for Isomorphisms \cite[\href{https://stacks.math.columbia.edu/tag/05XD}{Tag 05XD}]{stacks}, we will be able to conclude the argument.  

                Lastly, note that because an open subsheaf of a scheme is itself a scheme, $\Maps_{\Sch_{/S}}(X, Y)$ is now necessarily representable by a scheme. \textit{A priori}, $\Hilb_{X \x_S Y/S}$ is a projective $S$-scheme, so $\Maps_{\Sch_{/S}}(X, Y)$ is quasi-projective over $S$ by definition.
            \end{proof}
        There is a mild generalisation of the above to the setting of algebraic spaces that does away with some of the unnecessary hypotheses, which is perhaps useful when one is concerned with mapping spaces between certain quotients (e.g. by finite group actions, which arise naturally over positive characteristics).
        \begin{proposition}[Mapping spaces between proper algebraic spaces]
            
        \end{proposition}

            \section{Grassmannians, and a bit about uniformisation}
    
        \chapter{Moduli stacks of principal bundles}
            \begin{abstract}
                
            \end{abstract}
            
            \minitoc

            \input{Moduli theory/Bun_G/algebraicity_of_Bun_G}

            \section{Global geometry of \texorpdfstring{$\Bun_G$}{} of a curve}
    \subsection{Level structures and automorphic realisation}

    \subsection{Smoothness of \texorpdfstring{$\Bun_G$}{} of a curve}

        \chapter{Equivariant geometry and geometric invariant theory}
            \begin{abstract}
                
            \end{abstract}

            \section{Good moduli spaces}
    \subsection{Cohomologically affine morphisms}
        \begin{definition}[Cohomologically affine morphisms] \label{def: cohomologically_affine_morphisms}
            We say that a qcqs\footnote{We need this assumption so that $f_*: \scrO_{\scrX}\mod \to \scrO_{\scrY}\mod$ would preserve quasi-coherence.} morphism of algebraic stacks:
                $$f: \scrX \to \scrY$$
            is \textbf{cohomologically affine} if and only if the corresponding $\QCoh$-$*$-pushforward functor:
                $$f_*: \QCoh(\scrX) \to \QCoh(\scrY)$$
            is exact. 

            An algebraic stack over a scheme is cohomologically affine if and only if its structural morphism is cohomologically affine.
        \end{definition}
        If $\calX, \calY$ are algebraic spaces, then a morphism $f: \calX \to \calY$ is cohomologically affine if and only if it is affine; this is Serre's Criterion for Affineness. However, for algebraic stacks $\scrX, \scrY$, the same equivalence might fail. Namely, one can have a cohomologically affine morphism:
            $$f: \scrX \to \scrY$$
        \textit{without} having that:
            $$R^if_*\scrF \cong 0$$
        for all quasi-coherent $\scrO_{\scrX}$-modules $\scrF$ and all $i > 0$, as the following example illustrates.
        \begin{example}[Cohomologically affine but not affine] \label{example: cohomologically_affine_but_not_affine}
            Suppose that $G \to \Spec k$ is a group scheme over some field $k$, and consider its classifying stack $[\Spec k/G]$, which is \textit{a priori} algebraic. Using the fact that there is an equivalence:
                $$\Gamma: \QCoh([\Spec k/G]) \xrightarrow[]{\cong} kG\mod$$
            given by taking global sections, one sees thus that:
                $$H^i([\Spec k/G], \scrF) \cong \Ext^i_{kG}(k, \Gamma(\scrF))$$
            where on the RHS, $k$ is regarded as a trivial $kG$-module. Next, let:
                $$\scrF := \gamma_*\scrO_{\Spec k}$$
            for some choice of a point $\gamma: \Spec k \to [\Spec k/G]$. The global section of this sheaf of $\scrO_{[\Spec k/G]}$-modules is nothing but $k$ but with some non-trivial $G$-action, depending on the choice of $\gamma$. From this, one infers that:
                $$\Ext^i_{kG}(k, \Gamma(\gamma_*\scrO_{\Spec k}))$$
            is generally non-zero when $i > 0$ (in particular, when $i = 1$). At the same time, the canonical functor:
                $$\QCoh(\Spec k) \to kG\mod$$
            realising $k$-vector spaces as trivial $kG$-modules is trivially exact, and hence so is the $\QCoh$-$*$-pushforward functor:
                $$\gamma_*: \QCoh(\Spec k) \to \QCoh([\Spec k/G])$$
            Thus, we have found morphisms:
                $$\gamma: \Spec k \to [\Spec k/G]$$
            that are cohomologically affine but not affine.
        \end{example}
        That said, a fairly large class of algebraic stacks do admit morphisms between them that are cohomologically affine if and only if they are affine.
        \begin{proposition}[Affine when cohomologically affine ?]
            Let $f: \scrX \to \scrY$ be a morphism of algebraic stacks. The condition that $f$ is cohomologically affine is equivalent to $f$ being affine if and only if $\scrX$ and $\scrY$ both have affine diagonals.
        \end{proposition}
        \begin{example}
            If $G$ is an affine algebraic group over a field $k$, then $[\Spec k/G]$ will have affine diagonal, and hence any $k$-point $\Spec k \to [\Spec k/G]$ will be affine on top of being cohomologically affine.
        \end{example}

    \subsection{Cohomological ampleness and projectivity}

    \subsection{Good moduli spaces}

    \subsection{Tame stacks and tame moduli spaces}

            \input{Moduli theory/Equivariant geometry/adequate_moduli_spaces}

            \section{Equivariance via stacks}
    Throughout, let $S$ be a base scheme.

    The main theorem of this section is as follows:
    \begin{theorem}
        Let $G$ be an affine and (cohomologically) reductive group $S$-scheme acting on an algebraic space $\calX$ over $S$. Suppose also that there is a $G$-\textit{invariant} morphism:
            $$f: \calX \to Z$$
        of algebraic spaces over $S$. Then, there will exist a good quotient:
            $$\pi: \calX \to Y$$
        of algebraic spaces over $S$.
    \end{theorem}

    \subsection{Equivariant geometry vs. geometry of quotient stacks}
        The equivariant-stack dictionary is as follows, at least when $S := \Spec k$ is the spectrum of an algebraically closed field $k$ (so that $k$-points and closed points will be the same) and when $G$ is an affine algebraic group $k$-scheme acting on a \textit{finite-type} $k$-scheme $X$. Note that in this case, we have that $[X/G]$ is an algebraic stack over $S$ (and in particular, the classifying $S$-stack $[\Spec k/G]$ is algebraic), so it makes sense to speak of $\QCoh([X/G])$.
        \begin{table}[H]
            \centering
            \begin{tabular}{@{}|l|l|lll@{}}
                \text{$G$-orbit of $x \in X(\Spec k)$} & \text{points $[x] \in [X/G](\Spec k)$} \\
                \text{$G$-invariant morphisms $X \to Z$} & \text{morphisms to schemes $[X/G] \to Z$} \\
                \text{invariant submodules $H^i(X, -)^G$} & \text{sheaf cohomologies $H^i([X/G], -)$} \\
                \text{geometric/good quotients $X \to Y$} & \text{coarse/good moduli space $[X/G] \to Y$}
            \end{tabular}
            \caption{$G$-equivariant geometry vs. geometry of $[X/G]$. Note: the last comparison is only apt when the $G$-action on $X$ is proper.}
            \label{table: equivariant_geometry_vs_stack_geometry}
        \end{table}

        \begin{example}[Representations as sheaves on classifying stacks] \label{example: representations_as_sheaves_on_classifying_stacks}
            The category of $k$-linear $G$-representations, i.e. $G$-equivariant $k$-vector spaces, is equivalent to $\QCoh(\Spec k)^G$, which in turn is the same as $\QCoh([\Spec k/G])$. It can also be shown that $\QCoh([\Spec k/G])$ is equivalent to finite-dimensional $k$-linear representations of $G$.
        \end{example}

        Now, recall from example \ref{example: classifying_stacks} that if $H$ is any group $S$-scheme then $[S/H]$ will classify principal $H$-bundles, in the sense that if $T$ is any $S$-scheme, then:
            $$[S/H](T)$$
        will be the groupoid of principal $H$-bundles on $T$. Next, recall also that open subgroup schemes are closed, so when discussing subgroups of $G$, we can assume without loss of generality that they are closed. If $H \leq G$ is a closed subgroup of $G$,   

        \begin{example}[Borel-Weil-Bott Theorem]
            Let $k$ be an algebraically closed field of characteristic $0$ and let $G$ be a connected reductive group; for convenience, let us write:
                $$\pt := \Spec k$$
            
            Choose a (positive) Borel subgroup $B^+ \leq G$ and denote the corresponding maximal torus of $G$ by $H$. We now know that the sheaf quotient $G/B^+$ arises from the $(2, 1)$-pullback:
                $$
                    \begin{tikzcd}
                    	{G/B^+} & {[\pt/B^+]} \\
                    	\pt & {[\pt/G]}
                    	\arrow[from=1-1, to=1-2]
                    	\arrow[from=1-1, to=2-1]
                    	\arrow["{(2, 1)}"{description}, "\lrcorner"{anchor=center, pos=0.125}, draw=none, from=1-1, to=2-2]
                    	\arrow[from=1-2, to=2-2]
                    	\arrow[from=2-1, to=2-2]
                    \end{tikzcd}
                $$
            and that because the morphism of algebraic stacks $[\pt/B^+] \to [\pt/G]$ is representable by schemes, finitely presented, and quasi-projective, $G/B^+$ as a $k$-scheme is also finitely presented and quasi-projective. 
        \end{example}

        \begin{example}[Beilinson-Bernstein Localisation]
            
        \end{example}

    \subsection{Recovering Matsushima's Theorem}
        \begin{proposition}[Matsushima's Theorem via stacks]
            Let $G$ be an affine and (cohomologically) reductive group $S$-scheme and let $H \leq G$ be a flat, finitely presented, and separated $S$-subgroup. In that case, the following statements are equivalent:
            \begin{enumerate}
                \item $H$ is cohomologically reductive.
                \item $G/H$ is affine over $S$.
                \item The induction functor $\Ind_H^G: \QCoh(S)^H \to \QCoh(S)^G$ is exact. 
            \end{enumerate}
        \end{proposition}

    \subsection{Observable subgroups}
        Recall that if $f: \calX \to \calY$ is a morphism of algebraic stacks representable by (affine) schemes, then it is \textbf{quasi-affine} if and only if the adjunction counit $f^* \circ f_* \Rightarrow \id_{\QCoh(\calX)}$ is a natural isomorphism.
    
        \begin{proposition}
            Let $G$ be a flat, finitely presented, and quasi-affine group scheme and let $H \leq G$ be a flat, finitely presented, and quasi-affine $S$-subgroup. In that situation, the following are equivalent:
            \begin{enumerate}
                \item $H$ is observable.
                \item The adjunction counit $\Ind_H^G \circ \Res_H^G \Rightarrow \id_{\QCoh(S)^G}$ is a surjective natural transformation.
                \item The morphism of algebraic stacks $[S/H] \to [S/G]$ is quasi-affine. 
                \item $G/H$ is quasi-affine over $S$.
            \end{enumerate}
            If, furthermore, the base scheme $S$ is Noetherian (in which case, the adjunction $\Ind_H^G \ladjoint \Res_H^G$ restricts down to between $\QCoh(S)^G$ and $\QCoh(S)^H$), then the statements above will also be equivalent to the following:
            \begin{itemize}
                \item The adjunction counit $\Ind_H^G \circ \Res_H^G \Rightarrow \id_{\Coh(S)^G}$ is a surjective natural transformation.
            \end{itemize}
        \end{proposition}

    \subsection{Existence of good quotients}
        \begin{proposition}
            Let $G$ be a connected algebraic group over $S$ acting on an $S$-scheme $X$, and suppose that, for every pair of points $x, y \in |X|$, there exists a $G$-invariant open neighbourhood $U_{x, y} \subseteq X$ containing said points. Suppose furthermore that $U_{x, y}$ admits a good quotient. Then, $X$ itself will also admit a good quotient.
        \end{proposition}
	
    \addcontentsline{toc}{section}{References}
    \printbibliography

\end{document}