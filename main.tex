\input{book preambles}

\setcounter{section}{-1}

\input{commands}

\begin{document}

	\title{Moduli problems and deformation theory}
	
	\author{Dat Minh Ha}
	\maketitle
	
	\begin{abstract}
	    
	\end{abstract}
	
	{
      \hypersetup{} 
      \dominitoc
      \tableofcontents %sort sections alphabetically
    }
    
    \chapter{Schemes, algebraic spaces, and algebraic stacks}
        \begin{abstract}
            
        \end{abstract}
        
        \minitoc
    
        \section{Schemes}
    \subsection{The Zariski topology and affine schemes}
        \subsubsection{The prime spectrum of a commutative ring} \index{$\Spec$}
            \begin{definition}[Spectra of commutative rings]
                To any commutative ring $R$, let us \textit{contravariantly functorially} associate, first of all, a set $\Spec R$ consisting of all prime ideals of $R$. This is called the spectrum, or the prime spectrum, of $R$. In other words, $\Spec$ is a functor from $\Cring^{\op}$ to $\Sets$.
            \end{definition}
            \begin{remark}[Why is $\Spec$ a contravariant functor ?]
                One can very well construct the theory of affine schemes based on an alternative \textit{covariant} functor $"\Spec": \Cring \to \Sets$, which assigns to a commutative ring its set of prime ideals too. However, this would not have made our lives easy, as the images under ring homomorphisms of a prime ideal is not necessarily prime, whereas the preimage under ring homomorphisms of a prime ideal is always prime. In particular, this means that requiring $\Spec$ to be a contravariant functor ensures that to-be morphisms between affine schemes would always exist, given that the correpsonding morphism between commutative rings exists. In fact, this also proves that $\Spec$ is a well-defined functor, as it guarantees that each commutative ring homomorphism $f: A \to B$ is sent to a \textit{function} $\Spec f$ that maps each element $\q \in \Spec B$ to a unique element $(\Spec f)(\q)$ of $\Spec A$.
            \end{remark}
            \begin{example}[Some interesting underlying sets of ring spectra] \label{example: spectra_sets}
                \noindent
                \begin{enumerate}
                    \item \textbf{(Spectra of fields):} If $k$ is any field, then $\Spec k = \{(0)\}$. To see why this is the case, first of all, let $I$ be an ideal of $k$ that is neither $(0)$ nor $k$. We know that ideals are closed under linear combinations; in this particular instance, $I$ is closed under $k$-linear combinations. Thus, if we view $k$ as a vector space over itself, then $I$ must be a non-zero proper subspace of $k$, since $I$ is a subset of $k$ that is closed under $k$-linear combinations (one could also argue that by the first isomorphism theorem, $I$ is the kernel of some ring homomorphism whose domain is $k$, and we know that kernels are subspaces). Either way, this would mean that:
    					$$0 = \dim_k 0 < \dim_k I < \dim_k k = 1$$
    				and since dimensions of vector spaces are natural numbers, there will be no such natural number $\dim_k I$, i.e. $I$ does not exist. Note that the zero ideal $(0)$ is trivially the zero subspace $0$ of $k$.
    				\item \textbf{(Spectrum of the zero ring):} $\Spec 0 = \varnothing$, because prime ideals are defined to be proper.
    				\item \textbf{(The zero ideal):} The zero ideal is not necessarily prime; as a matter of fact, it is only so inside an integral domain. When zero-divisors are present, say in $\Z/n\Z$ with $n$ composite, the statement:
    				    $$xy \in (0)$$
				    might not imply that $x = y = 0$, but instead, that $x \in (p)$ and $y \in (q)$, with $p, q$ integers such that $pq = n$.
    				\item \textbf{(The complex affine line):} Due to the algebraic closure of $\bbC$, the points of the complex affine line $\A^1_{\bbC} := \Spec \bbC[x]$ (with $t$ some formal variable) are prime ideals of the form $(x - a)$ wherein $a \in \bbC$, along with the prime ideal $(0)$. To see in more details why this is the case, let us first recall because $\bbC$ is a separably closed field, every single-variable polynomial over $\bbC$ splits completely into linear factor. This, along with the fact that $\bbC[x]$ is a PID, tells us that ideals of $\bbC[x]$ are actually all contained in \textit{principal} ideals generated by linear polynomials; note that these principal ideals are prime, precisely because $\bbC$ is separably closed. Lastly. because $\bbC$ is algebraically closed, there is a bijective correspondence between the principal ideals generated by linear polyonomials $(x - a)$ and the individual complex numbers $a \in \bbC$. Thus, the prime ideals of $\bbC[x]$ are either of the form $(0)$ or $(x - a)$, wherein $a \in \bbC$. 
    				\\
    				Below is an illustration by Ravi Vakil of the complex affine line:
    				    \begin{figure}[H]
    				        \centering
    				        \includegraphics[width=\linewidth,height=\textheight,keepaspectratio]{Figures/complex affine line.png}
    				        \caption{The complex affine line $\A^1_{\bbC}$ (\cite{risingsea}, figure 3.1)}
    				        \label{fig: complex_affine_line}
    				    \end{figure}
    				This result generalises in an obvious manner to algebraically closed fields other than $\bbC$; so for instance, studying schemes over the field $\overline{\Q}$ of algebraic numbers might help one understand more about polynomials with rational coefficients.
    				\item \textbf{(The affine line over a separably closed but not algebraically closed field):} Let $k$ be a field that is separably closed but not algebraically closed (we can take $k = \F_p(t)^{\sep}$, for example). Then, the set of non-zero prime ideals of $\A^1_k$ need not be in bijection with $k$ itself.
    				\item \textbf{(Spectrum of the integers):} The prime ideals of $\Z$ are either generated by prime numbers themselves, or the zero ideal $(0)$. Thus, the set $\Spec \Z$ is in bijection with the \textit{union} of the set of all prime numbers and the set $\{(0)\}$. 
                \end{enumerate}
            \end{example}
        
        \subsubsection{The Zariski topology as a point-set topology} \index{Topology!Zariski}
            \begin{definition}[Zariski-closed subsets] \label{def: zariski_closed}
                Let $R$ be a commutative ring and let $\calF$ be an arbitrary subset of $R$. Then, let us declare that sets of the following form are closed in the to-be Zariski topology:
                    $$V(\calF) := \left\{\p \in \Spec R \mid \p \supset \calF \right\}$$
                When it might be possible to confuse Zariski-closed subsets of different ring spectra, we will write $V_R(\calF)$ instead of simply $V(\calF)$.
            \end{definition}
            \begin{remark}
                Of course, sets of the form $\Spec R \setminus V(\calF)$ are Zariski-open (that is, if we are assuming that the \href{https://ncatlab.org/nlab/show/excluded+middle}{\underline{the Law of Excluded Middle}} holds).
            \end{remark}
            
            \begin{proposition}[Well-definiteness of the Zariski topology] \label{prop: zariski_closed_well_definiteness}
                Let $R$ be an arbitrary commutative ring and assume the Law of Excluded Middle. Then, Zariski-closed subsets as defined in \ref{def: zariski_closed} actually define a topology on $\Spec R$, which of course, is called the Zariski topology.
            \end{proposition}
                \begin{proof}
                    For each subset $\calF$ of $R$, let us write $I(\calF)$ for the $R$-ideal generated by $\calF$. Let us now verify the axioms defining topologies on sets one-by-one.
                        \begin{enumerate}
                            \item \textbf{(The empty set and the whole set are closed):} The empty set is just $V(R)$ and that $\Spec R$ is just $V\left((0)\right)$, which are, by definition, closed in the Zariski topology. Thus, both the empty set and the whole space are Zariski-closed.
                            \item \textbf{(Finite unions of closed sets are closed):} Let $\{V(\calF_{\alpha})\}_{\alpha \in A}$ be a \textit{finite} set of Zariski-closed subsets of $\Spec R$ and consider the following chain of logical \textit{implications} (wherein $\p$ is a prime ideal of $R$, even though this fact can be inferred from the statements themselves):
                                $$
                                    \begin{aligned}
                                        & \p \in \bigcup_{\alpha \in A} V(\calF_{\alpha})
                                        \\
                                        \iff & \exists \alpha \in A: \p \in V(\calF_{\alpha})
                                        \\
                                        \iff & \exists \alpha \in A: \p \supset \calF_{\alpha}
                                        \\
                                        \iff & \bigvee_{\alpha \in A} (\p \supset \calF_{\alpha})
                                        \\
                                        \implies & \forall \left(f_{\alpha}\right)_{\alpha \in A} \in \prod_{\alpha \in A} \calF_{\alpha}: \prod_{\alpha \in A} f_{\alpha} \in \p
                                        \\
                                        \iff & \bigwedge_{\left(f_{\alpha}\right)_{\alpha \in A} \in \prod_{\alpha \in A} \calF_{\alpha}} \left(\prod_{\alpha \in A} f_{\alpha} \in \p\right)
                                        \\
                                        \iff & \p \supset \left\{\prod_{\alpha \in A} f_{\alpha} \: \bigg| \: \forall \alpha \in A: f_{\alpha} \in \calF_{\alpha} \right\}
                                        \\
                                        \iff & \p \in V\left(\left\{\prod_{\alpha \in A} f_{\alpha} \: \bigg| \: \forall \alpha \in A: f_{\alpha} \in \calF_{\alpha} \right\}\right)
                                    \end{aligned}
                                $$
                            wherein the fourth line, in particular, holds due to the fact that ideals, by definition, are closed under scalar multiplication by elements of their ambient rings. Now, to upgrade the fourth line to an equivalence, we can show that:
                                $$\bigwedge_{\left(f_{\alpha}\right)_{\alpha \in A} \in \prod_{\alpha \in A} \calF_{\alpha}} \left(\prod_{\alpha \in A} f_{\alpha} \in \p\right) \implies \bigvee_{\alpha \in A} (\p \supset \calF_{\alpha})$$
                            or, as we have assumed that the Law of Excluded Middle holds, we have the following:
                                $$
                                    \begin{aligned}
                                        & \left(\bigwedge_{\left(f_{\alpha}\right)_{\alpha \in A} \in \prod_{\alpha \in A} \calF_{\alpha}} \left(\prod_{\alpha \in A} f_{\alpha} \in \p\right) \implies \bigvee_{\alpha \in A} (\p \supset \calF_{\alpha})\right)
                                        \\
                                        \vdash & \left(\neg \bigwedge_{\left(f_{\alpha}\right)_{\alpha \in A} \in \prod_{\alpha \in A} \calF_{\alpha}} \left(\prod_{\alpha \in A} f_{\alpha} \in \p\right) \implies  \neg \bigvee_{\alpha \in A} (\p \supset \calF_{\alpha})\right)
                                    \end{aligned}
                                $$
                            meaning that we can prove the contraposition instead. To that end, consider the following:
                                $$
                                    \begin{aligned}
                                        & \neg \bigwedge_{\left(f_{\alpha}\right)_{\alpha \in A} \in \prod_{\alpha \in A} \calF_{\alpha}} \left(\p \ni \prod_{\alpha \in A} f_{\alpha}\right)
                                        \\
                                        \implies & \bigvee_{\left(f_{\alpha}\right)_{\alpha \in A} \in \prod_{\alpha \in A} \calF_{\alpha}} \neg \left(\p \ni \prod_{\alpha \in A} f_{\alpha}\right)
                                        \\
                                        \implies & \bigwedge_{\alpha \in A} \left(\bigvee_{f_{\alpha} \in \calF_{\alpha}} \neg(\p \ni f_{\alpha})\right)
                                        \\
                                        \implies & \bigwedge_{\alpha \in A} \left(\neg \bigwedge_{f_{\alpha} \in \calF_{\alpha}} (\p \ni f_{\alpha})\right)
                                        \\
                                        \implies & \bigwedge_{\alpha \in A} \neg (\p \supset \calF_{\alpha})
                                        \\
                                        \implies & \neg \bigvee_{\alpha \in A} (\p \supset \calF_{\alpha})
                                    \end{aligned}
                                $$
                            Thus, we have managed to show that:
                                $$\neg \bigwedge_{\left(f_{\alpha}\right)_{\alpha \in A} \in \prod_{\alpha \in A} \calF_{\alpha}} \left(\prod_{\alpha \in A} f_{\alpha} \in \p\right) \implies  \neg \bigvee_{\alpha \in A} (\p \supset \calF_{\alpha})$$
                            and therefore:
                                $$\p \in \bigcup_{\alpha \in A} V(\calF_{\alpha}) \iff \p \in V\left(\left\{\prod_{\alpha \in A} f_{\alpha} \: \bigg| \: \forall \alpha \in A: f_{\alpha} \in \calF_{\alpha} \right\}\right)$$
                            Because $\p$ was chosen arbitrarily, this implies that:
                                $$\bigcup_{\alpha \in A} V(\calF_{\alpha}) = V\left(\left\{\prod_{\alpha \in A} f_{\alpha} \: \bigg| \: \forall \alpha \in A: f_{\alpha} \in \calF_{\alpha} \right\}\right)$$
                            Hence, the \textit{finite} union $\bigcup_{\alpha \in A} V(\calF_{\alpha})$ is Zariski-closed by definition, and consequently, all finite unions of Zariski-closed sets are closed in the Zariski topology themselves (since $\{\calF_{\alpha}\}_{\alpha \in A}$ is an arbitrary fintie set of subsets of $R$). Note that the finiteness assumption on the index set $A$ is crucial, as without it, one would not be able to properly make sense of the product $\prod_{\alpha \in A} f_{\alpha}$.
                            \item \textbf{(Intersections of closed sets are closed):} Let $\{V(\calF_{\alpha})\}_{\alpha \in A}$ be an \textit{arbitrary} set of Zariski-closed subsets of $\Spec R$ and consider the following chain of logical equivalences (wherein $\p$ is a prime ideal of $R$, and again, this fact can be deduced from the statements themselves):
                                $$
                                    \begin{aligned}
                                        & \p \in \bigcap_{\alpha \in A} V(\calF_{\alpha})
                                        \\
                                        \iff & \forall \alpha \in A: \p \in V(\calF_{\alpha})
                                        \\
                                        \iff & \forall \alpha \in A: \p \supset \calF_{\alpha}
                                        \\
                                        \iff & \p \supset I\left(\bigcup_{\alpha \in A} \calF_{\alpha}\right)
                                        \\
                                        \iff & \p \in V\left(I\left(\bigcup_{\alpha \in A} \calF_{\alpha}\right)\right)
                                    \end{aligned}
                                $$
                            It tells us that \textit{any} prime ideal $\p$ is in an intersection of Zariski-closed subsets of $\Spec R$ defined by subsets $\calF_{\alpha}$ of $R$ if and only if it is in the Zariski-closed subset of $\Spec R$ defined by the ideal generated by the union of the sets $\calF_{\alpha}$, or in other words, that:
                                $$\bigcap_{\alpha \in A} V(\calF_{\alpha}) = V\left(I\left(\bigcup_{\alpha \in A} \calF_{\alpha}\right)\right)$$
                            In turn, this implies that the intersection of the Zariski-closed sets $V(\calF_{\alpha})$ is itself closed in the Zariski topology, and since the index set $A$ is was chosen arbitrarily, this means that arbitrary intersections of Zariski-closed sets are themselves Zariski-closed.
                        \end{enumerate}
                    Thus, with closed sets as in definition \ref{def: zariski_closed}, the Zariski topology on prime spectra of commutative rings is well-defined.
                \end{proof}
            \begin{corollary}[Quotients are closed] \label{coro: quotients_are_closed}
                Let $R$ be a commutative ring and let $I$ be an $R$-ideal. Then, $\Spec R/I$ is homeomorphic to a Zariski-closed subset of $\Spec R$. 
            \end{corollary}
                \begin{proof}
                    This comes from a straightforward application of the third isomorphism theorem for modules.
                \end{proof}
                
            \begin{definition}[A different approach: Zariski-open sets] \label{def: zariski_open}
                Let $R$ be a commutative ring. Then, let us declare that subsets of $\Spec R$ of the following form are open in the to-be Zariski topology:
                    $$D(f) := \{\p \in \Spec R \mid \p \not \ni f\}$$
                Whenever referring to more than one ring spectra, it might be beneficial to specifically write $D_R(f)$ instead of $D(f)$.
            \end{definition}
            \begin{remark} \label{remark: basic_opens_complements}
                For any commutative ring $R$, one can show through the following logical equivalences that:
                    $$D(f) = \Spec R \setminus V\left((f)\right)$$
                wherein $\p$ is an arbitrary prime ideal of $R$:
                    $$
                        \begin{aligned}
                            & \p \in D(f)
                            \\
                            \iff & \p \in \{\q \in \Spec R \mid \q \not \ni f\}
                            \\
                            \iff & \neg(\p \ni f)
                            \\
                            \iff & \neg(\p \supset (f))
                            \\
                            \iff & \p \in \Spec R \setminus \{\q \in \Spec R \mid \q \supset (f)\}
                            \\
                            \iff & \p \in \Spec R \setminus V\left((f)\right)
                        \end{aligned}
                    $$
            \end{remark}
            
            \begin{proposition}[Well-definiteness of the Zariski topology] \label{prop: zariski_open_well_definiteness}
                Let $R$ be an arbitrary commutative ring and assume the Law of Excluded Middle. Then, Zariski-open subsets as defined in \ref{def: zariski_open} actually define a topology on $\Spec R$, which of course, is called the Zariski topology.
            \end{proposition}
                \begin{proof}
                    Let us verify the axioms defining topologies on sets one-by-one.
                        \begin{enumerate}
                            \item \textbf{(The empty set and the whole set are open):} Consider the set $D(1)$, which by definition, is given by:
                                $$D(1) := \{\p \in \Spec R \mid \p \not \ni 1\}$$
                            Because prime ideals are defined to be proper, and because proper ideals are never (multiplicatively) unital, one gets that:
                                $$D(1) = \Spec R$$
                            In other words, the whole of $\Spec R$ is open by definition. Now, consider the following:
                                $$D(0) := \{\p \in \Spec R \mid \p \not \ni 0\} = \varnothing$$
                            which holds because ideals are submodules of their ambient rings, and modules over rings must contain $0$ (as an additive identity) by definition. Thus, the empty set is also Zariski-open by definition. 
                            \item \textbf{(Unions of open sets are open):} Let $\{f_{\alpha}\}_{\alpha \in A}$ be an \textit{arbitrary} set of elements of $R$ and let us apply remark \ref{remark: basic_opens_complements} to get the following chain of logical equivalences regarding the union of the sets $D(f_{\alpha})$, wherein $\p$ is an \textit{arbitrary} prime ideal of $R$:
                                $$
                                    \begin{aligned}
                                        & \p \in \bigcup_{\alpha \in A} D(f_{\alpha})
                                        \\
                                        \iff & \bigvee_{\alpha \in A} \left(\p \in D(f_{\alpha})\right)
                                        \\
                                        \iff & \bigvee_{\alpha \in A} \neg \left(\p \in V\left((f_{\alpha})\right)\right)
                                        \\
                                        \iff & \neg \bigwedge_{\alpha \in A} \left(\p \in V\left((f_{\alpha})\right)\right)
                                        \\
                                        \iff & \p \in \Spec R \setminus \bigcap_{\alpha \in A} V\left((f_{\alpha})\right)
                                    \end{aligned}
                                $$
                            This shows that:
                                $$\bigcup_{\alpha \in A} D(f_{\alpha}) = \Spec R \setminus \bigcap_{\alpha \in A} V\left((f_{\alpha})\right)$$
                            In proposition \ref{prop: zariski_closed_well_definiteness}, we have already shown using only definition \ref{def: zariski_closed} that arbitrary intersections of Zariski-closed sets are Zariski-closed themselves; in particular, this means that $\bigcap_{\alpha \in A} V\left((f_{\alpha})\right)$ is Zariski-closed. Then, by using the Law of Excluded Middle, one can see that the complement $\Spec R \setminus \bigcap_{\alpha \in A} V\left((f_{\alpha})\right)$ is necessarily Zariski-open. Thus, arbitrary unions of Zariski-open sets are Zariski-open themselves.
                            \item \textbf{(Finite intersections of open sets are open):} Let $\{f_{\alpha}\}_{\alpha \in A}$ be an \textit{finite} set of elements of $R$ and consider the following chain of logical equivalences regarding the union of the sets $D(f_{\alpha})$, wherein $\p$ is a prime ideal of $R$:
                                $$
                                    \begin{aligned}
                                        & \neg \left(\p \in \bigcap_{\alpha \in A} D(f_{\alpha})\right)
                                        \\
                                        \iff & \neg \bigwedge_{\alpha \in A} \left(\p \in D(f_{\alpha})\right)
                                        \\
                                        \iff & \bigvee_{\alpha \in A} \neg \left(\p \in D(f_{\alpha})\right)
                                        \\
                                        \iff & \bigvee_{\alpha \in A} \left(\p \in \Spec R \setminus D(f_{\alpha})\right)
                                        \\
                                        \iff & \bigvee_{\alpha \in A} \left(\p \in V\left((f_{\alpha})\right)\right)
                                        \\
                                        \iff & \p \in \bigcup_{\alpha \in A} V\left((f_{\alpha})\right)
                                    \end{aligned}
                                $$
                            (let us note that the fifth line holds thanks to remark \ref{remark: basic_opens_complements}). Now, the contraposition of the equivalence:
                                $$\neg \left(\p \in \bigcap_{\alpha \in A} D(f_{\alpha})\right) \iff \p \in \bigcup_{\alpha \in A} V\left((f_{\alpha})\right)$$
                            is:
                                $$\neg \neg \left(\p \in \bigcap_{\alpha \in A} D(f_{\alpha})\right) \iff \neg \left(\p \in \bigcup_{\alpha \in A} V\left((f_{\alpha})\right) \right)$$
                            From this, one gets the following proof:
                                $$
                                    \begin{aligned}
                                        & \neg \neg \left(\p \in \bigcap_{\alpha \in A} D(f_{\alpha})\right) \iff \neg \left(\p \in \bigcup_{\alpha \in A} V\left((f_{\alpha})\right) \right)
                                        \\
                                        \vdash & \left(\p \in \bigcap_{\alpha \in A} D(f_{\alpha})\right) \iff \left(\p \in \Spec R \setminus \bigcup_{\alpha \in A} V\left((f_{\alpha})\right) \right) 
                                        \\
                                        \vdash & \left(\bigcap_{\alpha \in A} D(f_{\alpha}) = \Spec R \setminus \bigcup_{\alpha \in A} V\left((f_{\alpha})\right)\right)
                                    \end{aligned}
                                $$
                            Lastly, let us recall that by \ref{prop: zariski_closed_well_definiteness}, the \textit{finite} union $\bigcup_{\alpha \in A} V\left((f_{\alpha})\right)$ of Zariski-closed sets is Zariski-closed itself, meaning that by the Law of Excluded Middle, the complement $\Spec R \setminus \bigcup_{\alpha \in A} V\left((f_{\alpha})\right)$ must be Zariski-open. Thus, the union $\bigcap_{\alpha \in A} D(f_{\alpha})$ is Zariski-open. Note that the finiteness assumption on the index set $A$ is crucial, as otherwise, the union $\bigcup_{\alpha \in A} V\left((f_{\alpha})\right)$ might not be Zariski-closed.
                        \end{enumerate}
                    Thus, with open sets as in definition \ref{def: zariski_open}, the Zariski topology on prime spectra of commutative rings is well-defined.
                \end{proof}
            
            \begin{proposition}[Unifying the two definitions] \label{prop: zariski_topology_equivalence}
                By asuming the Law of Excluded Middle, one gets the same topology on spectra of commutative rings via the approaches presented in definitions \ref{def: zariski_closed} and \ref{def: zariski_open}.
            \end{proposition}
                \begin{proof}
                    Let $R$ be a commutative ring. It is sufficient to show that for each element $f \in R$, the complement $\Spec R \setminus D(f)$ is closed in the sense of definition \ref{def: zariski_closed}, or equivalently, for each subset $\calF \subset R$, the complement $\Spec R \setminus V(\calF)$ is open in the sense of definition \ref{def: zariski_open}. We will be attempting the second approach. To that end, let us directly the following chain of logical equivalences:
                        $$
                            \begin{aligned}
                                & \p \in \bigcup_{\alpha \in A} D(f_{\alpha})
                                \\
                                \iff & \exists \alpha \in A: \p \in D(f_{\alpha})
                                \\
                                \iff & \exists \alpha \in A: \p \in \{\q \in \Spec R \mid \q \not \ni f_{\alpha}\}
                                \\
                                \iff & \exists \alpha \in A: \p \not \ni f_{\alpha}
                                \\
                                \iff & \bigvee_{\alpha \in A} \neg\left(\p \ni f_{\alpha}\right)
                                \\
                                \iff & \neg \bigwedge_{\alpha \in A} (\p \ni f_{\alpha}) 
                                \\
                                \iff & \neg \left(\p \supset \bigcup_{\alpha \in A} \{f_{\alpha}\}\right)
                                \\
                                \iff & \neg \left(\p \supset \{f_{\alpha}\}_{\alpha \in A}\right)
                                \\
                                \iff & \p \in \Spec R \setminus \left\{\q \in \Spec R \mid \q \supset \{f_{\alpha}\}_{\alpha \in A}\right\}
                                \\
                                \iff & \p \in \Spec R \setminus V\left(\{f_{\alpha}\}_{\alpha \in A}\right)
                            \end{aligned}
                        $$
                    Thus:
                        $$\bigcup_{\alpha \in A} D(f_{\alpha}) = \Spec R \setminus V\left(\{f_{\alpha}\}_{\alpha \in A}\right)$$
                    i.e. the complement of the Zariski-closed set $V\left(\{f_{\alpha}\}_{\alpha \in A}\right)$ inside $\Spec R$ is a union of Zariski-open sets $D(f_{\alpha})$, which we know from proposition \ref{prop: zariski_open_well_definiteness} to be Zariski-open itself. As stated, this implies that definitions \ref{def: zariski_closed} and \ref{def: zariski_open} give us the same Zariski topology on prime spectra of commutative rings.
                \end{proof}
            \begin{corollary} \label{coro: zariski_basis}
                Let $R$ be a commutative ring and let $f$ denote elements of $R$. Then, the distinguished Zariski-open sets $D(f)$ form a base of the Zariski topology on $\Spec R$.
            \end{corollary}
                \begin{proof}
                    This is a direct consequence of the fact that complements of Zariski-closed sets are unions of Zariski-open sets of the form $D(f)$.
                \end{proof}
            
            We have managed to show that on each ring spectrum, there is a canonical topology, namely the Zariski topology. A naturaly follow-up question is thus: can we upgrade $\Spec$ to a functor whose domain is $\Top$ instead of $\Sets$ ? Luckily, the answer is yes, although we will need to do some work to show that this is the case. 
            \begin{proposition}[Continuous functions between spectra] \label{prop: continuous_functions_between_spectra}
                By equipping prime spectra of commutative rings with the Zariski topology (in either the sense of definition \ref{def: zariski_closed} or \ref{def: zariski_open}), one naturally gets a functor:
                    $$\Spec: \Cring^{\op} \to \Top$$
                assigning to commutative rings their respective Zariski topological spaces.
            \end{proposition}
                \begin{proof}
                    It will suffice to show that given any ring homomorphism $f: A \to B$, the induced map $\Spec f: \Spec B \to \Spec A$ is continuous, which we can do by showing that preimages of Zariski-open subsets of $\Spec A$ under $\Spec f$ are closed in $\Spec B$; in fact, we can restrict our attention to \textit{basic} open subsets of $\Spec A$ (i.e. subsets of the form $D_A(a)$, for some $a \in A$), as they form a basis for the Zariski topology on $\Spec A$. Let $D_A(a)$ be such a basic open set. It preimage under $\Spec f$ is thus the following subset of $\Spec B$:
                        $$(\Spec f)^{-1}\left(D_A(a)\right) = \{\q \in \Spec B \mid (\Spec f)(\q) \in D_A(a)\}$$
                    Writing out the definition of $D_A(a)$ (cf. definition \ref{def: zariski_open}) then gives the following chain of logical equivalences:
                        $$
                            \begin{aligned}
                                & \q \in (\Spec f)^{-1}\left(D_A(a)\right)
                                \\
                                \iff & (\Spec f)(\q) \in D_A(a)
                                \\
                                \iff & f^{-1}(\q) \in D_A(a)
                                \\
                                \iff & \neg(f^{-1}(\q) \ni a)
                                \\
                                \iff & \neg(\q \ni f(a))
                                \\
                                \iff & \q \in D_B(f(a))
                            \end{aligned}
                        $$
                    which proves that:
                        $$(\Spec f)^{-1}\left(D_A(a)\right) = D_B(f(a))$$
                    and because $D_B(f(a))$ is open by definition, so is the preimage $(\Spec f)^{-1}(D_A(a))$. As stated at the beginning, this implies that $\Spec f$ is a continuous function, and thus there exists a functor:
                        $$\Spec: \Cring^{\op} \to \Top$$
                    assigning commutative rings and homomorphisms between them to ring spectra equipped with the Zariski topology and continuous maps in between.
                \end{proof}
                
            \begin{example}[Topologically interesting ring spectra]
                \noindent
                \begin{enumerate}
                    \item \textbf{(The complex affine plane and complex affine $n$-spaces):}
                        \begin{figure}[H]
                            \centering
                            \includegraphics[width=\linewidth,height=\textheight,keepaspectratio]{Figures/complex affine plane.png}
                            \caption{The complex affine plane $\A^2_{\bbC}$ (\cite{risingsea}, figure 3.3)}
                            \label{fig: complex_affine_plane}
                        \end{figure}
                    \item \textbf{(Revisiting $\Spec \Z$):}
                        \begin{figure}[H]
                            \centering
                            \includegraphics[width=\linewidth,height=\textheight,keepaspectratio]{Figures/Spec Z.png}
                            \caption{$\Spec \Z$ (\cite{risingsea}, figure 3.2)}
                            \label{fig: Spec_Z}
                        \end{figure}
                    \item \textbf{(A conic):}
                        \begin{figure}[H]
                            \centering
                            \includegraphics[width=\linewidth,height=\textheight,keepaspectratio]{Figures/conic.png}
                            \caption{A conic which is Zariski-closed inside $\A^3_{\bbC}$ (\cite{risingsea}, figure 3.4)}
                            \label{fig: conic}
                        \end{figure}
                \end{enumerate}
            \end{example}
            
            \begin{remark}[Comparing $\Spec$ and $\Spm$]
                Historically (and only because mathematicians were more interested in complex algebraic geometry back in the days), it was not the set of prime ideals of a commutative ring that was considered, but rather, the set of \textit{maximal} ideals. This was not out of \textit{na\"ivet\'e}, though. Maximal ideals enjoy being closed points in prime spectra of commutative rings (one can prove this by first looking at varieties $V(\m)$ associated to maximal ideals $\m$ of some commutative ring $R$, and then applying the definition of the (underlying set of) these varieties as spaces whose points are prime ideals containing $\m$, and then lastly, applying the usual definition of topological closures; as a corollary, one gets that prime ideals that are not maximal get sent by $\Spec$ to non-closed points in $\Spec R$), and so doing geometry with them is a lot more intuitive (albeit more restrictive as well) then doing so with all prime ideals. For instance, the underlying set of $\Spm \bbC[x]$ is precisely $\bbC$, whereas that of $\Spec \bbC[x]$ can be thought of as $\bbC \cup \{\infty\}$, i.e. as the Riemann sphere; in particular, the zero ideal $(0)$ corresponds to the point \say{at infinity}, which we denote by $\infty$. 
            \end{remark}
            
            \begin{example}[Non-isomorphic rings with homeomorphic spectra] \label{example: nonisomorphic_rings_with_the_same_spectra}
                The following examples are of non-isomorphic rings with homeomorphic prime spectra; through them, we are able to show that the functor $\Spec: \Cring^{\op} \to \Top$ is not an equivalence of categories (nor even a fully faithful inclusion). 
                \begin{enumerate}
                    \item \textbf{(Fields):} The prime spectra of any field is just the one-point space, but clearly, not all fields are isomorphic.
                    \item \textbf{(Discrete valuation rings):} The spectrum of any \href{https://en.wikipedia.org/wiki/Discrete_valuation_ring}{\underline{discrete valuation ring}} is homeomorphic to the \href{https://ncatlab.org/nlab/show/Sierpinski+space}{\underline{Sierpi\'nski space}} (to see why this is the case, firstly check that discrete valuation rings only have two prime ideals, one being the zero ideal and one being the unique maximal ideal, and that the latter is a closed point in the spectrum whereas the former is generic), but of course, not all discrete valuation rings are isomorphic to one another.
                \end{enumerate}
            \end{example}
        
        \subsubsection{Affine schemes}
            Next, we will be discussing the idea of so-call \textbf{structure sheaves}, but in order to make sense of these entities, we will need to know what $\C$-valued sheaves are for categories $\C$ more general than $\Sets$:
            \begin{definition}[$\C$-valued sheaves] \label{def: C_valued_sheaves}
                \noindent
                \begin{enumerate}
                    \item \textbf{($\C$-valued sheaves):} Let $(\S, J)$ be a site\footnote{... which is not necessarily small, as cases such as $\S \cong \Top$ and $\S \cong \Mfd^{\smooth}_{/\R}$ are interesting in their own rights.} and let $\C$ be a category with \textit{enough small limits} and \textit{enough filtered colimits} (the purpose of the second hypothesis is to ensure that stalks, should they exist, are well-defined); note that $\C$ need not be small. Additionally, fix an \textit{arbitrary} object $x$ of $\S$ along with a covering sieve $\calU_{/x} \in J$ thereon. Also, let $j: \S \to \Psh_{\C}(\S)$ be the Yoneda embedding. Then, a \textbf{$\C$-valued sheaf} on $(\S, J)$ is a functor $\calF: \S^{\op} \to \C$ such that $\calF(x) \cong \calF\left( \underset{u \in \calU_{/x}}{\colim} ju \right)$.
                    \item \textbf{($\C$-topoi):} $\C$-valued sheaves on a given site $(\S, J)$ form a category in the obvious manner. We shall be writing $\Sh_{\C}(\S, J)$ for this category, and such categories will be called \textbf{$\C$-topoi}, even though this is an abuse of terminology.
                \end{enumerate}
            \end{definition}
            \begin{example}[Sheaves of rings]
                The notion of sheaves of rings, which subsumes that of structure sheaves (cf. proposition \ref{prop: structure_sheaf}), follows suite from definition \ref{def: C_valued_sheaves}. Note that such constructions are well-defined, as the category of rings is both complete and cocomplete.
            \end{example}
            
            Having defined sheaves that might take values categories other than $\Sets$, let us now try to define affine schemes as locally ringed spaces whose underlying topological spaces are spectra of commutative rings, and whose structure presheaves have a certain condition imposed upon them, which happens to guarantee that:
                \begin{enumerate}
                    \item these structure presheaves are indeed sheaves (proposition \ref{prop: structure_sheaf}) with local stalks (corollary \ref{coro: structure_sheaf_properties}), and
                    \item they are unique (proposition \ref{prop: structure_sheaf_uniqueness}), which is an important feature, because ringed spaces are uniquely defined by their structure sheaves; this fact will also be used to establish the fully faithfulness of $\Spec$ as a functor from $\Cring^{\op}$ to the category $\Loc\Ringed\Spc$ of locally ringed spaces. 
                \end{enumerate}
            Our efforts will culminate in definition \ref{def: affine_schemes}.
                
            \begin{proposition}[Structure sheaves of affine schemes] \label{prop: structure_sheaf} \index{Structure sheaves}
                Let $k$ be a base commutative ring, and let $\calO_{\Spec R}$ be \textit{a} presheaf of commutative rings on $\Ouv(\Spec R)$ determined by the following rule on objects:
                    $$\calO_{\Spec R}(D_R(f)) \cong R_f$$
                for all element $f \in R$. Any presheaf on $\Ouv(\Spec R)$ that are defined this way is a Zariski sheaf (i.e. a sheaf on the site $\Ouv(\Spec R)$ of Zariski-open subsets of $\Spec R$), and is called \textit{a} \textbf{structure sheaf} on $\Spec R$.
            \end{proposition}
            \begin{corollary}[On the locality of stalks] \label{coro: structure_sheaf_properties}
                Let $R$ be a commutative ring and let $\p$ be an arbitrary prime ideal of $R$. Then one has the following characterisation of the stalk $\calO_{\Spec R, \p}$ at $\p$ of the structure sheaf $\calO_{\Spec R}$:
                    $$\calO_{\Spec R, \p} \cong R_{\p}$$
                This shows that affine schemes are, in fact, \textit{locally} ringed spaces and not just ringed spaces. 
            \end{corollary} 
                \begin{proof}
                    Recall that the stalk $\calF_x$ of a sheaf (of sets) $\calF$ on a topological space $(X, \Ouv(X))$ is given by the filtered colimit indexed by the poset of open neighbourhoods of the chosen point $x \in X$:
                        $$\calF_x \cong \underset{U \in \{V \in \Ouv(X) \mid V \ni x\}}{\colim} \calF(U)$$
                    By adapting this definition to the underlying Zariski-topological spaces of affine schemes, we get that:
                        $$\calO_{\Spec R, \p} \cong \underset{U \in \{V \in \Ouv(\Spec R) \mid V \ni \p\}}{\colim} \calO_{\Spec R}(U)$$
                    with $\Ouv(\Spec R)$ the Zariski topology defined via open sets as in definition \ref{def: zariski_open}. In corollary \ref{coro: zariski_basis}, we have already seen how the distinguished Zariski-open sets defined in definition \ref{def: zariski_open} form a basis for the Zariski topology on commutative ring spectra, and so the above filtered colimit can be rewritten as:
                        $$\calO_{\Spec R, \p} \cong \underset{D_R(f) \in \{V \in \Ouv(\Spec R) \mid V \ni \p\}}{\colim} \calO_{\Spec R}\left(D_R(f)\right)$$
                    and because $D_R(f) = \{\p \in \Spec R \mid \p \not \ni f\}$, one subsequently gets:
                        $$\calO_{\Spec R, \p} \cong \underset{f \in \{g \in R \mid g \not \in \p\}}{\colim} \calO_{\Spec R}\left(D_R(f)\}\right)$$
                    Lastly we have the following isomorphism:
                        $$\calO_{\Spec R, \p} \cong \underset{f \in \{g \in R \mid g \not \in \p\}}{\colim} \calO_{\Spec R}\left(D_R(f)\right) \cong \underset{f \in R \setminus \p}{\colim} R_f \cong R_{\p}$$
                    Thus $\calO_{\Spec R, \p} \cong R_{\p}$ as claimed.
                \end{proof}
                
            \begin{proposition}[Uniqueness of structure sheaves] \label{prop: structure_sheaf_uniqueness}
                Let $R$ be a commutative ring. Then, there is only one unique structure sheaf attached to $\Spec R$. 
            \end{proposition}
                \begin{proof}
                    Suppose to the contrary that there exist two \textit{distinct} Zariski sheaves of $R$-algebras on ${}^{R/}\Comm\Alg^{\op}$ $\calF$ and $\calG$ such that:
                        $$\forall f \in R: \calF(\Spec R_f) \cong \calG(\Spec R_f) \cong R_f$$
                    However, the localisation of any commutative at its multiplicative identity is just itself, and so:
                        $$\calF(\Spec R) \cong \calG(\Spec R) \cong R$$
                    for all commutative rings $R$. This means that the functors $\calF$ and $\calG$ are naturally isomorphic, i.e. they can not be distinct. Thus, the structure sheaf attached to a given ring spectrum is unique (up to natural isomorphisms, of course).
                \end{proof}
                
            \begin{example}[Spotting structure sheaves in the wild]
                Let $R$ be a discrete valuation ring that is a \href{https://en.wikipedia.org/wiki/Dedekind_domain}{\underline{Dedekind domain}} (so the only proper ideals of $R$ would be $(0)$ and its unique maximal ideal) with unique maximal ideal $\p$, and recall that its spectrum is (homeomorphic to) the Sierpi\'nski space (see example \ref{example: nonisomorphic_rings_with_the_same_spectra} for more details); in particular, the subset $\{(0)\}$ of $\Spec R = \{(0), \p\}$ is the only non-empty open proper subset. Now, suppose that $\calF$ is a Zariski sheaf on $\Spec R$ given by the following formula:
                    $$
                        \calF(U) \cong 
                        \begin{cases}
                            \text{$R$ if $U = \Spec R$}
                            \\
                            \text{$\Frac R$ if $U = \{(0)\}$}
                        \end{cases}
                    $$
                (note that discrete valuation rings are integral domains, so it makes sense to consider their fields of fractions). The point that is to be made here is that $\calF$ qualifies as a structure sheaf on $\Spec R$. To see why this is the case, note that because $R$ has only two prime ideals, namely $(0)$ and $\p$, 
            \end{example}
            
            \begin{example}[The complex affine line]
                Recall that in example \ref{example: spectra_sets}, we have seen how a point of the complex affine line $\A^1_{\bbC}$ is either the zero ideal, or of the form $(t - a)$ for any complex number $a$; one should keep the following picture in mind:
                    \begin{figure}[H]
				        \centering
				        \includegraphics[width=\linewidth,height=\textheight,keepaspectratio]{Figures/complex affine line.png}
				        \caption{The complex affine line $\A^1_{\bbC}$ (\cite{risingsea}, figure 3.1)}
				        \label{fig: complex_affine_line_stalks}
				    \end{figure}
			    \noindent
			    Now, as an affine scheme, $\A^1_{\bbC}$ comes equipped with a structure sheaf $\calO_{\A^1_{\bbC}}$, whose stalks, as shown in corollary \ref{coro: structure_sheaf_properties}, are precisely the localisations of $\bbC[t]$ at its prime ideals. There are thus two cases:
			        \begin{enumerate}
			            \item The stalk at $(0)$ is given by:
			                $$\calO_{\A^1_{\bbC}, (0)} \cong \bbC[t]_{(0)} \cong \bbC(t)$$
		                and thus the residue field is trivially $\bbC(t)$.
			            \item At non-zero primes, the stalks of the structure sheaf $\calO_{\A^1_{\bbC}}$ are given by the following localisations:
			                $$\calO_{\A^1_{\bbC}, (t - a)} \cong \bbC[t]_{(t - a)}$$
		                whose elements we note to be fractions of the form $\frac{f(t)}{g(t)}$ whose denominators do not vanish at $t = a$. Now, recall that the localisation of any commutative ring at a prime ideal is a local ring, and that inside \textit{any} local commutative ring, elements in the complement of the unique maximal ideal are units; in particular, these facts imply that the elements of the complement $\bbC[t]_{(t - a)} \setminus (t - a)\bbC[t]_{(t - a)}$ are all invertible. Consequently, these elements must be fractions $\frac{f(t)}{g(t)}$ whose numerators and denominators both do not vanish at $t = a$. Thus, a reasonable description of the canonical quotient map is the evaluation map:
		                    $$\frac{f(t)}{g(t)} \mapsto \frac{f(a)}{g(a)}$$
	                    whose image is precisely $\bbC$. Therefore, the stalks of $\calO_{\A^1_{\bbC}}$ are all isomorphic to $\bbC$.
			        \end{enumerate}
            \end{example}
        
            \begin{definition}[Affine schemes] \label{def: affine_schemes}
                An \textbf{affine scheme} is a locally ringed space that is isomorphic to one of the form $(|\Spec R|, \calO_{\Spec R})$ for some commutative ring $R$. Morphisms of affine schemes are morphisms of locally ringed spaces, and as such, one has a full subcategory $\Sch^{\aff}$ of affine schemes within the category of locally ringed spaces. 
            \end{definition}
            
            \begin{theorem}[Isbell Duality for locally ringed spaces] \label{theorem: isbell_duality_for_locally_ringed_spaces}
                There is an adjunction as follows:
                    $$
                        \begin{tikzcd}
                        	{\Cring^{\op}} & \Loc\Ringed\Spc
                        	\arrow[""{name=0, anchor=center, inner sep=0}, "\Spec"', bend right, from=1-1, to=1-2]
                        	\arrow[""{name=1, anchor=center, inner sep=0}, "\Gamma"', bend right, from=1-2, to=1-1]
                        	\arrow["\dashv"{anchor=center, rotate=-90}, draw=none, from=1, to=0]
                        \end{tikzcd}
                    $$
            \end{theorem}
            \begin{corollary}
                The adjunction from theorem \ref{theorem: isbell_duality_for_locally_ringed_spaces} restricts down to an adjoint equivalence $\Sch^{\aff} \cong \Cring^{\op}$.
            \end{corollary}

    \subsection{The category of schemes}
        \subsubsection{Schemes}
            \begin{definition}[Schemes] \label{def: schemes}
                A \textbf{scheme} is a locally ringed space $(|X|, \calO_X)$ such that every point $x \in |X|$ as a Zariski-open neighbourhood $U_x \ni x$ that is isomorphic to an affine scheme. Morphisms of schemes are nothing but morphisms of locally ringed spaces, meaning that schemes form a category $\Sch$ which embeds fully faithfully into the category of locally ringed spaces.
            \end{definition}
            
            \begin{proposition}[Open subschemes are open locally ringed subspaces] \label{prop: open_subschemes_are_open_locally_ringed_subspaces}
                Let $X$ be a scheme and let $U \subseteq X$ be an open locally ringed subspace. Then, $U$ will be an open subscheme of $X$. 
            \end{proposition}
                \begin{proof}
                    
                \end{proof}
            \begin{corollary}[Zariski-bases of schemes] \label{coro: zariski_bases_of_schemes}
                
            \end{corollary}
    
        \subsubsection{Properties of schemes and their morphisms}
        
        \subsubsection{Topologies on schemes; descent-theoretic results}
    
    \subsection{Varieties}

    \subsection{Cohomology of schemes and derived categories of coherent sheaves}
        
        \section{Algebraic spaces}
    \subsection{The category of algebraic spaces}
        \subsubsection{Algebraic spaces and their formal properties}
            \begin{convention}[Big sites of schemes] \label{conv: big_sites_of_schemes}
                From this point onwards, we shall be writing $(\Sch_{/S})_{\fppf}$ for the \textit{big} fppf site of a given scheme $S$. In the presence of small sites, which we denote by $(\Sch_{/S})_{\fppf}^{\petit}$, we might write $(\Sch_{/S})_{\fppf}^{\gros}$ for the sake of distinction. Likewise for the \'etale topology.
            \end{convention}
            \begin{convention}[Diagonals] \label{conv: algebraic_spaces_diagonals}
                Let $S$ be a scheme and $F$ be a presheaf of sets on $(\Sch_{/S})_{\fppf}$. We shall be denoting its \textbf{diagonal} by $\Delta_{F/S}: F \to F \x_S F$. This is the morphism of sheaves which is as follows, for all $T \in \Ob((\Sch_{/S})_{\fppf})$:
                    $$\Delta_{F/S}(T): F(T) \to F(T) \x_{S(T)} F(T)$$
                    $$t \mapsto (t, t)$$
            \end{convention}
            
            \begin{lemma}[Permanence of properties of representable morphisms] \label{lemma: permanence_of_properties_of_representable_morphisms}
                Consider sheaves on a site $(\C, J)$, which may or may not be small.
                    \begin{enumerate}
                        \item Representability of morphisms of sheaves is preserved by compositions.
                        \item Representability of morphisms of sheaves is preserved by pullbacks. Also, the diagonal of a representable morphism of sheaves is also representable.
                        \item Let $\varphi: F \to G$ be a representable morphisms of presheaves on $\C$. If $G$ satisfies $J$-descent, then so does $F$.
                    \end{enumerate}
            \end{lemma}
                \begin{proof}
                    \noindent
                    \begin{enumerate}
                        \item 
                        \item 
                        \item 
                    \end{enumerate}
                \end{proof}
            \begin{definition}[Properties of representable morphisms] \label{def: properties_of_representable_morphisms_of_fppf_sheaves}
                Denote by $\calP$ an fppf-local\footnote{Cf. \cite[\href{https://stacks.math.columbia.edu/tag/02KO}{Tag 02KO}]{stacks}.} property of morphisms of schemes that is stable under base-changes\footnote{See \cite[\href{https://stacks.math.columbia.edu/tag/02WE}{Tag 02WE}]{stacks} for a list of such properties.}. Then, one says that a representable morphism of presheaves $\varphi: F \to G$ on $(\Sch_{/S})_{\fppf}$ (for some base scheme $S$) has property $\calP$ if and only if for all schemes $U \in \Ob(\Sch_{/S})$, the canonical projection $\pr_2: F \x_G U \to U$ has property $\calP$ (note that by representability, the pullback $F \x_G U$ is an object of $\Sch_{/S}$).
            \end{definition}
            \begin{proposition}[A representability criterion for diagonals] \label{prop: representability_criterion_for_diagonals}
                Let $S$ be a scheme and let $F$ be a presheaf of sets on $(\Sch_{/S})_{\fppf}$. In addition, suppose that $U, V \in \Ob(\Sch_{/S})$ are two arbitrary $S$-schemes equipped with morphisms $u: U \to F$ and $v: V \to F$, and that the pullback $U_{u, F, v} V$ is representable by some $S$-scheme $T \in \Ob(\Sch_{/S})$. If the canonical morphism $T \to U \x_S V$ has a property $\calP$ that is fppf-local and stable under base-changes, then the diagonal $\Delta_{F/S}: F \to F \x_S F$ must be representable and have property $\calP$.
            \end{proposition}
                \begin{proof}
                    
                \end{proof}
                
            \begin{definition}[Atlases] \label{def: atlases}
                Let $S$ be a scheme and let $F$ be a presheaf of sets on $(\Sch_{/S})_{\fppf}$. An fppf (respectively, \'etale) \textbf{atlas} of $F$ is thus an \'etale surjection $U \to F$ from some scheme $U \in \Ob(\Sch_{/S})$.
            \end{definition}
            \begin{definition}[Algebraic spaces] \label{def: algebraic_spaces}
                An \textbf{algebraic space} over a scheme $S$ is a sheaf of sets on $(\Sch_{/S})_{\fppf}$ with an \'etale atlas and representable diagonal.
            \end{definition}
            \begin{remark}[Algebraic spaces are algebraic stacks]
                Later on, we shall see that algebraic spaces are the same as so-called \textbf{$0$-algebraic stacks}. More on this later, once we have discussed algebraic stacks.
            \end{remark}
            \begin{proposition}[The category of algebraic spaces] \label{prop: the_category_of_algebraic_spaces}
                For any given scheme $S$, there is a category of algebraic spaces over $S$, denoted by $\Alg\Spc_{/S}$, which is a full subcategory of $\Sh(\Sch_{S, \fppf})$. 
            \end{proposition}
                \begin{proof}
                    
                \end{proof}
                
            \begin{proposition}[(Co)limits of algebraic spaces] \label{prop: (co)limits_of_algebraic_spaces}
                For any given scheme $S$, the category $\Alg\Spc_{/S}$ of algebraic spaces over $S$ has the following (co)limits:
                    \begin{enumerate}
                        \item \textbf{(Limits):} terminal objects and finite pullbacks (and hence finite products),
                        \item \textbf{(Colimits):} arbitrary small coproducts.
                    \end{enumerate}
            \end{proposition}
                \begin{proof}
                    \noindent
                    \begin{enumerate}
                        \item \textbf{(Limits):} 
                        \item \textbf{(Colimits):} 
                    \end{enumerate}
                \end{proof}
            \begin{corollary}[Schemes are algebraic spaces] \label{coro: schemes_are_algebraic_spaces}
                For any given scheme $S$, the category $\Sch_{/S}$ of $S$-schemes is a full subcategory of $\Alg\Spc_{/S}$ which is closed under finite pullbacks.
            \end{corollary}
            \begin{remark}[A sheaf-theoretic definition of schemes] \label{remark: sheaf_theoretic_definition_of_schemes}
                In fact, it is easy to see - using the fact that the fppf topology is subcanonical - that \textit{schemes are fppf sheaves $X$ on the category of affine schemes such that their diagonals $\Delta_X$ are representable and such that they admit so-called Zariski atlases, i.e. a jointly surjective family of open immersions $U_i \hookrightarrow X$ from affine schemes $U_i$}. Note that this definition is not circular since one can define schemes either as objects of $\Cring^{\op}$ or as certain kinds of locally ringed spaces, which is a definition that makes use of only the Zariski topology on prime spectra of commutative rings and the notion of structure sheaves.
            \end{remark}
            \begin{definition}[Clopen immersions of ringed spaces] \label{def: clopen_immersions_of_ringed_spaces}
                An immersion of ringed spaces is said to be \textbf{clopen} whenever it is simultaneously closed and open.
            \end{definition}
            \begin{remark}[Clopen immersions of fppf sheaves] \label{remark: clopen_immersions_of_fppf_sheaves}
                Given definition \ref{def: clopen_immersions_of_ringed_spaces}, we can define a \textbf{clopen immersion of sheaves} on $(\Sch_{/S})_{\fppf}$ (with $S$ being some scheme) as a morphism that is representable by clopen immersions of schemes. Note that this is a well-defined notion, as open immersions and closed immersions are stable under base change (since tensor products commute with localisations and quotients) and are fppf-local (cf. \cite[\href{https://stacks.math.columbia.edu/tag/01JY}{Tag 01JY}]{stacks}).
            \end{remark}
            \begin{lemma}[Representability by schemes and algebraic spaces of disjoint summands] \label{lemma: representability_by_schemes_and_algebraic_spaces_of_disjoint_summands}
                Throughout, we work over a base scheme $S$.
                \begin{enumerate}
                    \item Let $F, G$ be sheaves of sets on $(\Sch_{/S})_{\fppf}$. Then, the cannical morphism of sheaves $\iota_1: F \to F \sqcup G$ (or equivalently, $\iota_2: G \to F \sqcup G$) is a clopen immersion.
                    \item Let $X$ be a $S$-scheme (respectively, an algebraic space over $S$) and suppose that there exists a disjoint set $\{F_i\}_{i \in I}$ of sheaves on $(\Sch_{/S})_{\fppf}$ such that $X \cong \coprod_{i \in I} F_i$. Then, each of the disjoint summand $F_i$ will be a clopen subscheme of $X$ (respectively, an algebraic space over $S$ such that the canonical morphism of sheaves $\iota_i: F_i \to X$ is a clopen immersion).
                    \item Let $F$ be a sheaf on $(\Sch_{/S})_{\fppf}$ such that there exists a set $\{F_i\}_{i \in I}$ of open\footnote{In the sense that the inclusions $F_i \hookrightarrow F$ are open immersions. Note that open immersions are fppf-local by virtue of being \'etale, and the reader is invited to check that indeed, open immersions are stable under base change (this is simple consequence of the fact that tensor products and localisations commute).} algebraic subspaces of $F$ such that $\coprod_{i \in I} F_i$ is an algebraic space over $S$ and that the canonically induced morphism of sheaves $\coprod_{i \in I} F_i \to F$ is surjective. In such a situation, $F$ will also be an algebraic space over $S$.
                \end{enumerate}
            \end{lemma}
                \begin{proof}
                    \noindent
                    \begin{enumerate}
                        \item 
                        \item 
                        \item 
                    \end{enumerate}
                \end{proof}
                
            \begin{definition}[\'Etale equivalence relations] \label{def: etale_equivalence_relations}
                Let $S$ be a scheme. An \textbf{\'etale equivalence relation} internal to the category $\Sch_{/S}$ of $S$-schemes (respectively, the category $\Alg\Spc_{/S}$ of algebraic spaces over $S$)\footnote{Note that both categories have all finite pullbacks.} is thus an internal equivalence relation (in the sense of definition \ref{def: equivalence_relations}) whose source and target morphisms are both \'etale.
            \end{definition}
            \begin{lemma}[Base-changing \'etale and flat equivalence relations in schemes] \label{lemma: base_changing_etale_and_flat_equivalence_relations_in_schemes}
                Let $S$ be a scheme, let $U$ be an $S$-scheme, and let $s, t: R \toto U$ be an \'etale equivalence relation on $U$ over $S$.
                    \begin{enumerate}
                        \item \textbf{(\'Etale base-changes):} Suppose that $f: U' \to U$ is an \'etale morphism of $S$-schemes and that $s', t': R' \toto U'$ is the pullback of $s, t: R \toto U$ along $f$. Then, $s', t': R' \toto U'$ will be an \'etale equivalence relation\footnote{Recall that pullbacks of internal equivalence relations are also internal equivalence relations, so $s', t': R' \toto U'$ is \textit{a priori} an equivalence relation on $U'$ over $S$ (cf. \cite[\href{https://stacks.math.columbia.edu/tag/02V8}{Tag 02V8}]{stacks}). Out task here is simply to establish geometric properties of the pullback $s', t': R' \toto U'$, such as whether or not it is \'etale.} on $U'$ over $S$.
                        \item \textbf{(Flat base-changes):} Suppose that the source and target morphisms $s, t: R \toto U$ are surjective, flat, and locally of finite presentation and that $f: U' \to U$ is a morphism of $S$-schemes that is flat and locally of finite presentation; in addition, denote by $s', t': R' \toto U'$ the pullback\footnote{We leave the verification of the fact that being surjective, flat, and locally of finite presentation are properties of morphisms of ($S$-)schemes which are stable under base-changes and are fppf-local up to our readers. } along $f$ of $s, t: R \toto U$. Then, the canonically induced morphism of sheaves $U'R/' \to U/R$ will be an open immersion. 
                    \end{enumerate}
            \end{lemma}
                \begin{proof}
                    \noindent
                    \begin{enumerate}
                        \item \textbf{(\'Etale base-changes):} Consider the following pullback diagrams in $\Sch_{/S}$:
                            $$
                                \begin{tikzcd}
                                	{R'} & {U'} \\
                                	R & U
                                	\arrow["{t'}"', shift right=2, from=1-1, to=1-2]
                                	\arrow["t"', shift right=2, from=2-1, to=2-2]
                                	\arrow["{f_s}"', shift right=2, from=1-1, to=2-1]
                                	\arrow["f", from=1-2, to=2-2]
                                	\arrow["\lrcorner"{anchor=center, pos=0.125}, draw=none, from=1-1, to=2-2]
                                	\arrow["s", shift left=2, from=2-1, to=2-2]
                                	\arrow["{s'}", shift left=2, from=1-1, to=1-2]
                                	\arrow["{f_t}", shift left=2, from=1-1, to=2-1]
                                \end{tikzcd}
                            $$
                        \'Etale morphisms are stable under base-changes so the canonical projections $f_s, f_t: R' \to\to R$ must be \'etale as a consequence of $f: U' \to U$ being \'etale by assumption. Compositions of \'etale morphisms are also \'etale themselves, and since $s, t: R \toto U$ are \'etale morphisms by assumption, the compositions $s \circ f_s$ and $t \circ f_t$ must also be \'etale. As such, the equivalence relation $s', t': R' \toto U'$ is \'etale. 
                        \item \textbf{(Flat base-changes):} By arguing as above and using the fact that surjectivity, flatness, and being locally of finite presentation are properties of morphisms of schemes which are stable under base-changes
                    \end{enumerate}
                \end{proof}
            \begin{proposition}[Base-chaning \'etale and flat equivalence relations in algebraic spaces] \label{prop: base_changing_etale_and_flat_equivalence_relations_in_algebraic_spaces}
                Let $S$ be a scheme, let $\calX$ be an algebraic space over $S$, and let $s, t: R \toto \calX$ be an \'etale equivalence relation on $\calX$ over $S$.
                    \begin{enumerate}
                        \item \textbf{(\'Etale base-changes):} Suppose that $f: \calX' \to \calX$ is an \'etale morphism of algebraic spaces over $\calX$ and that $s', t': R' \toto \calX'$ is the pullback of $s, t: R \toto \calX$ along $f$. Then, $s', t': R' \toto \calX'$ will be an \'etale equivalence relation on $\calX'$ over $S$.
                        \item \textbf{(Flat base-changes):} Suppose that the source and target morphisms $s, t: R \toto \calX$ are surjective, flat, and locally of finite presentation and that $f: \calX' \to \calX$ is a morphism of algebraic spaces over $S$ that is flat and locally of finite presentation; in addition, denote by $s', t': R' \toto \calX'$ the pullback along $f$ of $s, t: R \toto \calX$. Then, the canonically induced morphism of sheaves $\calX'/R' \to \calX/R$ will be an open immersion. 
                    \end{enumerate}
            \end{proposition}
            \begin{definition}[\'Etale presentations of algebraic spaces] \label{def: etale_presentations_of_algebraic_spaces}
                An \textbf{\'etale presentation} of an algebraic space $\calX$ over some scheme $S$, should it exist, is the quotient of an $S$-scheme $U$ by an \'etale equivalence relation $R$ on $U$ such that we have an isomorphism $\calX \cong U/R$ of sheaves on $(\Sch_{/S})_{\fppf}$. Phrased differently, an \'etale presentation of an algebraic space $\calX$ over some scheme $S$ is a coequaliser in $\Sh((\Sch_{/S})_{\fppf})$ of the following form, wherein $U$ and $R$ are as above:
                    $$
                        \begin{tikzcd}
                        	R & U & \calX
                        	\arrow[shift right=2, from=1-1, to=1-2]
                        	\arrow[shift left=2, from=1-1, to=1-2]
                        	\arrow["\coeq", two heads, from=1-2, to=1-3]
                        \end{tikzcd}
                    $$
            \end{definition}
            \begin{proposition}[Existence of \'etale presentation of algebraic spaces] \label{prop: existence_of_etale_presentations_of_algebraic_spaces}
                Let $S$ be a scheme, let $\calX$ be an algebraic space over $S$, and suppose that $\calX$ admits an \'etale atlas $u: U \to \calX$ by an $S$-scheme $U$ (cf. definition \ref{def: algebraic_spaces}). Then, we have the following \'etale presentation for $\calX$, wherein the two arrows $U \x_{u, \calX, u} U \toto U$ come from the canonical morphism $U \x_{u, \calX, u} U \to U \x_S U$:
                    $$
                        \begin{tikzcd}
                        	{U \x_{u, \calX, u} U} & U & \calX
                        	\arrow[shift right=2, from=1-1, to=1-2]
                        	\arrow[shift left=2, from=1-1, to=1-2]
                        	\arrow["{\coeq(u, u)}", two heads, from=1-2, to=1-3]
                        \end{tikzcd}
                    $$ 
            \end{proposition}
                \begin{proof}
                    
                \end{proof}
            \begin{proposition}[Quotients of schemes by \'etale equivalence relations are algebraic spaces] \label{prop: quotients_of_schemes_by_etale_equivalence_relations_are_algebraic_spaces}
                If $U$ is an $S$-scheme and $R$ is an \'etale equivalence relation on $U$, then not only will the sheaf $U/R$ be an algebraic space, but furthermore, the quotient morphism $U \to U/R$ will be \'etale\footnote{Because the quotient morphism $U \to U/R$ is \'etale, the \'etale presentation $R \toto U$ determines an \'etale atlas $U \to U/R$ of the algebraic space $U/R$.} surjection of sheaves on $(\Sch_{/S})_{\fppf}$. 
            \end{proposition}
                \begin{proof}
                    
                \end{proof}
                
            \begin{proposition}[Pushouts of algebraic spaces] \label{prop: puhsouts_of_algebraic_spaces}
                
            \end{proposition}
    
        \subsubsection{Properties of algebraic spaces and their morphisms}
        
        \subsubsection{Topologies on algebraic spaces; descent-theoretic results}
        
        \subsubsection{Criteria for sheaves being algebraic spaces}
    
    \subsection{Algebraic spaces over fields}
        
        \section{Algebraic stacks}
    \subsection{The \texorpdfstring{$2$}{}-category of algebraic stacks}
        \subsubsection{Algebraic stacks and their formal properties}
    
        \subsubsection{Properties of algebraic stacks and their morphisms}
        
        \subsubsection{Topologies on algebraic stacks; descent-theoretic results}
    
    \subsection{Algebraic stacks over fields}

    \subsection{Cohomology of algebraic stacks and derived categories of coherent sheaves}
        
        \begin{appendices}
            \chapter{Fibred categories, descent, and stacks}
                \begin{abstract}
            
                \end{abstract}
                
                \minitoc
                
                \section{Fibred categories}
    \subsection{Generalities on \texorpdfstring{$2$}{}-categories}
        \begin{definition}[$2$-categories] \label{def: 2_categories}
            
        \end{definition}
        
        \begin{definition}[$2$-commutative diagrams] \label{def: 2_commutative_diagrams}
            Let $\calK$ be a $2$-category. A $2$-commutative diagram therein is thus a square as follows, wherein there exists a $2$-isomorphism $\eta: p'f \Rightarrow pg$:
                $$
                    \begin{tikzcd}
                    	{y'} & y \\
                    	{x'} & x
                    	\arrow["{p'}"', from=1-1, to=2-1]
                    	\arrow["f", from=2-1, to=2-2]
                    	\arrow["g", from=1-1, to=1-2]
                    	\arrow["p", from=1-2, to=2-2]
                    	\arrow["\exists \eta", shorten <=8pt, shorten >=8pt, Rightarrow, from=2-1, to=1-2]
                    \end{tikzcd}
                $$
        \end{definition}
    
    \subsection{(Co)fibred categories}
        \subsubsection{Slice \texorpdfstring{$2$}{}-categories}
            \begin{definition}[Slice $2$-categories] \label{def: slice_2_categories}
                Let $\calK$ be a $2$-category and let $x \in \Ob(\calK)$ be an object therein. Then, we define the \textbf{slice} $2$-category $\calK_{/x}$ to be the $2$-category wherein:
                    \begin{itemize}
                        \item the objects are $1$-morphisms $(f: y \to x) \in 1\-\Mor(\calK)$,
                        \item the $1$-morphisms $\phi: (y', f') \to (y, f)$ are $1$-commutative triangles of $1$-morphisms $(f': y' \to x), (f: y \to x) \in 1\-\Mor(\calK)$ in $\calK$ of the following form:
                            $$
                                \begin{tikzcd}
                                	{y'} && y \\
                                	& x
                                	\arrow["\phi", from=1-1, to=1-3]
                                	\arrow["{f'}"', from=1-1, to=2-2]
                                	\arrow["f", from=1-3, to=2-2]
                                \end{tikzcd}
                            $$
                        \item and the $2$-morphisms between $1$-morphisms $\phi, \psi: (y', f') \to (y, f)$ are $2$-morphisms $(\eta: \psi \Rightarrow \phi) \in 2\-\Mor(\calK)$ such that the following diagram is $2$-commutative:
                            $$
                                \begin{tikzcd}
                                	{y'} && y \\
                                	& x
                                	\arrow[""{name=0, anchor=center, inner sep=0}, "\phi"', bend right, from=1-1, to=1-3]
                                	\arrow["{f'}"', from=1-1, to=2-2]
                                	\arrow["f", from=1-3, to=2-2]
                                	\arrow[""{name=1, anchor=center, inner sep=0}, "\psi", bend left, from=1-1, to=1-3]
                                	\arrow[shorten <=2pt, shorten >=2pt, Rightarrow, from=1, to=0, "\eta"']
                                \end{tikzcd}
                            $$
                    \end{itemize}
            \end{definition}
            \begin{example}[Over-categories] \label{example: over_categories}
                Let $\C$ be a category and let $1\-\Cat_2$ be the $2$-category with $1$-categories, functors, and natural transformations as objects, $1$-morphisms, and $2$-morphisms respectively. Then, there is a natural slice $2$-category $(1\-\Cat_2)_{/\C}$ wherein the objects are functors $p: \S \to \C$, $1$-morphisms are the evident $1$-commutative triangles of functors, and $2$-morphisms between $1$-morphisms $F, G: (\S', p') \to (\S, p)$ are natural transformations $\eta \in \Nat(F, G)$ such that $p(\eta_y) = \id_{p'(y)}$ for all $y \in \Ob(\S')$, i.e. such that the following square is strictly $2$-commutative:
                    $$
                        \begin{tikzcd}
                        	{p'(y)} & {p(F(y))} \\
                        	{p'(y)} & {p(G(y))}
                        	\arrow["{p(\eta_y)}", from=1-2, to=2-2]
                        	\arrow[from=1-1, to=1-2]
                        	\arrow["{\id_{p'(y)}}"', from=1-1, to=2-1]
                        	\arrow[from=2-1, to=2-2]
                        	\arrow[shorten <=8pt, shorten >=8pt, Rightarrow, from=2-1, to=1-2]
                        \end{tikzcd}
                    $$
                In this sense, $(1\-\Cat_2)_{/\C}$ is not just a $2$-category, but a strict one.
            \end{example}
            \begin{proposition}[$2$-pullbacks in slice $2$-categories] \label{prop: 2_pullbacks_in_slice_2_categories}
                
            \end{proposition}
                \begin{proof}
                            
                \end{proof}
            
            \begin{definition}[Prefibrations] \label{def: prefibrations}
                Consider a $1$-functor $p: \S \to \C$ between $1$-categories $\S$ and $\C$ is a \textbf{pre-fibration} if and only if it is surjective on the level of both objects and ($1$-)morphisms. 
            \end{definition}
            \begin{remark}[Fibres of prefibrations] \label{remark: fibres_of_prefibrations}
                Consider a prefibration $p: \S \to \C$, viewed as a $1$-morphism of $1\-\Cat_2$, and recall that $1\-\Cat_2$ admits $1$-terminal objects, namely categories that are equivalent to the singleton category $\pt$, as well as $1$-pullbacks. By viewing objects $U \in \Ob(\C)$ as functors $U: \pt \to \C$, one can define \textbf{fibres} of the prefibration $p: \S \to \C$ as $1$-pullbacks of the following kind:
                    $$
                        \begin{tikzcd}
                        	{\S_U} & \S \\
                        	\pt & \C
                        	\arrow["p", from=1-2, to=2-2]
                        	\arrow["U", from=2-1, to=2-2]
                        	\arrow["{p_U}"', from=1-1, to=2-1]
                        	\arrow[from=1-1, to=1-2]
                        	\arrow["\lrcorner"{anchor=center, pos=0.125}, draw=none, from=1-1, to=2-2]
                        \end{tikzcd}
                    $$
                It is then easy to see that $\S_U$ is the category wherein the objects are objects $x \in \S$ such that $p(x) = U$ and the morphisms are morphisms $(\phi: y \to x) \in \Mor(\S)$ such that $p(\phi) = \id_U$, and as such one ought to view $p_U: \S_U \to \pt$ as the unique functor from $\S_U$ to the singleton category whose only object is $U$ and whose only morphism is $\id_U$. Therefore, one can instead define prefibrations as functors $p: \S \to \C$ with non-empty fibres in the sense above and identify them by the families of $1$-pullbacks $\{\S \x_{p, \C, U} \pt\}_{U \in \Ob(\C)}$.
            \end{remark}
            \begin{remark}[The strict $2$-category of pre-fibrations]
                By definition, prefibrations $p: \S \to \C$ are objects of the strict $2$-category $(1\-\Cat_2)_{/\C}$, and so we might be tempted to declare that there is a $2$-full subcategory $\Pre\Fib(\C) \overset{2}{\subset} (1\-\Cat_2)_{/\C}$ spanned by prefibrations over $\C$, and indeed there is. To check that this is the case, one can verify that given any $1$-morphism $(F: (\S', p') \to (\S, p)) \in 1\-\Mor((1\-\Cat_2)_{/\C})$ and any object $U \in \C$, there exists a functor $F_U: \S'_U \to \S_U$ making the following diagram $1$-commutative in $1\-\Cat_2$:
                    $$
                        \begin{tikzcd}
                        	{\S'_U} &&&& {\S_U} \\
                        	{\S'} &&&& \S \\
                        	&& \pt \\
                        	&& \C
                        	\arrow["{p'}"', from=2-1, to=4-3]
                        	\arrow["U", from=3-3, to=4-3]
                        	\arrow["{p'_U}"', from=1-1, to=3-3]
                        	\arrow["\lrcorner"{anchor=center, pos=0.125}, draw=none, from=1-1, to=4-3]
                        	\arrow[from=1-1, to=2-1]
                        	\arrow["{p_U}", from=1-5, to=3-3]
                        	\arrow[from=1-5, to=2-5]
                        	\arrow["p", from=2-5, to=4-3]
                        	\arrow["F", from=2-1, to=2-5]
                        	\arrow["{F_U}", from=1-1, to=1-5, dashed]
                        	\arrow["\lrcorner"{anchor=center, pos=0.125, rotate=-90}, draw=none, from=1-5, to=4-3]
                        \end{tikzcd}
                    $$
                and similarly for $2$-morphisms in $\Pre\Fib(\C)$. We leave this as an exercise for the reader.
            \end{remark}
        
        \subsubsection{Fibred categories}
                
                \section{Descent theory}
    \subsection{Sieves and sites}
    
    \subsection{Descent data, sheaves, and stacks}
        \end{appendices}
        
    \chapter{Deformation theory}
        \begin{abstract}
            
        \end{abstract}
        
        \minitoc
    
        \section{Classical deformation theory}
    \subsection{Formal deformation theory}
        \subsubsection{A high-level overview}
            Deformation theory is a large topic, so let us preface our treatment of it with an overview of its features, along with the notable results in the subject. Our main references are \cite[\href{https://stacks.math.columbia.edu/tag/06G7}{Tag 06G7}]{stacks} and \cite[\href{https://stacks.math.columbia.edu/tag/08KW}{Tag 08KW}]{stacks}.
            
            For now, assume that $(\Lambda, \m, k)$ is the data of a Noetherian local ring with maximal ideal $\m$ and residue field $k$ and let $\C_{\Lambda, k}$ be the category of local Artinian $\Lambda$-algebras (always assumed to be commutative and unital)\footnote{The $\C$ stands for \say{deformation \textbf{c}ontext} whose residue fields are isomorphic to $k$, referred to in \cite[\href{https://stacks.math.columbia.edu/tag/06G7}{Tag 06G7}]{stacks} simply as the \say{base category}.}. At the same time, consider a left-exact functor (or should the reader prefer, a so-called \say{prestack} on $\C_{\Lambda, k}^{\op}$):
                $$F: \C_{\Lambda, k} \to \Grpd$$
            which we shall refer as a \textbf{pre-deformation functor}. Now, let $\hat{\C}_{\Lambda, k}$ be the category of strict pro-objects associated to $\C_{\Lambda, k}$, which is the full subcategory of the pro-completino $\Pro(\C_{\Lambda, k})$ spanned by cofiltered diagrams whose vertices are objects of $\C_{\Lambda, k}$ and whose (directed) edges are epimorphisms. In deformation theory, one seeks criteria under which pre-deformation functors are \textbf{strictly pro-representable}, which is to say that they are represented by objects of $\hat{\C}_{\Lambda, k}$ and thus can be thought of as objects belonging to formal geometry. As a consequence, many foundationally important results such as Grothendieck's Criteria for Pro-representability or Schlessinger's Criterion are of this nature.
    
        \subsubsection{Local Artinian algebras and deformation contexts}
            We begin by investigating the category $\C_{\Lambda, k}$ of local Artinian $\Lambda$-algebras (for $\Lambda$ a Noetherian local ring with residue field $k$) whose residue fields are isomorphic to $k$. This is a category that is inherently and concretely geometric in nature and as such shall serve as a template according to which we shall be able to axiomatically define more general so-called \say{\textbf{deformation contexts}}.
            \begin{definition}[Artinian rings] \label{def: artinian_rings} \index{Artinian rings}
                A commutative ring $\Lambda$ is said to be \textbf{Artinian} if and only if there exist no non-terminating descending chain of ideals (up to bijections, of course). Alternatively, since ideals of commutative rings corespond to Zariski-closed sets, one can define Artinian rings $\Lambda$ as those such that the underlying topological spaces of their corresponding affine schemes $\Spec \Lambda$ have no non-terminating \textit{ascending} chains of closed subsets. 
                
                It is not hard to see that for every given base commutative ring $\Lambda$ there is a category whose objects are local Artinian $\Lambda$-algebras and whose morphisms are local homomorphisms between them. We shall denote this category by $\C_{\Lambda}$. Furthermore, for each $R$-algebra $k$ that is a field, there is a corresponding full subcategory of $\C_{\Lambda}$, which we shall denote by $\C_{\Lambda, k}$ spanned by local Artinian $\Lambda$-algebras whose residue field is $k$. 
            \end{definition}
            
            \begin{lemma}[Basic properties of local Artinian rings] \label{lemma: artinian_rings_properties}
                \noindent
                \begin{enumerate}
                    \item Quotients and localisations of Artinian rings are also Artinian.
                    \item \cite[\href{https://stacks.math.columbia.edu/tag/00J6}{Tag 00J6}]{stacks} Finitely generated algebras over fields are Artinian.
                    \item A local Artinian ring $(\Lambda, \m)$ with residue field $\kappa$ is a finitely generated $\kappa$-algebra and admits a splitting $\Lambda \cong \kappa \oplus \m$.
                    \item \cite[\href{https://stacks.math.columbia.edu/tag/00J7}{Tag 00J7}]{stacks} Artinian rings only have finitely many maximal ideals.
                    \item \cite[\href{https://stacks.math.columbia.edu/tag/00J8}{Tag 00J8}]{stacks} Let $\Lambda$ be an Artinian ring. Then, its Jacobson radical is nilpotent. In fact, its Jacobson radical shall be the same as its nilradical.
                    \item \cite[\href{https://stacks.math.columbia.edu/tag/00JA}{Tag 00JA}]{stacks} Any commutative ring with finitely many maximal ideals and locally nilpotent Jacobson radical (such as Artinian rings) can be decomposed into the direct sum of its localisations at the maximal ideals. Furthermore, any prime ideal in such a ring is automatically maximal.
                    \item \cite[\href{https://stacks.math.columbia.edu/tag/00JB}{Tag 00JB}]{stacks} A commutative ring $A$ is simultaneously Artinian and Noetherian if and only if $A$ has finite length as a module over itself. 
                \end{enumerate}
            \end{lemma}
            
            \begin{definition}[Categories of local Artinian algebras] \label{def: categories_of_local_artinian_algebras}
                 Let $(\Lambda, \m, k)$ is the data of a Noetherian local ring with maximal ideal $\m$ and residue field $k$. We denote by $\C_{\Lambda, k}$ the category whose objects are local Artinian $\Lambda$-algebras whose residue fields are isomorphic to $k$ and whose morphisms are local $\Lambda$-algebra homomorphisms between them.
            \end{definition}
            \begin{proposition}[Finite completeness of categories of local Artinian algebras] \label{prop: finite_completeness_of_categories_of_local_artinian_algebras}
                Let $(\Lambda, \m, k)$ is the data of a Noetherian local ring with maximal ideal $\m$ and residue field $k$. Then, $\C_{\Lambda, k}$ is a finitely complete Artinian category.
            \end{proposition}
                \begin{proof}
                    That the category $\C_{\Lambda, k}$ is Artinian is obvious, and that it is finitely complete is a trivial consequence of lemma \ref{lemma: artinian_rings_properties}(3).
                \end{proof}
                
            \begin{definition}[Strict pro-objects] \label{def: strict_pro_objects}
                Let $\C$ be a small category and let $\Pro(\C)$ denote its pro-completion. Then, there exists a full subcategory $\hat{\C} \subset \Pro(\C)$ whose objects are cofiltered diagrams whose vertices are objects of $\C$ and whose (directed) edges are epimorphisms; objects of $\hat{\C}$ are referred to as \textbf{strict pro-objects} of $\C$. 
            \end{definition}
            \begin{example}[Completed Artinian local algebras] \label{example: completed_artinian_local_algebras}
                Let $(\Lambda, \m, k)$ is the data of a Noetherian local ring with maximal ideal $\m$ and residue field $k$. Then, the strict pro-completion $\hat{\C}_{\Lambda, k}$ shall be spanned by pro-objects of $\C_{\Lambda, k}$ which are of the form $\{\cdots \to A/\m_A^n \to \cdots \to A/\m_A^2 \to A/\m_A\}$ for some $(A, \m_A, k) \in \C_{\Lambda, k}$. 
                
                Because limits commute, and because $\C_{\Lambda, k}$ is finitely complete and Artinian (cf. proposition \ref{prop: finite_completeness_of_categories_of_local_artinian_algebras}), its strict pro-completion $\hat{\C}_{\Lambda, k}$ is must also be finitely complete and Artinian. It is also not hard to see that via taking limits of the cofiltered diagrams that are objects of $\hat{\C}_{\Lambda, k}$ is equivalent to the category of complete Noetherian local $\hat{\Lambda}$-algebras with residue field isomorphic to $k$.
            \end{example}
            \begin{theorem}[Grothendieck's pro-representability criterion] \label{theorem: grothendieck_pro_representability_criterion}
                \cite[Proposition 3.1]{grothendieck_fga_2} Let $\C$ be a finitely complete Artinian small category. Then, a functor $F: \C \to \Fin\Sets$ is strictly pro-representable if and only if it is left-exact.
            \end{theorem}
            
            \begin{proposition}[Pushouts and coproducts of completed Artinian local algebras] \label{prop: pushouts_and_coproducts_of_completed_artinian_local_algebras}
                Let $(\Lambda, \m, k)$ is the data of a Noetherian local ring with maximal ideal $\m$ and residue field $k$. Then, the category $\hat{\C}_{\Lambda, k}$ admits pushouts and initial objects\footnote{.. and as a result, all coproducts}.
            \end{proposition}
                \begin{proof}
                    
                \end{proof}
            \begin{corollary}
                Let $(\Lambda, \m, k)$ is the data of a Noetherian local ring with maximal ideal $\m$ and residue field $k$. Then, the category $\hat{\C}_{\Lambda, k}$ is cocomplete.
            \end{corollary}
                \begin{proof}
                    This is a direct consequence of the fact that a category is cocomplete if and only if it has all coproducts and coequalisers (cf. \cite[Theorem V.2.1]{maclane}).
                \end{proof}
                
            \begin{definition}[Deformation contexts] \label{def: deformation_context}
                A \textbf{deformation context} is a finitely complete category whose strict pro-completion (cf. definition \ref{def: strict_pro_objects}) is cocomplete. 
            \end{definition}
            \begin{definition}[Pre-deformation functors] \label{def: pre_deformation_functors}
                For $\C$ a deformation context, a \textbf{pre-deformation functor} (also called a \textbf{pre-deformation problem} or \textbf{pre-deformation prefibration}) on $\C$ is a category \textit{co}fibred in groupoids $p: \scrF \to \C$ (cf. definition \ref{def: categories_fibred_in_groupoids}) such that the associated prestack $F: (\C^{\op})^{\op} \to 1\-\Grpd_2$ takes finite ($1$-)limits in $\C \cong (\C^{\op})^{\op}$ to weak $2$-limits in the $2$-category $1\-\Grpd_2$ of wherein the objects are $1$-groupoids, $1$-morphisms are functors, and $2$-morphisms are natural transformations.
            \end{definition}
            \begin{example}[Completed local Artinian algebras and pre-deformation functors on them]
                Let $(\Lambda, \m, k)$ is the data of a Noetherian local ring with maximal ideal $\m$ and residue field $k$. Then, $\C_{\Lambda, k}$ will have a natural structure of a deformation context, as shown via propositions \ref{prop: finite_completeness_of_categories_of_local_artinian_algebras} and \ref{prop: pushouts_and_coproducts_of_completed_artinian_local_algebras}. As such, let us consider the category cofibred in groupoids $p: \scrF \to \C_{\Lambda, k}$ such that the associated prestack $F: \C_{\Lambda, k} \to 1\-\Grpd_2$ is such that $F(k) \cong \{*\}$.
            \end{example}
    
        \subsubsection{Obstructions}
            \begin{definition}
                One says that a category cofibred in groupoids $p: \scrF \to \C$ over a $1$-category $\C$ is \textbf{unobstructed} if and only if the associated prestack is formally smooth. 
            \end{definition}
        
    \subsection{Deformations of ringed topoi}
    
    \subsection{Examples of deformation problems}
        \subsubsection{Deformations of singularities}
        
        \subsubsection{Deformations of Galois representations}
        
        \input{Chapters/Deformation theory and representability/cotangent complexes}
        
        \section{Criteria for representability and Artin's Axioms}
        
    \chapter{Algebraisation and formal geometry}
        \begin{abstract}
            
        \end{abstract}
        
        \minitoc
        
        \input{Chapters/Algebraisation and formal geometry/formal algebraic spaces}
        
        \input{Chapters/Algebraisation and formal geometry/formal algebraic stacks}
	
	\addcontentsline{toc}{section}{References}
	\printbibliography

\end{document}