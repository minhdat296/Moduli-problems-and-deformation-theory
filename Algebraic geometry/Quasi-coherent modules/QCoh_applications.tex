\section{Some applications}
    \subsection{Coherent sheaves and perfect complexes; cohomology of projective spaces}
        \begin{definition}[Coherent modules] \label{def: coherent_modules}
            Let $X$ be a scheme with a fixed Zariski covering:
                $$\{ f_i: \Spec A_i \to X \}_{i \in I}$$
            The category $\Coh(X)$ of \textbf{coherent $\scrO_X$-modules} will then be the full subcategory of $\QCoh(X)$ consisting of finite-type objects, which is to say that $\scrM \in \Ob( \QCoh(X) )$ is coherent if and only if $\Gamma(\Spec A_i, f_i^*\scrM)$ is a finite $A_i$-module for every $i \in I$.
        \end{definition}
        Even though we just gave a definition of coherent sheaves on a general scheme, we will usually be considering coherent $\scrO_X$-modules over \textit{locally Noetherian} schemes $X$, for the same reason that finitely generated modules are usually considered over Noetherian rings. 
        \begin{proposition}[Coherent modules over (locally) Noetherian schemes] \label{prop: coherent_modules_over_noetherian_schemes}
            Let $X$ be a locally Noetherian scheme. Then, an $\scrO_X$-module $\scrM$ will be coherent if and only if it is finitely presented.
        \end{proposition}
        \begin{example}
            Let $X$ be a locally Noetherian scheme. Then the structure sheaf $\scrO_X$ itself is coherent. 
        \end{example}
        \begin{example}[Incoherent rings]
            If $X$ is a scheme such that $\scrO_X$ is not coherent as a module over itself, then naturally, we shall refer to $\scrO_X$ as an \textbf{incoherent sheaf of rings}. There is the following example of such a ring, due to Brian Conrad: \href{http://math.stanford.edu/~vakil/216blog/incoherent.pdf}{http://math.stanford.edu/~vakil/216blog/incoherent.pdf}.
        \end{example}

    \subsection{Zariski's Main Theorem}
        \begin{theorem}[The Theorem of Formal Functions] \label{theorem: formal_function_theorem}

        \end{theorem}
            \begin{proof}
                
            \end{proof}

        \begin{theorem}[Zariski's Main Theorem] \label{theorem: zariski_main_theorem}
            
        \end{theorem}
            \begin{proof}
                
            \end{proof}

        \begin{theorem}[Stein's Factorisation Theorem] \label{theorem: stein_factorisation}
            
        \end{theorem}

    \subsection{The archimedean GAGA principle of Serre}
        \begin{convention}
            In this subsection, we work exclusively over $\bbC$. 
        \end{convention}

        \begin{lemma}[Complex-analytic completions of finite-type $\bbC$-algebras] \label{lemma: complex_analytic_completions_of_finite_type_C_algebras}
            There is an \textbf{analytification functor}:
                $$(-)^{\an}: \bbC\-\Comm\Alg^{\ft} \to \bbC\-\Ban\Comm\Alg^{\locconvex}$$
            from the category of finite-type $\bbC$-algebras to that of locally convex Banach $\bbC$-algebras, sending objects $A$ of the former to objects $A^{\an}$ of the former, given as the complex-analytic completion of $A$. Furthermore, this functor is compatible with base-change in the sense that, if:
                $$\phi: A \to B$$
            is a homomorphism of finite-type $\bbC$-algebras then:
                $$B^{\an} \cong (A \tensor_{A, \phi} B)^{\an} \cong A^{\an} \hattensor_{A, \phi} B$$
        \end{lemma}
            \begin{proof}
                
            \end{proof}
        \begin{remark}[Associated complex-analytic topological spaces] \label{remark: associated_complex_analytic_topological_spaces}
            Let $X$ be a finite-type $\bbC$-scheme. Since $\bbC$ is algebraically closed, closed points of $X$ are in bijection with $X(\bbC)$ by Hilbert's \textit{Nullstellensatz}. At the same time, $X(\bbC)$ - by definition - consists of all solutions to the system of polynomials in $\bbC^n$ or $\P^{n, \an}_{\bbC}$ cut out by $X$ (say, $\dim X \leq n$), so $X(\bbC)$ naturally inherits the subspace topology from the complex-analytic space $\bbC^n$ or $\P^{n, \an}_{\bbC}$. In this sense, we have that:
                $$X^{\an} := X(\bbC)$$
            is the natural complex-analytic topological space associated to the underlying topological space of $X$.

            The identification of closed points of $X$ gives rise to a natural continuous embedding:
                $$i_X: X^{\an} \to X$$
        \end{remark}
        In order to obtain complex-analytic locally ringed spaces associated to locally finite-type $\bbC$-schemes, we need also to somehow complex-analytically complete the structure sheaves of these schemes. 
        \begin{proposition}[Complex-analytic completions of locally finite-type $\bbC$-schemes] \label{prop: complex_analytic_completions_of_locally_finite_type_C_schemes}
            There is an \textbf{analytification functor}:
                $$(-)^{\an}: \Sch_{/\Spec \bbC}^{\lft} \to \An\Spc$$
                $$(X, \scrO_X) \mapsto (X^{\an}, \scrO_{X^{\an}})$$
            from the category of locally finite-type $\bbC$-schemes to that of complex-analytic spaces, determined by:
                $$\scrO_X^{\an} := (i_X^*\scrO_X)^{\an}$$
            with notations as in remark \ref{remark: associated_complex_analytic_topological_spaces}. Furthermore, if:
                $$f: X \to Y$$
            is a morphism of locally finite-type $\bbC$-schemes then we will obtain a pullback square:
                $$
                    \begin{tikzcd}
                    {X^{\an}} & {Y^{\an}} \\
                    X & Y
                    \arrow["{f^{\an}}", from=1-1, to=1-2]
                    \arrow["f", from=2-1, to=2-2]
                    \arrow["{i_X}"', from=1-1, to=2-1]
                    \arrow["{i_Y}", from=1-2, to=2-2]
                    \arrow["\lrcorner"{anchor=center, pos=0.125}, draw=none, from=1-1, to=2-2]
                    \end{tikzcd}
                $$
        \end{proposition}
            \begin{proof}
                
            \end{proof}
        \begin{lemma}[(Faithful) flatness of complex-analytifications] \label{lemma: flatness_of_complex_analytifications}
            Let $A$ be an arbitrary finite-type $\bbC$-algebra. Then $A^{\an}$ will be flat over $A$. 
        \end{lemma}
            \begin{proof}
                
            \end{proof}
        \begin{corollary} \label{coro: flatness_of_complex_analytifications}
            For any locally finite-type $\bbC$-scheme $X$, the pullback functor:
                $$i_X^*: \scrO_X\mod \to \scrO_{X^{\an}}\mod$$
            is exact. 
        \end{corollary}
        
        \begin{remark}
            Consider a morphism:
                $$f: X \to Y$$
            between schemes locally of finite type over $\Spec \bbC$. This gives rise to a pullback square of locally ringed spaces:
                $$
                    \begin{tikzcd}
                    {X^{\an}} & {Y^{\an}} \\
                    X & Y
                    \arrow["{f^{\an}}", from=1-1, to=1-2]
                    \arrow["f", from=2-1, to=2-2]
                    \arrow["{i_X}"', from=1-1, to=2-1]
                    \arrow["{i_Y}", from=1-2, to=2-2]
                    \arrow["\lrcorner"{anchor=center, pos=0.125}, draw=none, from=1-1, to=2-2]
                    \end{tikzcd}
                $$
            as stated in proposition \ref{prop: complex_analytic_completions_of_locally_finite_type_C_schemes}, which in turn gives rise to the following cohomological base-change map/spectral sequence:
                $$Li_Y^* \circ R f_* \Rightarrow R f^{\an}_* \circ Li_X^*$$
            but since the functors $i_X^*, i_Y^*$ are exact \textit{a priori} (cf. corollary \ref{coro: flatness_of_complex_analytifications}), this reduces down to a cohomological comparison map as follows:
                $$i_Y^* \circ R f_* \Rightarrow R f^{\an}_* \circ i_X^*$$
            The existence of such a natural transformation allows us to formulate a version of proper base-change in this context, which yields us cohomological comparison isomorphisms that interpolate between the Zariski and complex-analytic topologies on locally finite-type $\bbC$-schemes and their associated analytic spaces respectively (see corollary \ref{coro: GAGA_cohomological_comparison}). 
        \end{remark}
        \begin{theorem}[Relative analytification of sheaves of modules] \label{theorem: relative_analytification_of_sheaves_of_modules}
            Suppose that:
                $$f: X \to Y$$
            is a morphism between schemes locally of finite type over $\Spec \bbC$. If $f$ is proper, then the canonical natural transformation:
                $$i_Y^* \circ R f_* \Rightarrow R f^{\an}_* \circ i_X^*$$
            will be a natural isomorphism of t-exact functors $D^+(\scrO_X\mod) \to D^+(\scrO_{Y^{\an}}\mod)$.
        \end{theorem}
            \begin{proof}
                This is a result of proper base-change for general ringed spaces (see \cite[\href{https://stacks.math.columbia.edu/tag/09V4}{Tag 09V4}]{stacks} for now). 
            \end{proof}
        \begin{corollary}[GAGA cohomological comparison] \label{coro: GAGA_cohomological_comparison}
            When $Y \cong \Spec \bbC$ (and hence $i_Y^*$ is just the identity functor), there are isomorphisms of $\bbC$-vector spaces:
                $$H^{\bullet}(X, \scrM) \cong H^{\bullet}(X^{\an}, i_X^*\scrM)$$
            In turn, this implies that the pullback functor:
                $$i_X^*: \scrO_X\mod \to \scrO_{X^{\an}}\mod$$
            is fully faithful on top of being exact, and since both of its domain and codomain are abelian categories, this implies in particular that short exact sequences are reflected. 
        \end{corollary}

        We shall now see that when we restrict our attention to coherent modules only, the module analytification functor $i_X^*$ (for any locally finite-type $\bbC$-scheme $X$) will furthermore be essentially surjective, thus giving rise to an adjoint equivalence:
            $$i_X^*: \Coh(X) \leftrightarrows \Coh(X^{\an}): i_{X *}$$
        between the categories of coherent modules on $X$ and on $X^{\an}$, with the latter being given as the category of coherent modules on the ringed space $(X^{\an}, \scrO_{X^{\an}})$ (cf. \cite[\href{https://stacks.math.columbia.edu/tag/01BU}{Tag 01BU}]{stacks}). The fact that:
            $$i_X^*: \scrO_X\mod \to \scrO_{X^{\an}}\mod$$
        is fully faithful means that, should its restriction down to coherent modules:
           $$i_X^*: \Coh(X) \to \Coh(X^{\an})$$
       be well-defined (cf. lemma \ref{lemma: absolute_analytification_of_coherent_modules}) then we will be able to exploit the compact generation of $\Coh(X)$ to see that the set of (compact) generators via finite colimits of $\Coh(X^{\an})$ contains that of $\Coh(X)$ as a subset. The proof of essential surjectivity then reduces down to a proof of essential surjectivity on these generators. 
        \begin{remark}[A few reminders on coherent modules on ringed spaces]
            Again, we refer the reader to \cite[\href{https://stacks.math.columbia.edu/tag/01BU}{Tag 01BU}]{stacks} for a more detailed discussion, but the following list of properties is important enough for our purposes to warrant at least a mention. Suppose for a moment that $(X, \scrO_X)$ is a general ringed space and write $\Coh(X)$ to denote the category of coherent $\scrO_X$-modules.
            \begin{itemize}
                \item $\Coh(X)$ is an abelian subcategory of $\scrO_X\mod$, and this is somewhat interesting, seeing how $\QCoh(X)$ is not generally even abelian for ringed spaces, unlike for schemes where quasi-coherent modules are extremely well-behaved (cf. theorem \ref{theorem: qcoh_homological_properties})
                \item In fact, $\Coh(X)$ is closed under all extensions/short exacct sequences, and thus is a Serre subcategory of $\scrO_X\mod$ by definition.
                \item $\Coh(X)$ has enough injectives, and said injective objects are flasque. 
                \item An $\scrO_X$-module is coherent if and only if it is finitely presented.
                \item If:
                    $$f: X \to Y$$
                is a morphism between general ringed spaces and if $\scrN$ is a coherent $\scrO_Y$-module, then we will not usually be guaranteed that the pullback $f^*\scrN$ is coherent over $\scrO_X$. When $f$ is proper, though, and if $\scrM$ is some coherent $\scrO_X$-module, then the pushforward $f_*\scrM$ will in fact be coherent (cf. lemma \ref{lemma: pushforwards_of_analytic_coherent_modules}), and this is one of the reasons why having the cohomological base-change formula as in theorem \ref{theorem: relative_analytification_of_sheaves_of_modules} is important for our purposes!

                That said, there is a well-defined pullback functor:
                    $$f^*: \scrO_Y\mod^{\ft} \to \scrO_X\mod^{\ft}$$
                between the categories of finitely generated/finite-type $\scrO_Y$- and $\scrO_X$-modules. The issue mentioned above stems from the fact that the pullback of a finitely presented module is only finitely generated in general.
            \end{itemize}
            
            These properties will from now on be used without explicit mention.
        \end{remark}
        \begin{lemma}[Pushforwards of analytic coherent modules] \label{lemma: pushforwards_of_analytic_coherent_modules}
            Suppose that:
                $$f: \calX \to \calY$$
            is a morphism of complex-analytic spaces. If $f$ is proper then there will be a well-defined t-exact functor:
                $$Rf_*: D^+(\Coh(\calX)) \to D^+(\Coh(\calY))$$
        \end{lemma}
            \begin{proof}
                
            \end{proof}
        \begin{lemma}[Absolute analytification of coherent modules] \label{lemma: absolute_analytification_of_coherent_modules}
            Let $X$ be a locally finite-type $\bbC$-scheme. Then, there is a well-defined functor:
                $$i_X^*: \Coh(X) \to \Coh(X^{\an})$$
            (i.e. the pullback functor $i_X^*: \scrO_X\mod \to \scrO_{X^{\an}}\mod$ in particular does send coherent $\scrO_X$-modules to coherent $\scrO_{X^{\an}}$-modules).
        \end{lemma}
            \begin{proof}[Sketch]
                $i_X^*$ is exact (cf. corollary \ref{coro: GAGA_cohomological_comparison}), so it preserves compactness of objects. 
            \end{proof}
        \begin{theorem}[Relative analytification of coherent modules] \label{theorem: relative_analytification_of_coherent_modules}
            Suppose that:
                $$f: X \to Y$$
            is a morphism between schemes locally of finite type over $\Spec \bbC$. If $f$ is proper, then the canonical natural transformation:
                $$i_Y^* \circ R f_* \Rightarrow R f^{\an}_* \circ i_X^*$$
            will be a natural isomorphism of t-exact functors $D^+(\Coh(X)) \to D^+(\Coh(Y^{\an}))$.

            When $Y \cong \Spec \bbC$, the above implies that:
                $$H^{\bullet}(X, \scrM) \cong H^{\bullet}(X^{\an}, i_X^*\scrM)$$
            are finite-dimensional $\bbC$-vector spaces for any coherent $\scrO_X$-module $\scrM$.
        \end{theorem}
        \begin{remark}[\textit{D\'evissage}]
            Let us recall Chow's Lemma, which says that should $S$ be a Noetherian base scheme and $\pi: X \to S$ be a proper $S$-scheme, then there will exist a projective $S$-scheme $\pi': X' \to S$ along with a morphism of $S$-schemes:
                $$f: X' \to X$$
            for which there is a \textit{dense} open subscheme $U \subset X$ such that:
                $$X' \x_{f, X} U \cong U$$
            If, in addition, both $X$ and $X'$ are irreducible then $f$ will be birational. Furthermore, if $X$ is reduced, irreducible, or integral, then the same can be assumed for $X'$; in particular, this means that if $X$ is a variety (i.e. when all those adjectives are satisfied and $S$ is the spectrum of a field) then we can assume without loss of generality that $X'$ too is a variety over the same field.

            Using Chow's Lemma, and letting $S := \Spec \bbC$, we see that any proper algebraic $\bbC$-variety $Y$ is birationally equivalent to a projective $\bbC$-variety $X$, for which there is some closed immersion:
                $$j_X: X \hookrightarrow \P^n_{\bbC}$$
            We know that the abelian category:
                $$\Coh(\P^n_{\bbC})$$
            is generated via finite colimits by finitely many (compact) objects (namely Serre's twisting line bundles)\footnote{This statement remains true when we replace $\bbC$ with an arbitrary commutative ring.}, 
        \end{remark}

        \begin{theorem}[Compact generation of coherent modules over analytifications] \label{theorem: compact_generation_of_coherent_modules_over_analytifications}
            For any locally finite-type proper $\bbC$-scheme $X$, any coherent $\scrO_{X^{\an}}$-module $\scrM$ admits a 
        \end{theorem}
            \begin{proof}
                
            \end{proof}
        The following is a corollary to a combination of corollary \ref{coro: GAGA_cohomological_comparison} and theorem \ref{theorem: compact_generation_of_coherent_modules_over_analytifications}.
        \begin{corollary}[Serre's GAGA]
            For any locally finite-type proper $\bbC$-scheme $X$, there is an adjoint equivalence of categories:
                $$i_X^*: \Coh(X) \leftrightarrows \Coh(X^{\an}): i_{X *}$$
        \end{corollary}