\section{Cotangent complexes}
    \subsection{Cotangent complexes in classical algebraic geometry}
        \subsubsection{K\"ahler differentials} \label{subsubsection: kahler_differentials}
            \begin{definition}[K\"ahler differentials] \label{def: kahler_differentials}
                Let $R$ be a commutative ring and let $\varphi: R \to S$ be a commutative $R$-algebra. An \textbf{$R$-derivation} (or simply \textbf{derivation} when the base ring $R$ is understood from context) from $S$ into an $S$-module $N$ is an $R$-linear map:
                    $$d: S \to N$$
                such that $d(\varphi(a)) = 0$ for all $a \in R$ and such that $d(fg) = fd(g) + d(f)gs$ for all $f, g \in S$.
            \end{definition}
            \begin{lemma}[Modules of derivations] \label{lemma: modules_of_derivations}
                Let $R$ be a commutative ring and let $\varphi: R \to S$ be a commutative $R$-algebra. The set $\Der_R(S, N)$ of all $R$-derivations from $S$ into a fixed $S$-module $N$ thus carries a natural $S$-module structure. Furthermore, the assignment:
                    $$\Der_R(S, -): S\mod \to S\mod$$
                is a covariant functor which is represented by an $S$-module $\Omega^1_{S/R}$, generated by the symbols $d(f)$ (for all $f \in S$) and determined by the relations $d(f) + d(g) = d(f + g), fd(g) + d(f)g = d(fg), d(\varphi(a)) = 0$ (for all $f, g \in S$ and all $a \in R$).
            \end{lemma}
                \begin{proof}
                    That $\Der_R(S, N)$ is trivial, so let us focus on showing that there is a well-defined functor $\Der_R(S, -): S\mod \to S\mod$ that is naturally isomorphic to $\Hom_S(\Omega^1_{S/R}, -)$. 
                \end{proof}
            \begin{theorem}[Universal property of K\"ahler differentials] \label{theorem: kahler_differentials_universal_property}
                Let $R$ be a commutative ring. Then, there is an adjunction between $\Omega^1_{-/R}: {}^{R/}\Comm\Alg \to $ 
            \end{theorem}
            
            \begin{remark}[K\"ahler differentials and colimits] \label{remark: differentials_and_colimits}
                This could be viewed as a corollary to theorem \ref{theorem: kahler_differentials_universal_property}. 
                
                Let $R$ be a base commutative ring and let $\{S_i\}_{i \in I}$ be a diagram of $R$-algebras $S_i$. Then, due to $\Omega^1_{-/R}$ being a left-adjoint (which means, in particular, that it would preserve colimits \textit{a priori}), one has the following identity:
                    $$\Omega^1_{\underset{i \in I}{\colim} S_i/R} \cong \underset{i \in I}{\colim} \Omega^1_{S_i/R}$$
            \end{remark}
            \begin{example}[K\"ahler differentials and localisations] \label{example: differentials_and_localisations}
                An example of a colimit of an infinite diagram of modules of K\"ahler differentials is how these modules interact with localisations of commutative rings. Let $S$ be a (possibly infinite) commutative ring let $\q \in |\Spec S|$ be a prime ideal thereof, and let $R \to S$ be a ring map. Then:
                    $$\Omega^1_{S_{\q}/R} \cong (\Omega^1_{S/R})_{\q}$$
                Note that this exhibits the commutativity of $\Omega^1_{-/R}$ with an \textit{infinite} colimit because:
                    $$S_{\q} \cong \underset{y \in S \setminus \q}{\colim} S[1/y]$$
            \end{example}
            Let us examine how K\"ahler differentials interact with colimits a bit closer through the following proposition, wherein we rely on the fact that finite colimits can be constructed out of finite coproducts and epimorphisms.
            \begin{proposition}[K\"ahler differentials and finite colimits] \label{prop: differentials_and_finite_colimits}
                Let $R$ be a base commutative ring. 
                    \begin{enumerate}
                        \item \textbf{(Module of differentials of a surjection):} If $R \to S$ is a surjective ring homomorphism, then:
                            $$\Omega^1_{S/R} \cong 0$$
                        \item \textbf{(Module of differentials and base change):} Consider a pushout diagram of commutative rings such as the following one:
                            $$
                                \begin{tikzcd}
                                	{S'} & {R'} \\
                                	S & R
                                	\arrow[from=2-2, to=2-1]
                                	\arrow[from=2-1, to=1-1]
                                	\arrow[from=2-2, to=1-2]
                                	\arrow[from=1-2, to=1-1]
                                	\arrow["\lrcorner"{anchor=center, pos=0.125}, draw=none, from=1-1, to=2-2]
                                \end{tikzcd}
                            $$
                        Then:
                            $$\Omega^1_{S'/R} \cong \Omega^1_{S/R} \oplus \Omega^1_{R'/R}$$
                    \end{enumerate}
            \end{proposition}
                \begin{proof}
                    \noindent
                    \begin{enumerate}
                        \item \textbf{(Module of differentials of a surjection):} 
                            \begin{enumerate}
                                \item First of all, we claim that $\Omega^1_{R/R} \cong 0$. To see why this is the case, recall firstly that $R$ is the initial object of ${}^{R/}\Comm\Alg$, the category of internal commutative and unital algebras in $R\mod$. The universal property of the initial object as the colimit of the empty diagram as well as the fact that $\Omega^1_{-/R}$ is a left-adjoint, then jointly imply that $\Omega^1_{R/R}$ must be initial in $R\mod$. Lastly, recall that $0$ is intial in $R\mod$: this implies that $\Omega^1_{R/R} \cong 0$. Note that this is a special case of $\Omega^1_{S/R} \cong 0$ whenever $R \to S$ is surjective, because the zero object $0 \in R\mod$ is also terminal (and also because $R\mod$ is an abelian category).
                                \item By remark \ref{remark: differentials_and_colimits}, there exists a surjective $R$-module homomorphism:
                                    $$\Omega^1_{R/R} \to \Omega^1_{S/R}$$
                                and because $\Omega^1_{R/R} \cong 0$, we can thus deduce that:
                                    $$\Omega^1_{S/R} \cong 0$$
                                from the universal property of the zero object $0$ in $R\mod$ as the limit of the empty diagram.
                            \end{enumerate}
                        \item \textbf{(Module of differentials and base change):} This is completely trivial.
                    \end{enumerate}
                \end{proof}
                
            \begin{lemma}[Surjections between modules of differentials] \label{lemma: surjections_between_modules_of_differentials}
                Consider the following commutative diagram in $\Cring$:
                    $$
                        \begin{tikzcd}
                        	{S'} & {R'} \\
                        	S & R
                        	\arrow[from=2-1, to=1-1]
                        	\arrow[from=2-2, to=1-2]
                        	\arrow[from=1-2, to=1-1]
                        	\arrow[from=2-2, to=2-1]
                        \end{tikzcd}
                    $$
                Should the arrow $S \to S'$ be surjective, then the naturally induced $S$-module homomorphism $\Omega^1_{S/R} \to \Omega^1_{S'/R'}$ shall also be surjective.
            \end{lemma}
                \begin{proof}
                    First of all, the $S$-module homomorphism $\Omega^1_{S/R} \to \Omega^1_{S'/R'}$ is well-defined as it comes from evaluating the natural transformation $\Omega^1_{-/R} \to \Omega^1_{-/R'}$ along the arrow $S \to S'$ in the following manner:
                        $$
                            \begin{tikzcd}
                            	& {} & {\Omega^1_{S'/R'}} & 0 \\
                            	{\Omega^1_{S'/R}} & {\Omega^1_{R'/R}} \\
                            	{\Omega^1_{S/R}} & 0
                            	\arrow[from=3-2, to=2-2]
                            	\arrow[from=3-2, to=3-1]
                            	\arrow[from=3-1, to=2-1]
                            	\arrow[from=2-2, to=2-1]
                            	\arrow[from=1-4, to=1-3]
                            	\arrow[from=2-1, to=1-3]
                            	\arrow[from=2-2, to=1-4]
                            \end{tikzcd}
                        $$
                    Next, note that the $S$-module homomorphism $\Omega^1_{S/R} \to \Omega^1_{S'/R}$ is trivially surjective via an application of proposition \ref{prop: differentials_and_finite_colimits}.
                \end{proof}
                
            \begin{proposition}[The canonical exact sequence] \label{prop: canonical_exact_sequence_of_differentials}
                For $A \to B \to C$ a composition of ring maps, there exists a canonically associated right-exact sequence of $C$-modules:
                    $$C \tensor_B \Omega^1_{B/A} \to \Omega^1_{C/A} \to \Omega^1_{C/B} \to 0$$
            \end{proposition}
                \begin{proof}
                    First of all, the morphisms $A \to B \to C$ gives rise to a natural transformations:
                        $$\Omega^1_{-/A} \to \Omega^1_{-/B} \to \Omega^1_{-/C}$$
                    In particular, this tells us that there are the following canonically defined commutative diagrams:
                        $$\Omega^1_{B/A} \to \Omega^1_{B/B}$$
                        $$\Omega^1_{C/A} \to \Omega^1_{C/B} \to \Omega^1_{C/C}$$
                    Second of all, recall that we know by proposition \ref{prop: differentials_and_finite_colimits} that:
                        $$\Omega^1_{B/B} \cong 0$$
                        $$\Omega^1_{C/C} \cong 0$$
                    Thus, there exists the following canonical commutative diagram of $C$-modules:
                        $$
                            \begin{tikzcd}
                            	{C \tensor_B \Omega^1_{B/A}} & {\Omega^1_{C/A}} & 0 \\
                            	0 & {\Omega^1_{C/B}} & 0
                            	\arrow[from=1-2, to=2-2]
                            	\arrow[from=1-1, to=2-1]
                            	\arrow[from=2-1, to=2-2]
                            	\arrow[from=1-1, to=1-2]
                            	\arrow[from=2-2, to=2-3]
                            	\arrow["{!}", from=1-2, to=1-3]
                            	\arrow[from=1-3, to=2-3]
                            \end{tikzcd}
                        $$
                    wherein:
                        \begin{itemize}
                            \item the horizontal arrows exist as a consequence of $C \tensor_B -: B\mod \to C\mod$ being a left-adjoint
                            \item $!: \Omega^1_{C/A} \to 0$ is the canonical terminal arrow, and
                            \item the arrows $0 \to \Omega^1_{C/B}$ and $\Omega^1_{C/B} \to 0$ are actually $C \tensor_B \Omega^1_{B/B} \to \Omega^1_{C/B}$ and $\Omega^1_{C/B} \to \Omega^1_{C/C}$, respectively.
                        \end{itemize}
                    An application of lemma \ref{lemma: surjections_between_modules_of_differentials} to the square:
                        $$
                            \begin{tikzcd}
                            	C & B \\
                            	C & A
                            	\arrow["{\id_C}", from=2-1, to=1-1]
                            	\arrow[from=2-2, to=1-2]
                            	\arrow[from=1-2, to=1-1]
                            	\arrow[from=2-2, to=2-1]
                            \end{tikzcd}
                        $$
                    (note that the identity morphism $\id_C: C \to C$ is trivially surjective) then helps us show the surjectivity of the map $\Omega^1_{C/A} \to \Omega^1_{C/B}$. This concludes the proof.
                \end{proof}
                
        \subsubsection{Cotangent complexes}

    \subsection{Cotangent complexes in derived algebraic geometry}